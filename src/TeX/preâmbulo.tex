\documentclass[
  fontsize=12pt,
  answers,
  addpoints,
  pagesize,
  paper=landscape,
  a5,
  DIV=calc,
  headsepline,
  footsepline,
  abstract=true,
  bibliography=totoc
]{kommaexam}

\usepackage[margin=.5cm]{geometry}

\usepackage[brazilian]{babel}
\usepackage{csquotes}

\usepackage{mathpazo}
\usepackage{fontspec}
\setmonofont{Fira Code}[Contextuals=Alternate]
\setmainfont{TeX Gyre Pagella}

\usepackage{mathtools,amssymb,amsthm}

\usepackage[svgnames]{xcolor}
\usepackage[useregional]{datetime2}

\usepackage{pdfpages}
\usepackage{svg,graphicx}
\usepackage{caption,subcaption}
\usepackage[shortlabels]{enumitem}
\graphicspath{ {img/} }

\usepackage{background}
\usepackage{tikz}
\usepackage{booktabs}
\usepackage{bookmark}

\usepackage{imakeidx}
\makeindex[
	columns=2,
	title=Índice,
	options= -s header.ist,
	intoc
]

\usepackage[
  citestyle=numeric,
  style=numeric,
  backend=biber,
  maxnames=5,
  minnames=3
]{biblatex}

\addbibresource{referências.bib}

\usepackage{hyperref}

\hypersetup{
  pdfinfo={
      Title={Álgebra linear~Verão 2021~Notas das aulas na escola},
      Author={Alex Dantas et al.},
      Keywords={corpos, sistemas lineares},
      Subject={Álgebra linear},
      Producer={TeXstudio 3.0.4 Using Qt Version 5.15.2},
      Creator={LuaHBTeX, Version 1.12.0 (TeX Live 2020/Arch Linux)},
    }
  %	hyperref,
  pdfencoding=auto,
  linktocpage=true,
  colorlinks=true,
  linkcolor=NavyBlue,
  urlcolor=magenta,
  linktoc=all,
  pdfpagelabels,
  bookmarks=true,
  unicode,
  pdfborder = {0 0 0}
  %	filecolor = red,
}

\backgroundsetup{
  scale=1,
  angle=0,
  opacity=.1,
  contents={
      \begin{tikzpicture}[remember picture,overlay]
        \node at ([yshift=11pt,xshift=5pt]current page.center) {\includegraphics[width=8.5cm]{unb}};
      \end{tikzpicture}}
}

\theoremstyle{definition}
\newtheorem{definition}{Definição}[chapter]
\newtheorem{proposition}{Proposição}[chapter]
\newtheorem{example}{Exemplo}[chapter]
\newtheorem{exercise}{Exercício}[chapter]

\newcommand{\MVAt}{{\usefont{U}{mvs}{m}{n}\symbol{`@}}}
\renewcommand{\qedsymbol}{\(\blacksquare\)}

\setkomafont{title}{\bfseries\sffamily\Huge\color{DarkBlue}}
\setkomafont{subtitle}{\bfseries\sffamily\Large\color{DarkBlue}}
\setkomafont{author}{\bfseries\sffamily\large\color{DarkBlue}}
\setkomafont{date}{\bfseries\sffamily\color{DarkBlue}}
\setkomafont{chapter}{\LARGE\color{DarkBlue}}
\setkomafont{section}{\Large\color{DarkBlue}}
\setkomafont{subsection}{\large\color{DarkBlue}}

\title{Álgebra Linear II}

\subtitle{
  \href{https://www.mat.unb.br/verao2021/verao/verao_pt.html}{
    XLIX Escola de Verão en Matemática da UnB
  }
}

\titlehead{\Huge
\[
  J_{j}=
  \begin{pmatrix}
    \lambda_{j} & 1           &        &             & 0           \\
                & \lambda_{j} & 1      &             &             \\
                &             & \ddots & \ddots      &             \\
                &             &        & \lambda_{j} & 1           \\
    0           &             &        &             & \lambda_{j}
  \end{pmatrix}
  \in\mathbb{C}^{s_{j}\times s_{j}}.
\]
}

\author{Aulas do professor Alex Carrazedo Dantas\thanks{alex\MVAt mat.unb.br}}

\date{
  Última modificação: \today{} às $\DTMcurrenttime$.
  \\[\baselineskip]
  \url{https://carlosal1015.github.io/Algebra-linear-II/main.pdf}
}