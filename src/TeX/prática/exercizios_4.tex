\chapter{Exercícios de Fixação\quad$\left(29/01/2021\right)$}

\begin{questions}
	\question\label{exercício:4.1}

	Determine a forma de Jordan de
	$T\colon\mathbb{C}^{6}\to\mathbb{C}^{6}$ definida por
	\[
		T\left(x,y,z,w,t,k\right)=
		\left(2x,x+2y,-x+2z,y+2w,x+y+z+w+2t,t-k\right).
	\]

	\begin{solutionordottedlines}
		.
	\end{solutionordottedlines}

	\question\label{exercício:4.2}

	Encontre a forma de Jordan das seguinte matrizes

	\begin{parts}
		\begin{multicols}{4}
			\part\label{exercício:4.2a}

			\[
				\begin{bNiceMatrix}
					5  & -9  & -4 \\
					6  & -11 & -5 \\
					-7 & 13  & 6
				\end{bNiceMatrix}
			\]

			\part\label{exercício:4.2b}

			\[
				\begin{bNiceMatrix}
					0 & -9 & 0 & 0 \\
					1 & 6  & 0 & 0 \\
					0 & 0  & 3 & 0 \\
					0 & 0  & 0 & 3
				\end{bNiceMatrix}
			\]

			\part\label{exercício:4.2c}

			\[
				\begin{bNiceMatrix}
					3 & -1 & 1  & -7 \\
					9 & -3 & -7 & -1 \\
					0 & 0  & 4  & -8 \\
					0 & 0  & 2  & -4
				\end{bNiceMatrix}
			\]

			\part\label{exercício:4.2d}
			\NiceMatrixOptions{
				cell-space-top-limit = 3pt
			}
			\[
				\begin{bNiceMatrix}
					\overline{8} & \overline{12} & \overline{0} & \overline{0} \\
					\overline{4} & \overline{12} & \overline{0} & \overline{0} \\
					\overline{0} & \overline{0}  & \overline{9} & \overline{2} \\
					\overline{0} & \overline{0}  & \overline{2} & \overline{6}
				\end{bNiceMatrix}
			\]
			\NiceMatrixOptions{
				cell-space-top-limit = 0pt
			}
		\end{multicols}
		A matriz de (d) está sobre o corpo $\mathbb{Z}/13\mathbb{Z}$.
	\end{parts}

	\begin{parts}
		\part

		\part

		\part

		\part
	\end{parts}

	\question\label{exercício:4.3}

	Mostre que as seguintes matrizes são semelhantes:

	\[
		\begin{bNiceMatrix}
			-1 & 0 & 0 & -2 \\
			0  & 1 & 0 & 4  \\
			-1 & 0 & 1 & 1  \\
			0  & 0 & 0 & 1
		\end{bNiceMatrix}
		\quad
		\begin{bNiceMatrix}
			1  & 0 & 0 & 0  \\
			-1 & 1 & 0 & 0  \\
			0  & 1 & 1 & 0  \\
			0  & 0 & 0 & -1
		\end{bNiceMatrix}
	\]

	\begin{solutionordottedlines}
		.
	\end{solutionordottedlines}

	\question\label{exercício:4.4}

	Seja
	\begin{math}
		T\colon
		\mathcal{P}_{n}
		\left(\mathbb{R}\right)\to
		\mathcal{P}_{n}
		\left(\mathbb{R}\right)
	\end{math}
	dado por $T\left(p\left(x\right)\right)=p\left(x+1\right)$.

	\begin{parts}
		\part\label{exercício:4.4a}

		Determine a forma de Jordan de $T$;

		\part\label{exercício:4.4b}

		Para $n=4,$ encontre uma base $\beta$ de
		$\mathcal{P}_{n}\left(\mathbb{R}\right)$ tal que
		${\left[T\right]}_{\beta}$ seja a forma de Jordan de $T$.
	\end{parts}

	\begin{parts}
		\part

		\part
	\end{parts}

	\question\label{exercício:4.5}

	Seja $A$ uma matriz $9\times9$ cujo polinômio característico é
	$-{\left(x-3\right)}^{5}{\left(x-2\right)}^{4}$ e cujo polinômio
	minimal é $\left(x-3\right)^{3}{\left(x-2\right)}^{2}$.
	Dê as possíveis formas de Jordan de $A$.


	\begin{solutionordottedlines}
		.
	\end{solutionordottedlines}

	\question\label{exercício:4.6}

	Seja
	\begin{math}
		T\colon
		\mathcal{M}_{2\times2}
		\left(\mathbb{R}\right)\to
		\mathcal{M}_{2\times2}
		\left(\mathbb{R}\right)
	\end{math}
	um operador linear cuja matriz em relação à base
	\[
		\beta=\left\{
		\begin{bNiceMatrix}
			1 & 0 \\
			1 & 0
		\end{bNiceMatrix},
		\begin{bNiceMatrix}
			1 & 0 \\
			0 & 0
		\end{bNiceMatrix},
		\begin{bNiceMatrix}
			0 & 1 \\
			0 & 1
		\end{bNiceMatrix},
		\begin{bNiceMatrix}
			0 & 0 \\
			0 & 1
		\end{bNiceMatrix}
		\right\}
	\]
	é dada por
	\[
		[T]_{\beta}=
		\begin{bNiceMatrix}
			3 & -1 & 1  & -7 \\
			9 & -3 & -7 & -1 \\
			0 & 0  & 4  & -8 \\
			0 & 0  & 2  & -4
		\end{bNiceMatrix}.
	\]
	Determine uma matriz
	$M\in\mathcal{M}_{4\times4}\left(\mathbb{R}\right)$ tal que
	$M^{-1}{\left[T\right]}_{\beta}M$ é a forma de Jordan de
	${\left[T\right]}_{\beta}$.

	\begin{solutionordottedlines}
		.
	\end{solutionordottedlines}

	\question\label{exercício:4.7}

	Determine, a menos de semelhança, todas matrizes $3\times3$
	complexas $A$ tais que $A^{3}=I_{3}$.

	\begin{solutionordottedlines}
		.
	\end{solutionordottedlines}

	\question\label{exercício:4.8}

	Sejam $n\geq2$ um inteiro positivo e $B$ uma matriz $n\times n$
	sobre um corpo $\mathbb{F}$.
	Suponha que $B^{n}=0$ e $B^{n-1}\neq0$.
	Mostre que não existe uma $n\times n$ matriz $A$ tal que $A^{2}=B$.

	\begin{solutionordottedlines}
		.
	\end{solutionordottedlines}

	\question\label{exercício:4.9}

	Seja $A$ uma matriz $6\times6$ sobre $\mathbb{R}$ tal que
	$A^{4}-8A^{2}+16I=0$.
	Quais são as possíveis formas de Jordan não semelhantes de $A$?

	\begin{solutionordottedlines}
		.
	\end{solutionordottedlines}
\end{questions}
