\chapter{Exercícios de Fixação\quad$\left(15/01/2021\right)$}

\begin{questions}
  \question\label{exercício:2.1}

  Defina sobre $\mathbb{R}^{2}$ as seguintes operações:

  \[
    \begin{aligned}
      +\colon\mathbb{R}^{2}\times\mathbb{R}^{2}      & \longrightarrow\mathbb{R}^{2} \\
      \left(\left(x,y\right),\left(a,b\right)\right) & \longmapsto
      \left(x+a,0\right)
    \end{aligned}\qquad
    \begin{aligned}
      \cdot\colon\mathbb{R}\times\mathbb{R}^{2} & \longrightarrow\mathbb{R}^{2} \\
      \left(c,\left(x,y\right)\right)           & \longmapsto
      \left(cx,0\right)
    \end{aligned}
  \]

  O conjunto $\mathbb{R}^{2}$ é um espaço vetorial com essas operações?

  \begin{solutionordottedlines}
  \end{solutionordottedlines}

  \question\label{exercício:2.2}

  Defina sobre
  \begin{math}
    V=
    \left\{
    \left(x,y\right)\in
    \mathbb{R}^{2}\mid
    x>0,y>0
    \right\}
  \end{math}
  as seguintes operações:

  \[
    \begin{aligned}
      +\colon\mathbb{R}^{2}\times\mathbb{R}^{2}      & \longrightarrow\mathbb{R}^{2} \\
      \left(\left(x,y\right),\left(a,b\right)\right) & \longmapsto
      \left(xa,yb\right)
    \end{aligned}\qquad
    \begin{aligned}
      \cdot\colon\mathbb{R}\times\mathbb{R}^{2} & \longrightarrow\mathbb{R}^{2} \\
      \left(c,\left(x,y\right)\right)           & \longmapsto
      \left(x^{c},y^{c}\right)
    \end{aligned}
  \]
  Mostre que $V$ é um espaço vetorial com essas operações.

  \begin{solutionordottedlines}
  \end{solutionordottedlines}

  \question\label{exercício:2.3}

  Resolva:
  \begin{parts}
    \part\label{exercício:2.3a}

    O vetor $\left(3,-1,0,-1\right)$ pertence ao subespaço
    \begin{math}
      W=
      \left\langle
      \left(2,-1,3,2\right),
      \left(-1,1,1,3\right),
      \left(1,1,9,-5\right)
      \right\rangle
    \end{math}
    de $\mathbb{R}^{4}$?

    \part\label{exercício:2.3b}

    Determine uma base para o subespaço vertorial de $\mathbb{R}^{5}$
    das soluções do sistema linear homogêneo
    $\csysteme[xyzwt]{x-2y+z+w+t=0,3x+y-z-4t=0,2x+y-3z+w=0}$;

    \part\label{exercício:2.3c}

    Determine uma base para o subespaço vertorial de
    ${\left(\mathbb{Z}/5\mathbb{Z}\right)}^{5}$ das soluções do
    sistema linear homogêneo
    \begin{math}
      \csysteme[xyzwt]{
        x-\overline{2}y+z+w+t=\overline{0},
        \overline{2}x+y-z+t=\overline{0},
        \overline{3}x+y+\overline{3}z+w=\overline{0}
      }
    \end{math},
  \end{parts}

  \begin{solutionordottedlines}
    \begin{parts}
      \part

      .

      \part

      .

      \part

      .
    \end{parts}
  \end{solutionordottedlines}

  \question\label{exercício:2.4}

  Sejam $V$ um espaço vetorial sobre um corpo $F$ e $U$ e $W$
  subespaços de $V$ tais que $U+W=V$ e $U \cap W=\left\{0\right\}$.
  Mostre que cada vetor $v\in V$ é escrito de maneira única como
  $v=u+w$, onde $u\in U$ e $w\in W$.

  \begin{solutionordottedlines}
  \end{solutionordottedlines}

  \question\label{exercício:2.5}

  Mostre que o conjunto dos polinômios sobre uma variável com
  coeficientes em $\mathbb{R}$ é um espaço vetorial sobre
  $\mathbb{R}$ munido das operações usuais de soma e multiplicação
  por escalar.
  Determine uma base para esse espaço vetorial.

  \begin{solutionordottedlines}
  \end{solutionordottedlines}

  \question\label{exercício:2.6}

  Seja $S$ um subconjunto de um espaço vetorial $V$.
  Mostre que $S$ é LD se, e somente se, existir um vetor $v\in S$ que
  pode ser escrito como combinação linear dos elementos de
  $S\setminus\left\{v\right\}$.

  \begin{solutionordottedlines}
  \end{solutionordottedlines}

  \question\label{exercício:2.7}

  Considere o seguinte espaço vetorial sobre $\mathbb{R}$:

  \[
    \mathcal{P}_{3}
    \left(\mathbb{R}\right)=
    \left\{
    a+bx+cx^{2}+dx^{3}\mid
    a,b,c,d\in\mathbb{R}
    \right\}.
  \]

  \begin{parts}
    \part\label{exercício:2.7a}

    Mostre que $\alpha=\left\{1,2+x,3x-x^{2},x-x^{3}\right\}$ é uma
    base de $\mathcal{P}_{3}\left(\mathbb{R}\right)$;

    \part\label{exercício:2.7b}

    Escreva as coordenadas de $p\left(x\right)=1+x+x^{2}+x^{3}$ com
    relação a base $\alpha$;

    \part\label{exercício:2.7c}

    Determine as matrizes mudança de base
    ${\left[I\right]}_{\alpha}^{e}$ e
    ${\left[I\right]}_{e}^{\alpha}$, onde
    $e=\left\{1,x,x^{2},x^{3}\right\}$.
  \end{parts}

  \begin{solutionordottedlines}
    \begin{parts}
      \part

      .

      \part

      .

      \part

      .
    \end{parts}
  \end{solutionordottedlines}

  \question\label{exercício:2.8}

  Faça o que se pede:

  \begin{parts}
    \part\label{exercício:2.8a}

    Considere a função
    $T\colon\mathbb{C}\to\mathcal{M}_{2}\left(\mathbb{R}\right)$
    dada por

    \[
      T\left(x+yi\right)=
      \begin{bmatrix*}
        x+7y & 5y \\
        -10y & x-7y
      \end{bmatrix*}.
    \]

    Moste que $T$ é uma transformação linear.
    Prove que
    \begin{math}
      T\left(z_{1}z_{2}\right)=
      T\left(z_{1}\right)
      T\left(z_{2}\right),
      \forall z_{1},z_{2}\in\mathbb{C}
    \end{math};

    \part\label{exercício:2.8b}

    Mostre que a composta de transformações lineares é uma transformação linear;

    \part\label{exercício:2.8c}

    Mostre que uma função $T\colon\mathbb{F}^{n}\to\mathbb{F}$ é uma
    transformação linear se, e somente se, existem escalares
    $c_{1},\dotsc,c_{n}$ no corpo $\mathbb{F}$ tais que

    \[
      T
      \left(x_{1},\dotsc,x_{n}\right)=
      c_{1}x_{1}+\dotsc+c_{n}x_{n}.
    \]
  \end{parts}

  \begin{solutionordottedlines}
    \begin{parts}
      \part

      .

      \part

      .

      \part

      .
    \end{parts}
  \end{solutionordottedlines}

  \question\label{exercício:2.9}

  Faça o que se pede:

  \begin{parts}
    \part\label{exercício:2.9a}

    Considere $\mathbb{R}^{4}$ e seus subespaços
    \begin{math}
      W=
      \left\langle
      \left(1,0,1,1\right),
      \left(0,-1,-1,-1\right)
      \right\rangle
    \end{math}
    e
    \begin{math}
      U=
      \left\{
      \left(x,y,z,w\right)\in
      \mathbb{R}^{4}\mid
      x+y=0,
      z+t=0
      \right\}
    \end{math}.
    Determine uma transformação linear
    $T\colon\mathbb{R}^{4}\to\mathbb{R}^{4}$ tal que
    $\operatorname{Nuc}\left(T\right)=U$ e
    $\operatorname{Im}\left(T\right)=W$;

    \part\label{exercício:2.9b}

    Considere ${\left(\mathbb{Z}/5\mathbb{Z}\right)}^{4}$ e seus
    subespaços
    \begin{math}
      W=
      \left\langle
      \left(
      \overline{1},
      \overline{0},
      \overline{1},
      \overline{1}
      \right),
      \left(
      \overline{0},
      \overline{4},
      \overline{4},
      \overline{4}
      \right)
      \right\rangle
    \end{math}
    e
    \begin{math}
      U=
      \left\{
      \left(x,y,z,w\right)\in
      {\left(\mathbb{Z}/5\mathbb{Z}\right)}^{4}\mid
      x+y=
      \overline{0},
      z+t=
      \overline{0}
      \right\}
    \end{math}.
    Determine uma transformação linear
    $T\colon\mathbb{R}^{4}\to\mathbb{R}^{4}$ tal que
    $\operatorname{Nuc}\left(T\right)=V$ e
    $\operatorname{Im}\left(T\right)=W$;

    \part\label{exercício:2.9c}

    Determine uma base para o núcleo e uma base para a imagem da
    transformação linear $T\colon\mathbb{C}^{2}\to\mathbb{R}^{2}$
    dada por $T\left(x+yi,a+bi\right)=\left(x+2a,-x+2b\right)$.
  \end{parts}

  \begin{solutionordottedlines}
    \begin{parts}
      \part

      .

      \part

      .
      \part

      .
    \end{parts}
  \end{solutionordottedlines}

  \question\label{exercício:2.10}

  Sejam $V$ e $W$ espaços vetoriais sobre um mesmo corpo $\mathbb{F}$
  e $T\colon V\to W$.
  Mostre que $T$ é injetora se, e somente se, $T$ leva subconjunto LI
  em subconjunto LI.

  \begin{solutionordottedlines}
  \end{solutionordottedlines}

  \question\label{exercício:2.11}

  Seja
  $T\colon\mathbb{C}^{3}\to\mathcal{P}_{2}\left(\mathbb{C}\right)$
  a transformação linear definida por
  $T\left(1,0,0\right)=1+ix^{2}$,
  $T\left(0,1,0\right)=x+x^{2}$ e
  $T\left(0,0,1\right)=i+x$.
  Exiba uma fórmula para $T$ e decida se $T$ é um isomorfismo.

  \begin{solutionordottedlines}
  \end{solutionordottedlines}

  \question\label{exercício:2.12}

  Seja $F$ um corpo e $T\colon\mathbb{F}^{2}\to\mathbb{F}^{2}$ dada
  por
  \begin{math}
    T\left(x,y\right)=
    \left(x+y,x\right),
    \forall\left(x,y\right)\in
    \mathbb{F}^{2}
  \end{math}.
  Mostre que $T$ é um isomorfismo e exiba uma fómula para $T^{-1}$.

  \begin{solutionordottedlines}
  \end{solutionordottedlines}

  \question\label{exercício:2.13}

  Considere as bases $\alpha=\left\{1,1+x,1+x^{2}\right\}$ de
  $\mathcal{P}_{2}\left(\mathbb{R}\right)$ e
  \begin{math}
    \beta=
    \left\{
    \left(1,0\right),
    \left(i,0\right),
    \left(1,1\right),
    \left(1,i\right)
    \right\}
  \end{math}
  de $\mathbb{C}^{2}$ como espaços vetoriais sobre $\mathbb{R}$.
  Determine as coordenadas da transformação linear
  $T\colon\mathcal{P}_{2}\left(\mathbb{R}\right)\to\mathbb{C}^{2}$ dada
  por $T\left(a+bx+cx^{2}\right)=\left(a+bi,b+ci\right)$ com relação
  à base de
  \begin{math}
    L
    \left(
    \mathcal{P}_{2}\left(\mathbb{R}\right),
    \mathbb{C}^{2}
    \right)
  \end{math}
  construída no Teorema 2-(ii) da Aula $9$.

  \begin{solutionordottedlines}
  \end{solutionordottedlines}

  \question\label{exercício:2.14}

  Considere a base
  \begin{math}
    \alpha=
    \left\{
    \left(1,0,-1\right),
    \left(1,1,1\right),
    \left(2,2,0\right)
    \right\}
  \end{math}
  de $\mathbb{C}^{3}$ como espaço vetorial sobre $\mathbb{C}$.
  Determine a base dual $\alpha^{\ast}$ de
  ${\left(\mathbb{C}^{3}\right)}^{\ast}$.

  \begin{solutionordottedlines}
  \end{solutionordottedlines}

  \question\label{exercício:2.15}

  Considere $T\colon\mathbb{R}^{2}\to\mathbb{R}^{3}$ dada por
  $T\left(x,y\right)=\left(2x+3y,y-x,3x\right)$ e as bases
  $\alpha=\left\{\left(1,2\right),\left(2,-1\right)\right\}$ de
  $\mathbb{R}^{2}$ e
  \begin{math}
    \beta=
    \left\{
    \left(1,1,1\right),
    \left(0,1,1\right),
    \left(0,0,1\right)
    \right\}
  \end{math}
  de $\mathbb{R}^{3}$.

  Calcule ${\left[T\right]}_{\beta}^{\alpha}$,
  ${\left[T\right]}_{\beta}^{e_{1}}$ e
  ${\left[T\right]}_{e_{2}}^{\alpha}$ onde $e_{1}$ é a base canônica
  de $\mathbb{R}^{2}$ e $e_{2}$ é a base canônica de
  $\mathbb{R}^{3}$.

  \begin{solutionordottedlines}
  \end{solutionordottedlines}

  \question\label{exercício:2.16}

  Sejam
  $T\colon\mathbb{R}^{3}\to\mathcal{P}_{2}\left(\mathbb{R}\right)$ e
  $G\colon\mathcal{P}_{2}\left(\mathbb{R}\right)\to\mathbb{R}^{3}$
  transformações lineares tais que
  \[
    {\left[T\right]}_{\beta}^{\alpha}=
    \begin{bmatrix*}
      1 & 2 & -1 \\
      1 & 0 & -1 \\
      0 & -1 & 0
    \end{bmatrix*}\quad
    \text { e }\quad
    {\left[G\right]}_{\alpha}^{\beta}=
    \begin{bmatrix*}
      1 & 1 & 2 \\
      1 & -1 & 0 \\
      -1 & 1 & 2
    \end{bmatrix*}
  \]
  onde
  \begin{math}
    \alpha=
    \left\{
    \left(1,1,0\right),
    \left(0,1,0\right),
    \left(0,0,1\right)
    \right\}
  \end{math}
  é base de $\mathbb{R}^{3}$ e $\beta=\left\{1,1+x,1+x^{2}\right\}$ é
  base de $\mathcal{P}_{2}\left(\mathbb{R}\right)$.
  Determine bases para
  \begin{math}
    \operatorname{Nuc}
    \left(T\right),
    \operatorname{Im}
    \left(T\right),
    \operatorname{Nuc}
    \left(G\circ T\right)
  \end{math}
  e $\operatorname{Im}\left(G\circ T\right)$.

  \begin{solutionordottedlines}
  \end{solutionordottedlines}

  \question\label{exercício:2.17}

  Seja
  \begin{math}
    T\colon
    \mathcal{M}_{2\times2}
    \left(\mathbb{C}\right)\to
    \mathcal{M}_{2\times2}
    \left(\mathbb{C}\right)
  \end{math}
  a transformação linear definida por
  \[
    T
    \begin{bmatrix*}
      x & y \\
      z & w
    \end{bmatrix*}
    =
    \begin{bmatrix*}
      0 & x \\
      z-w & 0
    \end{bmatrix*}.
  \]

  \begin{parts}
    \part\label{exercício:2.17a}

    Determine a matriz $\left[T\right]$ de $T$ com relação à base canônica $e$;

    \part\label{exercício:2.17b}

    Determine a matriz de $T$ com relação à base
    \[
      \alpha=
      \left\{
      \begin{bmatrix*}
        1 & 0 \\
        0 & 1
      \end{bmatrix*},
      \begin{bmatrix*}
        0 & 1 \\
        1 & 0
      \end{bmatrix*},
      \begin{bmatrix*}
        1 & 0 \\
        1 & 1
      \end{bmatrix*},
      \begin{bmatrix*}
        0 & 1 \\
        0 & 1
      \end{bmatrix*}
      \right\};
    \]

    \part\label{exercício:2.17c}

    Exiba a matriz $M$ tal que ${\left[T\right]}_{\beta}=M^{-1}\left[T\right]M$.
  \end{parts}

  \begin{solutionordottedlines}
    \begin{parts}
      \part


      \part


      \part

    \end{parts}
  \end{solutionordottedlines}

  \question\label{exercício:2.18}

  Seja
  \begin{math}
    T\colon
    \mathcal{M}_{2\times2}
    \left(\mathbb{Z}/7\mathbb{Z}\right)\to
    \mathcal{M}_{2\times2}
    \left(\mathbb{Z}/7\mathbb{Z}\right)
  \end{math}
  a transformação linear definida por
  \[
    T
    \begin{bmatrix*}
      x & y \\
      z & w
    \end{bmatrix*}
    =
    \begin{bmatrix*}
      \overline{0} & x \\
      z+\overline{6}w & \overline{0}
    \end{bmatrix*}.
  \]

  \begin{parts}
    \part\label{exercício:2.18a}

    Determine a matriz $[T]$ de $T$ com relação à base canônica $e$;

    \part\label{exercício:2.18b}

    Determine a matriz de $T$ com relação à base

    $$
      \alpha=\left\{
      \begin{bmatrix*}
        \overline{1} & \overline{0} \\
        \overline{0} & \overline{1}
      \end{bmatrix*},
      \begin{bmatrix*}
        \overline{0} & \overline{1} \\
        \overline{1} & \overline{0}
      \end{bmatrix*},
      \begin{bmatrix*}
        \overline{1} & \overline{0} \\
        \overline{1} & \overline{1}
      \end{bmatrix*},
      \begin{bmatrix*}
        \overline{0} & \overline{1} \\
        \overline{0} & \overline{1}
      \end{bmatrix*}.
      \right\}
    $$

    \part\label{exercício:2.18c}

    Exiba a matriz $M$ tal que ${\left[T\right]}_{\beta}=M^{-1}\left[T\right]M$.
  \end{parts}

  \question\label{exercício:2.19}

  Seja
  \begin{math}
    T\colon
    \mathcal{M}_{2\times2}
    \left(\mathbb{Z}/7\mathbb{Z}\right)\to
    \mathcal{M}_{2\times2}
    \left(\mathbb{Z}/7\mathbb{Z}\right)
  \end{math}
  a transformação linear definida por
  $$
    T
    x y \\
    z w
    =
    \overline{0}  x \\
    z+\overline{6} w \overline{0}
  $$

  \begin{parts}
    \begin{multicols}{3}
      \part\label{exercício:2.19a}

      Determine a matriz $\left[T\right]$ de $T$ com relação à base
      canônica $e$;

      \part\label{exercício:2.19b}

      Determine a matriz de $T$ com relação à base
      $$
        \alpha=
        \left\{
        \overline{1} \overline{0} \\
        \overline{0} \overline{1}
        ,
        \overline{0} \overline{1} \\
        \overline{1} \overline{0}
        ,
        \overline{1} \overline{0} \\
        \overline{1} \overline{1}
        \overline{0} \overline{1} \\
        \overline{0} \overline{1}
        \right\}
      $$

      \part\label{exercício:2.19c}

      Exiba a matriz $M$ tal que
      ${\left[T\right]}_{\beta}=M^{-1}\left[T\right]M$.
    \end{multicols}
  \end{parts}

  \begin{solutionordottedlines}
    \begin{parts}
      \part


      \part


      \part

    \end{parts}
  \end{solutionordottedlines}

  \question\label{exercício:2.20}

  Seja $T\colon\mathbb{Q}^{3}\to\mathbb{Q}^{3}$ uma transformação linear
  cuja matriz com relação à base canônica seja
  $$
    1 1 0 \\
    -1 0 1 \\
    0 -1 -1
  $$

  \begin{parts}
    \begin{multicols}{3}
      \part\label{exercício:2.20a}

      Determine $T\left(x,y,z\right)$;

      \part\label{exercício:2.20b}

      Qual é a matriz do operador linear $T$ com relação à base
      \begin{math}
        \alpha=
        \left\{
        \left(-1,1,0\right),
        \left(1,-1,1\right),
        \left(0,1,-1\right)
        \right\}
      \end{math}?

      \part\label{exercício:2.20c}

      O operador $T$ é invertível? Justifique!
    \end{multicols}
  \end{parts}


  \begin{solutionordottedlines}
    \begin{parts}
      \part


      \part


      \part

    \end{parts}
  \end{solutionordottedlines}
\end{questions}