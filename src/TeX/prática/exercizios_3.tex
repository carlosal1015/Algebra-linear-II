\chapter{Exercícios de Fixação\quad$\left(27/01/2021\right)$}

\begin{questions}
	\question\label{exercício:3.1}

	Mostre que $T\colon\mathbb{R}^{3}\to\mathbb{R}^{3}$ definida por
	$T\left(x,y,z\right)=\left(x+2y-z,-2x-3y+z,2 x+2y-2z\right)$ é
	diagonalizável.

	\begin{solutionordottedlines}
		.
	\end{solutionordottedlines}

	\question\label{exercício:3.2}

	Em cada um dos casos abaixo, decida se o operador linear
	$T\colon\mathbb{F}^{n}\to\mathbb{F}^{n}$ dado por sua matriz
	${\left[T\right]}_{\beta}$ é diagonalizável.
	Em caso positivo, determine uma base de autovetores e sua forma
	diagonal.

	\begin{parts}
		\begin{multicols}{3}
			\part\label{exercício:3.2a}

			\[
				\begin{bNiceMatrix}
					2  & 3 \\
					-1 & 1
				\end{bNiceMatrix},\quad
				\mathbb{F}=\mathbb{C};
			\]

			\part\label{exercício:3.2b}

			\[
				\begin{bNiceMatrix}
					1 & 1 \\
					1 & 1
				\end{bNiceMatrix},\quad
				\mathbb{F}=\mathbb{C};
			\]

			\part\label{exercício:3.2c}

			\[
				\begin{bNiceMatrix}
					-9  & 4 & 4 \\
					-8  & 3 & 4 \\
					-16 & 8 & 7
				\end{bNiceMatrix},\quad
				\mathbb{F}=\mathbb{R};
			\]

			\part\label{exercício:3.2d}

			\[
				\begin{bNiceMatrix}
					6  & -3 & -2 \\
					4  & -1 & -2 \\
					10 & 5  & -3
				\end{bNiceMatrix},\quad
				\mathbb{F}=\mathbb{C};
			\]

			\part\label{exercício:3.2e}

			\NiceMatrixOptions{
				cell-space-top-limit = 3pt
			}
			\[
				\begin{bNiceMatrix}
					\overline{2}  & \overline{6}  & \overline{3}  \\
					\overline{10} & \overline{6}  & \overline{10} \\
					\overline{6}  & \overline{12} & \overline{5}
				\end{bNiceMatrix},\quad
				\mathbb{F}=\mathbb{Z}/13\mathbb{Z};
			\]
			\NiceMatrixOptions{
				cell-space-top-limit = 0pt
			}
			\part\label{exercício:3.2f}

			\[
				\begin{bNiceMatrix}
					-2 & -1 & 2 \\
					-3 & 0  & 2 \\
					-8 & -4 & 7
				\end{bNiceMatrix},\quad
				\mathbb{F}=\mathbb{R}.
			\]
		\end{multicols}
	\end{parts}

	\begin{solutionordottedlines}
		.
		\begin{parts}
			\part

			\part

			\part

			\part

			\part

			\part
		\end{parts}
	\end{solutionordottedlines}

	\question\label{exercício:3.3}

	Seja $T\colon V\to V$ um operador linear com $V$ um espaço vetorial
	de dimensão finita sobre um corpo $\mathbb{F}$.
	Mostre que:

	\begin{parts}
		\part\label{exercício:3.3a}

		Se $p_{T}\left(x\right)$ possui todas as raízes com
		multiplicidade algébrica igual a $1$, então $T$ é diagonalizável;

		\part\label{exercício:3.3b}

		Se $\dim\left(\operatorname{Im}\left(T\right)\right)=m$, então
		$T$ tem no máximo $m+1$ autovalores;

		\part\label{exercício:3.3c}

		Se $\dim\left(V\right)=2$ e $\mathbb{F}=\mathbb{C}$, então a
		matriz de $T$ é semelhante a uma das matrizes:
		\[
			\begin{bNiceMatrix}
				a & 0 \\
				0 & b
			\end{bNiceMatrix},\quad
			a,b\in\mathbb{C};\quad
			\begin{bNiceMatrix}
				a & 0 \\
				1 & a
			\end{bNiceMatrix},\quad
			a\in\mathbb{C}.
		\]
	\end{parts}

	\begin{solutionordottedlines}
		.
		\begin{parts}
			\part

			\part

			\part
		\end{parts}
	\end{solutionordottedlines}

	\question\label{exercício:3.4}

	\begin{parts}\leavevmode
		\part\label{exercício:3.4a}

		Mostre que se $B,M\in\mathcal{M}_{m\times m}\left(F\right)$, com
		$M$ invertível, então ${\left(M^{-1}BM\right)}^{n}=M^{-1}B^{n}M$,
		para todo $n\in\mathbb{N}$;

		\part\label{exercício:3.4b}

		Calcule $A^{n}$, com $n\in\mathbb{N}$, onde
		\[
			\begin{bNiceMatrix}
				2 & 4  \\
				3 & 13
			\end{bNiceMatrix};
		\]

		\part\label{exercício:3.4c}

		Seja
		\[
			A=	\begin{bNiceMatrix}
				0  & 7 & -6 \\
				-1 & 4 & 0  \\
				0  & 2 & 2
			\end{bNiceMatrix},\quad
			\mathcal{M}_{3\times2}\left(\mathbb{C}\right).
		\]
		Dado $n\in\mathbb{N}$, determine
		$B\in\mathcal{M}_{3\times3}\left(\mathbb{C}\right)$ tal que
		$B^{n}=A$.
	\end{parts}

	\begin{solutionordottedlines}
		\begin{parts}
			\part

			\part

			\part
		\end{parts}
	\end{solutionordottedlines}

	\question\label{exercício:3.5}

	Seja $T\colon\mathbb{Q}^{3}\to\mathbb{Q}^{3}$ um operador linear
	que tem como autovetores $\left(3,1\right)$ e $\left(-2,1\right)$
	associados aos autovaloes $-2$ e $3$, respectivamente.
	Calcule $T\left(x,y\right)$.

	\begin{solutionordottedlines}
		.
	\end{solutionordottedlines}

	\question\label{exercício:3.6}
	Seja
	\begin{math}
		T\colon
		\mathcal{M}_{2\times2}
		\left(\mathbb{R}\right)\to
		\mathcal{M}_{2\times2}
		\left(\mathbb{R}\right)
	\end{math}
	um operador linear cuja matriz em relação à base
	\[
		\beta=
		\left\{
		\begin{bNiceMatrix}
			1 & 0 \\
			1 & 0
		\end{bNiceMatrix},
		\begin{bNiceMatrix}
			1 & 0 \\
			0 & 0
		\end{bNiceMatrix},
		\begin{bNiceMatrix}
			0 & 1 \\
			0 & 1
		\end{bNiceMatrix},
		\begin{bNiceMatrix}
			0 & 0 \\
			0 & 1
		\end{bNiceMatrix}
		\right\}
	\]
	é dada por
	\[
		{\left[T\right]}_{\beta}=
		\begin{bNiceMatrix}
			-1 & -4 & -2 & -2 \\
			-4 & -1 & -2 & -2 \\
			2  & 2  & 1  & 4  \\
			2  & 2  & 4  & 1
		\end{bNiceMatrix}.
	\]
	Determine uma matriz
	$M\in\mathcal{M}_{4\times4}\left(\mathbb{R}\right)$
	tal que $M^{-1}{\left[T\right]}_{\beta}M$ é diagonal.

	\begin{solutionordottedlines}
		.
	\end{solutionordottedlines}

	\question\label{exercício:3.7}

	Considere o $\mathbb{R}$-espaço vetorial $V$ de todas funções
	$f\colon\mathbb{R}\to\mathbb{R}$ infinitamente diferenciáveis.
	Seja $W$ o subespaço de $V$ gerado pelas funções
	$f_{1}\colon x\mapsto e^{2x}$,
	$f_{2}\colon x\mapsto e^{2x}\operatorname{sen}\left(x\right)$,
	$f_{3}\colon x\mapsto e^{2x}\cos\left(x\right)$.
	Mostre que $W$ é invariante pelo operador linear $D\colon V \to V$
	definido por $D\left(f\right)=f^{\prime}$, para todo $f\in V$.
	Mostre que $\beta=\left\{f_{1},f_{2},f_{3}\right\}$ é uma base de
	$W$.
	Determine a matriz de ${\left.D\right|}_{W}$ em relação a base
	$\beta$.
	Determine os autovaloes e autovetores de $D$.
	$D$ é diagonalizável?

	\begin{solutionordottedlines}
		.
	\end{solutionordottedlines}

	\question\label{exercício:3.8}
	Seja $T\colon V\to V$ um operador linear, com $V$ um espaço
	vetorial de dimensão finita sobre um corpo $F$.
	Mostre que:
	\begin{parts}
		\part\label{exercício:3.8a}

		Se $p_{T}\left(x\right)=x^{n}$, mostre que existe $m\geq1$ tal que
		$T^{m}=0$;

		\part\label{exercício:3.8b}

		Se $m_{T}(x)=(x-\lambda),$ mostre que $T$ é diagonalizável.

	\end{parts}

	\begin{solutionordottedlines}
		\begin{parts}
			\part

			\part
		\end{parts}
	\end{solutionordottedlines}

	\question\label{exercício:3.9}
	Encontre todas as possibilidades para o polinômio minimal de um
	operador linear $T\colon\mathbb{R}^{5}\to\mathbb{R}^{5}$ com
	polinômio característico:
	\begin{parts}
		\part\label{exercício:3.9a}

		\begin{math}
			p_{T}
			\left(x\right)=
			-{\left(x-3\right)}^{3}
			{\left(x-2\right)}^{2};
		\end{math}

		\part\label{exercício:3.9b}

		\begin{math}
			p_{T}
			\left(x\right)=
			-\left(x-1\right)
			\left(x-2\right)
			\left(x-3\right)
			\left(x-4\right)
			\left(x-5\right);
		\end{math}

		\part\label{exercício:3.9c}


		\begin{math}
			p_{T}
			\left(x\right)=
			-{\left(x-1\right)}^{m}
		\end{math},
		$m\geq1$.
	\end{parts}

	\begin{solutionordottedlines}
		\begin{parts}
			\part

			\part

			\part
		\end{parts}
	\end{solutionordottedlines}
\end{questions}
