\makeatletter
\patchcmd{\PEX@}{\dp\Pbox@>\dp\z@}{\ht\Pbox@>\dp\z@}{}{}
\patchcmd{\SQEX@}{\dp\Sbox@>\dp0}{\ht\Sbox@>\dp0}{}{}
\makeatother

\makeatletter
% Fix weird space
\patchcmd{\LEFTRIGHT}
{\kern-2\nulldelimiterspace\mskip-\thinmuskip}
{\kern-2\nulldelimiterspace}
{}{}
% Two new boxes
\newsavebox{\mtp@matrixbox}
\newsavebox{\mtp@casesbox}
% Round parentheses should always be used by default
% `pmatrix' from `amsmath'
\renewenvironment{pmatrix}{%
  \matrix@check\pmatrix\setbox\mtp@matrixbox=\hbox\bgroup$\env@matrix
}{%
  \endmatrix$\egroup\PARENS{\copy\mtp@matrixbox}%
}
% Curly braces are used only if `curlybraces' is set
% From `mtpro2.sty': \DeclareOption{curlybraces}{\let\mtp@br=c}
\ifx\mtp@br c
  % `Bmatrix' from `amsmath'
  \renewenvironment{Bmatrix}{%
    \setbox\mtp@matrixbox=\hbox\bgroup$\env@matrix
  }{%
    \endmatrix$\egroup\LEFTRIGHT\lbrace\rbrace{\copy\mtp@matrixbox}%
  }
  % `cases' from `amsmath'
  \renewcommand*\env@cases{%
  \let\@ifnextchar\new@ifnextchar
  \setbox\mtp@casesbox=\hbox\bgroup$%
    \def\arraystretch{1.2}%
    \array{@{}l@{\quad}l@{}}%
    }
    \renewenvironment{cases}{%
    \matrix@check\cases\env@cases
    }{%
    \endarray$\egroup\LEFTRIGHT\lbrace.{\copy\mtp@casesbox}%
  }
\fi
% Now, the matrices from `mathtools'
\MHInternalSyntaxOn
\MaybeMHPrecedingSpacesOff
% `pmatrix*' from `mathtools'
\renewenvironment{pmatrix*}[1][c]
{\setbox\mtp@matrixbox=\hbox\bgroup$\MT_matrix_begin:N #1}
{\MT_matrix_end:$\egroup\PARENS{\copy\mtp@matrixbox}}
\MH_if_meaning:NN \mtp@br c
% `Bmatrix*' from `mathtools'
\renewenvironment{Bmatrix*}[1][c]
{\setbox\mtp@matrixbox=\hbox\bgroup$\MT_matrix_begin:N #1}
{\MT_matrix_end:$\egroup\LEFTRIGHT\lbrace\rbrace{\copy\mtp@matrixbox}}
\MH_fi:
\MHPrecedingSpacesOn
\MHInternalSyntaxOff
\makeatother
% Patches end

\NewDocumentCommand{\csysteme}{som}{%
  \LEFTRIGHT\{.{%
      \sysdelim..
      \IfBooleanTF{#1}
      {\IfNoValueTF{#2}{\systeme*{#3}}{\systeme*[#2]{#3}}}
      {\IfNoValueTF{#2}{\systeme{#3}}{\systeme[#2]{#3}}}%
    }%
}

%\setlength\columnsep{10pt} % This is the default columnsep for all pages
\setmainfont{TeX Gyre Pagella}
\setmonofont{Fira Code}[Contextuals=Alternate]

\setkomafont{title}{\bfseries\sffamily\Huge\color{DarkBlue}}
\setkomafont{subtitle}{\bfseries\sffamily\Large\color{DarkBlue}}
\setkomafont{author}{\bfseries\sffamily\large\color{DarkBlue}}
\setkomafont{date}{\bfseries\sffamily\color{DarkBlue}}
\setkomafont{chapter}{\LARGE\color{DarkBlue}}
\setkomafont{section}{\Large\color{DarkBlue}}
\setkomafont{subsection}{\large\color{DarkBlue}}

\urlstyle{rm}

\newunicodechar{∈}{\makebox[\fontcharwd\font`a]{\(\in\)}}
\newunicodechar{⊗}{\makebox[\fontcharwd\font`a]{\(\otimes\)}}

\usetikzlibrary{matrix,arrows.meta,positioning}

\definecolor{myyellow}{RGB}{240,217,1}
\definecolor{mygreen}{RGB}{143,188,103}
\definecolor{myred}{RGB}{234,38,40}
\definecolor{myblue}{RGB}{53,101,167}

\newcommand{\MVAt}{{\usefont{U}{mvs}{m}{n}\symbol{`@}}}
\renewcommand{\qedsymbol}{\(\blacksquare\)}

\newcounter{tmp}

\newcommand\tikzmark[1]{%
  \tikz[remember picture,baseline=-0.65ex]
  \node[inner sep=0,outer sep=0] (#1){};%
}

\newcommand\mess[4][25pt]{%
  \stepcounter{tmp}%
  \begin{tikzpicture}[remember picture,overlay,>=latex,xshift=#1,cyan]
    \node[cyan,inner sep=2pt] at ([xshift=#1]#2) (a\thetmp) {$#4$};%circle,draw,
    \draw[->] (a\thetmp.south) |- (#3);
  \end{tikzpicture}%
}

\renewcommand{\solutiontitle}{\noindent\textbf{Solução}\par\noindent}
\definecolor{SolutionColor}{rgb}{0.2,0.9,1}
\colorsolutionboxes
\definecolor{SolutionBoxColor}{rgb}{0.2,0.9,1}
\vqword{Questão}
\vpword{Pontos}
\vsword{Pontuaçõe}

\theoremstyle{definition}
\newtheorem{definition}{Definição}[chapter]
\newtheorem{theorem}[definition]{Teorema}
\newtheorem{proposition}[definition]{Proposição}
\newtheorem{example}[definition]{Exemplo}
\newtheorem{exercise}[definition]{Exercício}
\newtheorem{remark}[definition]{Observação}

\graphicspath{ {img/} }

% \newmintedfile[juliacode]{julia}{
%   breaklines,
%   breakanywhere,
%   tabsize=4,
%   samepage=false,
%   showspaces=false,
%   showtabs =false
% }

\backgroundsetup{
  scale=1,
  angle=0,
  opacity=.1,
  contents={
      \begin{tikzpicture}[remember picture,overlay]
        \node at ([yshift=11pt,xshift=5pt]current page.center) {\includegraphics[width=8.5cm]{unb}};
      \end{tikzpicture}}
}

\hypersetup{
  pdfinfo={
      Title={Álgebra linear~Verão 2021~Notas das aulas na escola},
      Author={Alex Dantas et al.},
      Keywords={corpos, sistemas lineares},
      Subject={Álgebra linear},
      Producer={TeXstudio 3.0.4 Using Qt Version 5.15.2},
      Creator={LuaHBTeX, Version 1.12.0 (TeX Live 2020/Arch Linux)},
    }
  %	hyperref,
  pdfencoding=auto,
  linktocpage=true,
  colorlinks=true,
  linkcolor=NavyBlue,
  urlcolor=magenta,
  linktoc=all,
  pdfpagelabels,
  bookmarks=true,
  unicode,
  pdfborder = {0 0 0}
  %	filecolor = red,
}

\addbibresource{referências.bib}
