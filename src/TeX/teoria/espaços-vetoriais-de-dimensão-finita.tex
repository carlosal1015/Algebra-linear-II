\chapter{Espaços vetoriais de dimensão finita\quad$\left(13/01/2021\right)$}

\begin{definition}
	Seja $V$ um espaço vetorial sobre um corpo $\mathbb{F}$.
	Um subconjunto $S$ de $V$ é chamado linearmente dependente (LD)
	se existem $n$ vetores distintos $v_{1},\dotsc,v_{n}\in S$ e
	escalares $c_{1},\dotsc,c_{n}\in\mathbb{F}$, não todos nulos
	$\left(\left(c_{1},\dotsc,c_{n}\right)\in\mathbb{F}^{n}\setminus\left\{\left(0,\dotsc,0\right)\right\}\right)$,
	tais que
	\[
		c_{1}1v_{1}+\cdots+c_{n}v_{n}=
		0.
	\]
	Caso contrário, $S$ é chamado de linearmente independiente (LI),
	ou seja, se $v_{1},\dotsc,v_{n}\in S$ são vetores distintos de $S$
	tais que
	\[
		c_{1}1v_{1}+\cdots+c_{n}v_{n}=
		0,
	\]
	então $c_{1}=\cdots=c_{n}=0$.
\end{definition}

\begin{definition}
	Dado um espaço vetorial $V$ sobre um corpo $\mathbb{F}$, dizemos
	que um subconjunto $S$ é uma base de $V$ se
	\begin{enumerate}
		\item

		      \begin{math}
			      \left\langle S\right\rangle=
			      V
		      \end{math};

		\item

		      $S$ é LI.
	\end{enumerate}
\end{definition}

\begin{example}
	\begin{enumerate}\leavevmode
		\item

		      Considere o espaço vetorial $\mathcal{C}^{3}$ sobre
		      $\mathcal{C}$.
		      Note que o conjunto
		      \[
			      \left\{
			      \left(i,1,0\right),
			      \left(0,1,0\right),
			      \left(0,i,i\right),
			      \left(0,3,2\right)
			      \right\}
		      \]
		      é LD, pois
		      \[
			      \left(0,0,0\right)=
			      0\cdot\left(i,1,0\right)+
			      1\cdot\left(0,1,0\right)+
			      \left(-2i\right)\cdot\left(0,i,i\right)+
			      \left(-1\right)\cdot\left(0,3,2\right)
		      \]
		      mas
		      \[
			      \left\langle
			      \left(i,1,0\right),
			      \left(0,1,0\right),
			      \left(0,i,i\right),
			      \left(0,3,2\right)
			      \right\rangle=
			      \mathbb{C}^{3};
		      \]
		\item

		      Já
		      \begin{math}
			      \left\{
			      \left(i,1,0\right),
			      \left(0,1,0\right),
			      \left(0,i,i\right)
			      \right\}
		      \end{math}
		      é uma base de $\mathbb{C}^{3}$, pois
		      \[
			      x\left(i,1,0\right)+
			      y\left(0,1,0\right)+
			      z\left(0,i,i\right)=
			      \left(0,0,0\right)
			      \implies
			      \csysteme[xyz]{
				      ix=0,
				      x+y+iz=0,
				      iz=0
			      }
			      \implies x=y=z=0.
		      \]
		      logo é LI e gera $\mathbb{C}^{3}$
		\item
		      \begin{math}
			      e=
			      \left\{
			      \underbrace{\left(1,0,\dotsc,0\right)}_{n},
			      \underbrace{\left(0,1,\dotsc,0\right)}_{n},
			      \underbrace{\left(0,0,\dotsc,1\right)}_{n}
			      \right\}
		      \end{math}
		      é uma base para $\mathbb{F}^{n}$, chamada canônica.
	\end{enumerate}
\end{example}

\begin{theorem}
	Seja $V$ um espaço vetorial e $v_{1},\dotsc,v_{m}$ vetores de $V$
	com mais do que $m$ elementos é LD.
\end{theorem}

\begin{proof}
	Sejam $u_{1},u_{2},\dotsc,u_{n}\in V$ com $n>m$.
	Escreva
	\[
		\begin{NiceMatrix}
			u_{1} & =a_{11}v_{1}+\cdots+a_{m1}v_{m}  \\
			u_{2} & =a_{12}v_{2}+\cdots+a_{m2}v_{m}  \\
			      & =\vdots                          \\
			u_{n} & =a_{1n}v_{1}+\cdots+a_{mn} v_{m}
		\end{NiceMatrix}
	\]
	Afirmação: existem escalares $x_{1}\dotsc,x_{n}\in\mathbb{F}$,
	não todos nulos, tais que
	\[
		x_{1}u_{1}+
		\cdots+
		x_{n}u_{n}=0.
	\]
	De fato,
	\begin{gather}
		first equation\\
		\begin{split}
			second & equation\\
			& on two lines
		\end{split}
		\\
		third equation
	\end{gather}
	% \[
	% 	x_{1}u_{1}+\cdots+x_{n}u_{n}
	% 	&=x_{1}\left(a_{11}v_{1}+\cdots+a_{m1}v_{m}\right)+
	% 	\cdots \\
	% 	&\phantom{=}\cdots+x_{n}\left(a_{1n}v_{1}+\cdots+a_{mn}v_{m}\right)
	% 	=\left(
	% 	\underbrace{
	% 		a_{11}x_{1}+\cdots+a_{1n}x_{n}
	% 	}_{\in\mathbb{F}}\right)v_{1}+
	% 	\cdots+
	% 	\left(
	% 	\underbrace{
	% 		a_{m1}x_{1}+\cdots+a_{mn}x_{n}
	% 	}_{\in\mathbb{F}}
	% 	\right)v_{m}.
	% \]
	Para que $x_{1}u_{1}+\cdots+x_{n}u_{n}=0$ basta que o sistema linear homogêneo
	% \[
	% 	\csysteme[xyz]{
	% 	a_{n}x_{1}+\cdots+a_{1n}x_{n}=0,
	% 	%\vdots=\vdots,
	% 	a_{m1}x_{1}+\cdots+a_{mn}x_{n}=0
	% 	}
	% \]
\end{proof}
