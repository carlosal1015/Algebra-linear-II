\chapter{Espaços vetoriais\quad$\left(12/01/2021\right)$}

\begin{definition}
  Um conjunto não vazio $V$ é chamado de espaço vetorial sobre um corpo $\mathbb{F}$ se em $V$ estão definidas duas operações

  \[
    \begin{aligned}
      +\colon V\times V & \longrightarrow V \\
      \left(u,v\right)  & \longmapsto u+v
    \end{aligned}\qquad
    \begin{aligned}
      \cdot\colon\mathbb{F}\times V & \longrightarrow V    \\
      \left(c,v\right)              & \longmapsto c\cdot v
    \end{aligned}
  \]

\end{definition}

% (A1) $u+\left(v+w\right)=\left(u+v\right)+w,\forall u,v,\omega\in V$.
% (A2) Exr un $O\in V$ taa ave avk $v+\left(-v\right)=\left(-v\right)+v=0$;
% (A4) $u_{+}v=v+u,\forall u,v\in V_{i}$
% (E1) $I.v=v,\forall v\in V$, owor $1\in\mathbb{F}$;
% (E2) $\left(ck\right)v=c\left(kv\right), \forall v\in V,E\forall c, k \in \mathbb{F}$
% $(\pm 3) c\left(u+v\right)=cu+cv,\forall u,v\in V,\forall c\in\mathbb{F}$
% (E4) $\left(c+k\right)v=cv+kv,\quad\forall v\in V_{E}c,k\in\mathbb{F}$.

% \mathbb{F}^{n}=\left\{\left(x_{2},\dotsc,x_{n}\right)\mid x_{i}\in \mathbb{F},i=1,\dotsc,n\right\},

% \mathbb{F}^{m\times n}=
% \mathscr{N}_{N\times n}\left(\mathbb{F}\right)=
% \left\{A=\left(a_{ij}\right)_{w+1}\mid a_{ij}\in\mathbb{F}\right\}


% +\colon A+B=
% \left(a_{ij}\right)_{mx+1}+
% \left(b_{i j}\right)_{wxn}=
% \left(a_{ij}+b_{ij}\right)_{w\times n}\\
% \therefore cA=c\cdot\left(a_{ij}\right)_{mx_{n}}=
% \left(c\cdot a_{ij}\right)_{m\times n}


% i) $0\in W$;
% iii) $c\omega\in\omega,\forall c\in\mathbb{F}_{r}w\in\omega$

% i) W $\pi$ wat vazro
% ii) $w_{1}+cw_{2}\in W,\forall c\in\mathbb{F}_{k}\forall\omega_{1},w_{2}\in W$.

% A\left(X_{0}+c X_{1}\right)=A X_{0}+c A X_{1}

% c) W=
% \left\{A=\left(a_{ij}\right)_{3\times 2}\mid a_{n}=0\right\}

% \begin{aligned}
%    & \text { d) } \omega=\{f\colon\mathbb{R}\to\mathbb{R}\mid f(1)=0\} \text { i } \\
%    & \mathbb{F}=\{f: \mathbb{R} \rightarrow \mathbb{R}) \text { f is runcto }\}
% \end{aligned}

% +:\left(f+g\right)\left(x\right)=f\left(x\right)+g\left(x\right),\quad\therefore\left(cf\right)\left(x\right)=cf\left(x\right)


% i) \cap \omega_{i} \text { ; } \\
% ii)\sum_{i\in 1}\omega_{i}=\left\{w_{i}+a_{i}+\dotsb+w_{i_{n}}\mid \omega_{i}\in W_{j}\in n\in\mathbb{Z}_{>1}\right\}


% x+c v & =\left(u_{i_{1}}+\dotsb+u_{i_{e}}\right)+c\left(v_{j_{1}}+\dotsb+v_{j k}\right)=          \\
% & =u_{i_{2}}+\dotsb+u_{i_{e}}+c v_{j_{2}}+\dotsb+c_{v_{k}} \in \sum_{i \in \tau} \omega_{i}

% \bigcap_{i\in 2}\omega_{i}\subseteq U_{i\in I}\omega_{i}\subseteq \sum_{i\in Z}\omega_{i}

% \left(x,y,z\right)=x\left(1,0,0\right)+y\left(0,1,0\right)+z\left(0,0,1\right)

% \left(x,y,z\right)=-x i\left(i,1,0\right)+\left(y+xi-z\right)\left(0,1,0\right)-zi\left(0,i,i\right)

% W=\left\{c_{2}v_{2}+\dotsb+c_{n}v_{n}\mid v_{i}\in S_{\varepsilon}c_{i}\in\mathbb{F}\right\}

% \begin{aligned}
%   u+cv & =\left(c_{2}u_{2}+\dotsb+c_{n}u_{n}\right)+c\left(k_{1}v_{2}+\dotsb+k_{m}v_{m}\right)=       \\
%        & =c_{1}u_{2}+\dotsb+c_{n}u_{n}+\left(ck_{1}\right)v_{2}+\dotsb+\left(ck_{m}\right)v_{m}\in W.
% \end{aligned}

% \left\langle S\right\rangle=\bigcap_{S\leq w}w

% \left\langle S\right\rangle=
% \left\{c_{2}v_{2}+\dotsb+c_{n}v_{n}\mid v_{i}\in S_{\varepsilon}c_{2} \in\mathbb{F}\right\}=W

% \omega=c_{2}v_{2}+\dotsb+c_{n}v_{n}\in\cap u

% \mathbb{C}^{3}=
% \left\langle\left(i,1,0\right),\left(0,1,0\right),\left(0,i,i\right)\right\rangle

% x+y+z=0                                        \\
% 2x+y+2z=0                                      \\
% \mathbb{L}<z\to\alpha_{2}-2\alpha_{L} \\
% x+y+z=0\to x=-z                         \\
% -y=0\to y=0

% DS Szs7rum DADO. No7k aun
% \[
%   W=
%   \left\{
%   z\left(-1,0,1\right)\mid
%   z\in\mathbb{R}
%   \right\}= \\
%   =\left\langle\left(-1,0,1\right)\right\rangle
% \]

% \[
%   W=
%   \left\{
%   \left(z+y,y,z\right)\mid
%   z,y\in\mathbb{R}
%   \right\}=       \\
%   =\left\{
%   y\left(1,1,0\right)+
%   z\left(1,0,1\right)\mid
%   y,z\in\mathbb{R}\right\}= \\
%   =
%   \left\langle
%   \left(1,1,0\right),
%   \left(1,0,1\right)
%   \right\rangle
% \]

% \[
%   x+y+z=1 \\
%   2 x+y+2 z=2
%   \quad \implies
%   x+y+z =1 \to x=1-z \\
%   -y    =0 \to y=0
% \]


% $S=
%   \left\{
%   \left(1-z,0,z\right)\mid
%   z\in\mathbb{R}
%   \right\}=
%   \left\{
%   \left(1,0,0\right)+
%   z\left(-1,0,1\right)\mid
%   z\in\mathbb{R}
%   \right\}
% $