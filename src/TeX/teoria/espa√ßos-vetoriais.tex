\chapter{Espaços vetoriais\quad$\left(12/01/2021\right)$}

\begin{definition}
	Um conjunto não vazio $V$ é chamado de espaço vetorial sobre um corpo $\mathbb{F}$ se em $V$ estão definidas duas operações

	\[
		\begin{aligned}
			+\colon V\times V & \longrightarrow V \\
			\left(u,v\right)  & \longmapsto u+v
		\end{aligned}\qquad
		\begin{aligned}
			\cdot\colon\mathbb{F}\times V & \longrightarrow V    \\
			\left(c,v\right)              & \longmapsto c\cdot v
		\end{aligned}
	\]

	\begin{enumerate}[label={(A\arabic*)},leftmargin=0em,itemindent=*]
		\item\label{adição_vetores:1}

		      \begin{math}
			      u+
			      \left(v+w\right)=
			      \left(u+v\right)+
			      w,
			      \forall u,v,\omega\in V.
		      \end{math}

		\item\label{adição_vetores:2}

		      Existe um único vetor nulo $O\in V$ tal que

		      \begin{math}
			      v+0=
			      0+v=
			      v,
			      \forall v\in V.
		      \end{math}

		\item\label{adição_vetores:3}

		      Dado $v\in V$, existe um único vetor $-v\in V$ tal que
		      \begin{math}
			      v+\left(-v\right)=
			      \left(-v\right)+v=
			      0;
		      \end{math}

		\item\label{adição_vetores:4}

		      \begin{math}
			      u+v=
			      v+u,
			      \forall u,v\in V.
		      \end{math}
	\end{enumerate}

	\begin{enumerate}[label={(E\arabic*)},leftmargin=0em,itemindent=*]
		\item\label{multiplicação_vetores:1}

		      \begin{math}
			      1\cdot v=
			      v,
			      \forall v\in V
		      \end{math},
		      onde $1\in\mathbb{F}$.

		\item\label{multiplicação_vetores:2}

		      \begin{math}
			      \left(ck\right)v=
			      c\left(kv\right),
			      \forall v\in V,E
			      \text{ e }
			      \forall c,k\in\mathbb{F}.
		      \end{math}

		\item\label{multiplicação_vetores:3}

		      \begin{math}
			      c\left(u+v\right)=
			      cu+cv,
			      \forall u,v\in V,
			      \forall c\in\mathbb{F}
		      \end{math}
		\item\label{multiplicação_vetores:4}

		      \begin{math}
			      \left(c+k\right)v=
			      cv+kv,
			      \forall v\in V
			      \text{ e }
			      c,k\in\mathbb{F}
		      \end{math}.
	\end{enumerate}
\end{definition}


\begin{example}\leavevmode
	\begin{enumerate}
		\item
		      Se $\mathbb{F}$ é um corpo, então
		      \[
			      \mathbb{F}^{n}=
			      \left\{
			      \left(x_{1},\dotsc,x_{n}\right)
			      \mid x_{i}\in\mathbb{F},
			      i=1,\dotsc,n
			      \right\},
		      \]
		      com $n$ um enteiro positivo, é um espaço vetorial sobre
		      $\mathbb{F}$, onde as operações são dada por

		      \[
			      \begin{aligned}
				      +\colon
				      \mathbb{F}^{n}\times
				      \mathbb{F}^{n} & \longrightarrow
				      \mathbb{F}^{n}                   \\
				      \left(
				      \left(x_{1},\dotsc,x_{n}\right),
				      \left(y_{1},\dotsc,y_{n}\right)
				      \right)        & \longmapsto
				      \left(x_{1}+y_{1},\dotsc,x_{n}+y_{n}\right)
			      \end{aligned},\qquad
			      \begin{aligned}
				      \cdot\colon
				      \mathbb{F}\times
				      \mathbb{F}^{n}                                 & \longrightarrow
				      \mathbb{F}^{n}                                                   \\
				      \left(c,\left(x_{1},\dotsc,x_{n}\right)\right) & \longmapsto
				      \left(cx_{1},\dotsc,cx_{n}\right).
			      \end{aligned}
		      \]

		\item

		      Se $\mathbb{K}$ é um subcorpo de $\mathbb{F}$, então $\mathbb{F}$ é um espaço vetorial sobre $\mathbb{K}$.

		\item
		      O conjunto
		      \[
			      \mathbb{F}^{m\times n}=
			      \mathcal{M}_{m\times n}\left(\mathbb{F}\right)=
			      \left\{
			      A=
			      {\left(a_{ij}\right)}_{w+1}
			      \mid a_{ij}\in
			      \mathbb{F}
			      \right\},
		      \]
		      onde $\mathbb{F}$ é um corpo, munido das operações
		      \[
			      \begin{aligned}
				      +\colon
				      \mathbb{F}^{m\times n}\times
				      \mathbb{F}^{m\times n} &
				      \longrightarrow
				      \mathbb{F}^{m\times n}               \\
				      \left(A,B\right)       & \longmapsto
				      {\left(a_{ij}\right)}_{m\times n}+
				      {\left(b_{ij}\right)}_{m\times n}=
				      {\left(a_{ij}+b_{ij}\right)}_{m\times n}
			      \end{aligned},\qquad
			      \begin{aligned}
				      \cdot\colon
				      \mathbb{F}\times
				      \mathbb{F}^{m\times n} & \longrightarrow
				      \mathbb{F}^{n}                           \\
				      \left(c,A\right)       & \longmapsto
				      c\cdot
				      {\left(a_{ij}\right)}_{m\times n}=
				      {\left(c\cdot a_{ij}\right)}_{m\times n}.
			      \end{aligned}
		      \]
		      é um espaço vetorial sobre $\mathbb{F}$.
	\end{enumerate}
\end{example}

\begin{proposition}
	Em um espaço vetorial valem
	\begin{enumerate}
		\item $0\cdot v=0$;
		\item $c\cdot 0=0$;
		\item $-v=\left(-1\right)v$, para todo vetor $v$.
	\end{enumerate}
\end{proposition}

\begin{definition}[Subespaço vetorial]
	Seja $V$ um espaço vetorial sobre	um corpo $\mathbb{F}$.
	Um subconjunto não vazio $W$ de $V$ é chamado de subespaço vetorial
	de $V$ se $W$ munido das operações de $V$ é um espaço vetorial.
\end{definition}

\begin{itemize}
	\item

	      Em outras palavras, $W$ é um subespaço vetorial de $V$ se

	      \begin{enumerate}
		      \item

		            $0\in W$;

		      \item

		            $w_{1}+w_{2}\in W$, $\forall w_{1},w_{2}\in W$;

		      \item

		            $c\cdot w\in W$, $\forall c\in\mathbb{F}$ e $w\in W$.
	      \end{enumerate}
	\item

	      Mais ainda, $W$ é um subespaço vetorial de $V$ se
	      \begin{enumerate}
		      \item

		            $W$ é  vazio;

		      \item

		            $w_{1}+cw_{2}\in W$, $\forall c\in\mathbb{F}$ e $\forall w_{1},w_{2}\in W$.
	      \end{enumerate}
\end{itemize}

\begin{example}
	\begin{enumerate}\leavevmode
		\item

		      $\left\{0\right\}$, $V$ são subespaços vetoriais de $V$;

		\item

		      Seja $A={\left(a_{ij}\right)}_{m\times n}$ uma matriz sobre um corpo $\mathbb{F}$. Então o conjunto solução do sistema homogêneo
		      \[
			      A
				      {X}_{n\times1}=
			      {0}_{m\times1}
		      \]
		      é um subespaço vetorial de $\mathbb{F}^{n}$.

		      \begin{proof}
			      Seja $W$ o conjunto solução de $AX=0$.
			      Note que $W$ é não vazio, pois $\left(0,\dotsc,0\right)\in W$.
			      Sejam $X_{0}$, $X_{1}$ duas soluções de $AX=0$.
			      Assim
			      \begin{align*}
				      A\left(X_{0}+cX_{1}\right)
				       & =AX_{0}+cAX_{1} &  & \\
				       & =0 + c\cdot 0=0 &  &
			      \end{align*}
			      para todo $c\in\mathbb{F}$.
			      Daí $X_{0}+cX_{1}\in W$ e o resultado segue.
		      \end{proof}

		\item

		      \begin{math}
			      W=
			      \left\{
			      A=
			      {\left(a_{ij}\right)}_{3\times2}
			      \mid a_{11}=0
			      \right\}
		      \end{math}
		      é um subespaço vetorial de $\mathbb{F}^{3\times2}=\mathcal{M}_{3\times2}\left(\mathbb{F}\right)$;

		\item

		      \begin{math}
			      W=
			      \left\{
			      f\colon
			      \mathbb{R}\to\mathbb{R}
			      \mid f\left(1\right)=0
			      \right\}
		      \end{math}
		      é um subespaço vetorial
		      \begin{math}
			      \mathcal{F}=
			      \left\{
			      f\colon
			      \mathbb{R}\to\mathbb{R}
			      \mid f\text{ é função.}
			      \right\}
		      \end{math}.
	\end{enumerate}
\end{example}


\fbox{$
		+\colon
		\left(f+g\right)\left(x\right)=
		f\left(x\right)+
		g\left(x\right),\quad
		\cdot\colon
		\left(cf\right)\left(x\right)=
		cf\left(x\right).
	$}

\begin{proposition}
	Sejam $V$ um espaço vetorial e
\end{proposition}

\begin{enumerate}
	\item

	      \begin{math}
		      \cap_{i\in I}w_{i}
	      \end{math};

	\item

	      \begin{math}
		      \sum_{i\in 1}w_{i}=
		      \left\{
		      w_{i}+a_{i}+\dotsb+w_{i_{n}}
		      \mid w_{i}\in W_{j}\in
		      n\in\mathbb{Z}_{>1}
		      \right\}
	      \end{math};

\end{enumerate}

% x+c v &=
% \left(u_{i_{1}}+\dotsb+u_{i_{e}}\right)+
% c\left(v_{j_{1}}+\dotsb+v_{j k}\right)=\\
% &=
% u_{i_{2}}+\dotsb+u_{i_{e}}+c v_{j_{2}}+\dotsb+c_{v_{k}} \in \sum_{i\in\tau}
% w_{i}

% \bigcap_{i\in 2}w_{i}\subseteq U_{i\in I}w_{i}\subseteq\sum_{i\in Z}w_{i}

% \left(x,y,z\right)=
% x\left(1,0,0\right)+
% y\left(0,1,0\right)+
% z\left(0,0,1\right)

% \left(x,y,z\right)=
% -xi\left(i,1,0\right)+
% \left(y+xi-z\right)
% \left(0,1,0\right)-
% zi\left(0,i,i\right)

% W=
% \left\{
% c_{2}v_{2}+\dotsb+c_{n}v_{n}
% \mid v_{i}\in S_{\varepsilon}c_{i}\in\mathbb{F}
% \right\}

% \begin{aligned}
% 	u+cv & =
% 	\left(
% 	c_{2}u_{2}+
% 	\dotsb+
% 	c_{n}u_{n}
% 	\right)+
% 	c\left(k_{1}v_{2}+\dotsb+k_{m}v_{m}\right)= \\
% 	     & =
% 	c_{1}u_{2}+
% 	\dotsb+
% 	c_{n}u_{n}+
% 	\left(ck_{1}\right)v_{2}+
% 	\dotsb+
% 	\left(ck_{m}\right)v_{m}
% 	\in W.
% \end{aligned}

% \left\langle S\right\rangle=
% \bigcap_{S\leq w}w

% \left\langle S\right\rangle=
% \left\{
% c_{2}v_{2}+
% \dotsb+
% c_{n}v_{n}
% \mid v_{i}\in
% S_{\varepsilon}c_{2}\in
% \mathbb{F}\right\}=
% W

% W=
% c_{2}v_{2}+
% \dotsb+
% c_{n}v_{n}\in
% \cap u

% \mathbb{C}^{3}=
% \left\langle
% \left(i,1,0\right),
% \left(0,1,0\right),
% \left(0,i,i\right)
% \right\rangle

% x+y+z=0                                        \\
% 2x+y+2z=0                                      \\
% \mathbb{L}<z\to\alpha_{2}-2\alpha_{L} \\
% x+y+z=0\to x=-z                         \\
% -y=0\to y=0

% DS Szs7rum DADO. No7k aun
% \[
% 	W=
% 	\left\{
% 	z\left(-1,0,1\right)\mid
% 	z\in\mathbb{R}
% 	\right\}= \\
% 	=\left\langle\left(-1,0,1\right)\right\rangle
% \]

% \[
% 	W=
% 	\left\{
% 	\left(z+y,y,z\right)\mid
% 	z,y\in\mathbb{R}
% 	\right\}=       \\
% 	=\left\{
% 	y\left(1,1,0\right)+
% 	z\left(1,0,1\right)\mid
% 	y,z\in\mathbb{R}\right\}= \\
% 	=
% 	\left\langle
% 	\left(1,1,0\right),
% 	\left(1,0,1\right)
% 	\right\rangle
% \]

% \[
% 	x+y+z=1 \\
% 	2 x+y+2 z=2
% 	\quad \implies
% 	x+y+z =1 \to x=1-z \\
% 	-y    =0 \to y=0
% \]


% $S=
% 	\left\{
% 	\left(1-z,0,z\right)\mid
% 	z\in\mathbb{R}
% 	\right\}=
% 	\left\{
% 	\left(1,0,0\right)+
% 	z\left(-1,0,1\right)\mid
% 	z\in\mathbb{R}
% 	\right\}
% $
