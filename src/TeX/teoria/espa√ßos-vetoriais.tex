\chapter{Espaços vetoriais\quad$\left(12/01/2021\right)$}

\begin{definition}
	Um conjunto não vazio $V$ é chamado de espaço vetorial sobre um
	corpo $\mathbb{F}$ se em $V$ estão definidas duas operações

	\[
		\begin{aligned}
			+\colon V\times V & \longrightarrow V \\
			\left(u,v\right)  & \longmapsto u+v
		\end{aligned}\qquad
		\begin{aligned}
			\cdot\colon\mathbb{F}\times V & \longrightarrow V    \\
			\left(c,v\right)              & \longmapsto c\cdot v
		\end{aligned}
	\]

	\begin{enumerate}[label={(A\arabic*)},leftmargin=0em,itemindent=*]
		\item\label{adição_vetores:1}

		      \begin{math}
			      u+
			      \left(v+w\right)=
			      \left(u+v\right)+
			      w,
			      \forall u,v,\omega\in V.
		      \end{math}

		\item\label{adição_vetores:2}

		      Existe um único vetor nulo $O\in V$ tal que

		      \begin{math}
			      v+0=
			      0+v=
			      v,
			      \forall v\in V.
		      \end{math}

		\item\label{adição_vetores:3}

		      Dado $v\in V$, existe um único vetor $-v\in V$ tal que
		      \begin{math}
			      v+\left(-v\right)=
			      \left(-v\right)+v=
			      0;
		      \end{math}

		\item\label{adição_vetores:4}

		      \begin{math}
			      u+v=
			      v+u,
			      \forall u,v\in V.
		      \end{math}
	\end{enumerate}

	\begin{enumerate}[label={(E\arabic*)},leftmargin=0em,itemindent=*]
		\item\label{multiplicação_vetores:1}

		      \begin{math}
			      1\cdot v=
			      v,
			      \forall v\in V
		      \end{math},
		      onde $1\in\mathbb{F}$.

		\item\label{multiplicação_vetores:2}

		      \begin{math}
			      \left(ck\right)v=
			      c\left(kv\right),
			      \forall v\in V,E
			      \text{ e }
			      \forall c,k\in\mathbb{F}.
		      \end{math}

		\item\label{multiplicação_vetores:3}

		      \begin{math}
			      c\left(u+v\right)=
			      cu+cv,
			      \forall u,v\in V,
			      \forall c\in\mathbb{F}
		      \end{math}
		\item\label{multiplicação_vetores:4}

		      \begin{math}
			      \left(c+k\right)v=
			      cv+kv,
			      \forall v\in V
			      \text{ e }
			      c,k\in\mathbb{F}
		      \end{math}.
	\end{enumerate}
\end{definition}


\begin{example}\leavevmode
	\begin{enumerate}
		\item
		      Se $\mathbb{F}$ é um corpo, então
		      \[
			      \mathbb{F}^{n}=
			      \left\{
			      \left(x_{1},\dotsc,x_{n}\right)
			      \mid x_{i}\in\mathbb{F},
			      i=1,\dotsc,n
			      \right\},
		      \]
		      com $n$ um enteiro positivo, é um espaço vetorial sobre
		      $\mathbb{F}$, onde as operações são dada por

		      \[
			      \begin{aligned}
				      +\colon
				      \mathbb{F}^{n}\times
				      \mathbb{F}^{n} & \longrightarrow
				      \mathbb{F}^{n}                   \\
				      \left(
				      \left(x_{1},\dotsc,x_{n}\right),
				      \left(y_{1},\dotsc,y_{n}\right)
				      \right)        & \longmapsto
				      \left(x_{1}+y_{1},\dotsc,x_{n}+y_{n}\right)
			      \end{aligned},\qquad
			      \begin{aligned}
				      \cdot\colon
				      \mathbb{F}\times
				      \mathbb{F}^{n}                                 & \longrightarrow
				      \mathbb{F}^{n}                                                   \\
				      \left(c,\left(x_{1},\dotsc,x_{n}\right)\right) & \longmapsto
				      \left(cx_{1},\dotsc,cx_{n}\right).
			      \end{aligned}
		      \]

		\item

		      Se $\mathbb{K}$ é um subcorpo de $\mathbb{F}$, então
		      $\mathbb{F}$ é um espaço vetorial sobre $\mathbb{K}$.

		\item
		      O conjunto
		      \[
			      \mathbb{F}^{m\times n}=
			      \mathcal{M}_{m\times n}\left(\mathbb{F}\right)=
			      \left\{
			      A=
			      {\left(a_{ij}\right)}_{w+1}
			      \mid a_{ij}\in
			      \mathbb{F}
			      \right\},
		      \]
		      onde $\mathbb{F}$ é um corpo, munido das operações
		      \[
			      \begin{aligned}
				      +\colon
				      \mathbb{F}^{m\times n}\times
				      \mathbb{F}^{m\times n} &
				      \longrightarrow
				      \mathbb{F}^{m\times n}               \\
				      \left(A,B\right)       & \longmapsto
				      {\left(a_{ij}\right)}_{m\times n}+
				      {\left(b_{ij}\right)}_{m\times n}=
				      {\left(a_{ij}+b_{ij}\right)}_{m\times n}
			      \end{aligned},\qquad
			      \begin{aligned}
				      \cdot\colon
				      \mathbb{F}\times
				      \mathbb{F}^{m\times n} & \longrightarrow
				      \mathbb{F}^{n}                           \\
				      \left(c,A\right)       & \longmapsto
				      c\cdot
				      {\left(a_{ij}\right)}_{m\times n}=
				      {\left(c\cdot a_{ij}\right)}_{m\times n}.
			      \end{aligned}
		      \]
		      é um espaço vetorial sobre $\mathbb{F}$.
	\end{enumerate}
\end{example}

\begin{proposition}
	Em um espaço vetorial valem
	\begin{enumerate}
		\item $0\cdot v=0$;
		\item $c\cdot 0=0$;
		\item $-v=\left(-1\right)v$, para todo vetor $v$.
	\end{enumerate}
\end{proposition}

\begin{definition}[Subespaço vetorial]
	Seja $V$ um espaço vetorial sobre	um corpo $\mathbb{F}$.
	Um subconjunto não vazio $W$ de $V$ é chamado de subespaço vetorial
	de $V$ se $W$ munido das operações de $V$ é um espaço vetorial.
\end{definition}

\begin{itemize}
	\item

	      Em outras palavras, $W$ é um subespaço vetorial de $V$ se

	      \begin{enumerate}
		      \item

		            $0\in W$;

		      \item

		            $w_{1}+w_{2}\in W$, $\forall w_{1},w_{2}\in W$;

		      \item

		            $c\cdot w\in W$, $\forall c\in\mathbb{F}$ e $w\in W$.
	      \end{enumerate}
	\item

	      Mais ainda, $W$ é um subespaço vetorial de $V$ se
	      \begin{enumerate}
		      \item

		            $W$ é  vazio;

		      \item

		            $w_{1}+cw_{2}\in W$, $\forall c\in\mathbb{F}$ e
		            $\forall w_{1},w_{2}\in W$.
	      \end{enumerate}
\end{itemize}

\begin{example}
	\begin{enumerate}\leavevmode
		\item

		      $\left\{0\right\}$, $V$ são subespaços vetoriais de $V$;

		\item

		      Seja $A={\left(a_{ij}\right)}_{m\times n}$ uma matriz sobre
		      um corpo $\mathbb{F}$.
		      Então o conjunto solução do sistema homogêneo
		      \[
			      A
				      {X}_{n\times1}=
			      {0}_{m\times1}
		      \]
		      é um subespaço vetorial de $\mathbb{F}^{n}$.

		      \begin{proof}
			      Seja $W$ o conjunto solução de $AX=0$.
			      Note que $W$ é não vazio, pois
			      $\left(0,\dotsc,0\right)\in W$.
			      Sejam $X_{0}$, $X_{1}$ duas soluções de $AX=0$.
			      Assim
			      \begin{align*}
				      A\left(X_{0}+cX_{1}\right)
				       & =AX_{0}+cAX_{1} &  & \\
				       & =0 + c\cdot 0=0 &  &
			      \end{align*}
			      para todo $c\in\mathbb{F}$.
			      Daí $X_{0}+cX_{1}\in W$ e o resultado segue.
		      \end{proof}

		\item

		      \begin{math}
			      W=
			      \left\{
			      A=
			      {\left(a_{ij}\right)}_{3\times2}
			      \mid a_{11}=0
			      \right\}
		      \end{math}
		      é um subespaço vetorial de
		      \begin{math}
			      \mathbb{F}^{3\times2}=
			      \mathcal{M}_{3\times2}\left(\mathbb{F}\right)
		      \end{math};

		\item

		      \begin{math}
			      W=
			      \left\{
			      f\colon
			      \mathbb{R}\to\mathbb{R}
			      \mid f\left(1\right)=0
			      \right\}
		      \end{math}
		      é um subespaço vetorial
		      \begin{math}
			      \mathcal{F}=
			      \left\{
			      f\colon
			      \mathbb{R}\to\mathbb{R}
			      \mid f\text{ é função.}
			      \right\}
		      \end{math}.
	\end{enumerate}
\end{example}


\fbox{$
		+\colon
		\left(f+g\right)\left(x\right)=
		f\left(x\right)+
		g\left(x\right),\quad
		\cdot\colon
		\left(cf\right)\left(x\right)=
		cf\left(x\right).
	$}

\begin{proposition}
	Sejam $V$ um espaço vetorial e ${\left\{W_{i}\right\}}_{i\in I}$
	uma família de subespaços vetorialis de $V$, onde $I$ é um conjunto
	de índices.
	Então são subespaços de $V$:
	\begin{enumerate}
		\item

		      \begin{math}
			      \cap_{i\in I}W_{i}
		      \end{math};

		\item

		      \begin{math}
			      \sum_{i\in I}W_{i}=
			      \left\{
			      w_{i_{1}}+w_{i_{2}}+\dotsb+w_{i_{n}}
			      \mid w_{i_{j}}\in W_{i_{j}}
			      \text{ e }
			      n\in\mathbb{Z}_{>1}
			      \right\}
		      \end{math}.
	\end{enumerate}
\end{proposition}

\begin{proof}
	\begin{enumerate}\leavevmode
		\item Note que $\cap_{i\in I}W_{i}$ é não vazia, pois $0\in W_{i}$,
		      $\forall i\in I$.
		      Se $u,v\in\cap_{i\in I}W_{i}$ e $c\in\mathbb{F}$, então
		      \[
			      u+cv\in W_{i}
			      \text{ para todo }
			      i\in I
		      \]
		      pois $W_{i}$ é subespaço, daí $u+cv\in\cap_{i\in I}W_{i}$ e o
		      resultado segue.
		\item Note que $\sum_{i\in I}W_{i}$ é não vazia pois $0\in\sum_{i\in I}W_{i}$.
		      Tome $u,v\in\sum_{i\in I}W_{i}$ e $c\in\mathbb{F}$.
		      Assim
		      \begin{align*}
			      u+cv & =
			      \left(u_{i_{1}}+\dotsb+u_{i_{\ell}}\right)+
			      c\left(v_{j_{1}}+\dotsb+v_{j_{k}}\right)= \\
			           & =
			      u_{i_{1}}+\dotsb+u_{i_{\ell}}+cv_{j_{1}}+\dotsb+c_{v_{k}}\in\sum_{i\in I}W_{i}.
		      \end{align*}
	\end{enumerate}
\end{proof}


\begin{remark}
	Nem sempre a união de subespaços é um subespaço, mas
	\begin{math}
		\bigcap_{i\in I}W_{i}\subseteq
		\cup_{i\in I}W_{i}\subseteq
		\sum_{i\in I}W_{i}
	\end{math}.
\end{remark}

\begin{definition}
	Sejam $V$ um espaço vetorial e $v_{1},\dotsc,v_{n}$ vetores de $V$.
	Dizemos que um vetor $v\in V$ é uma combinação linear dos vetores
	$v_{1},\dotsc,v_{n}$ se existem escalares
	$c_{1},\dotsc,c_{n}\in\mathbb{F}$ tais que
	\[ v=c_{1}v_{1}+\dotsb+c_{n}v_{n}. \]
\end{definition}

\begin{example}
	\begin{enumerate}\leavevmode
		\item Todo vetor de $\mathbb{R}^{3}$ é uma combinação linear dos
		      vetores $\left(1,0,0\right)$, $\left(0,1,0\right)$, $\left(0,0,1\right)$.
		      De fato,
		      \[
			      \left(x,y,z\right)=
			      x\left(1,0,0\right)+
			      y\left(0,1,0\right)+
			      z\left(0,0,1\right);
		      \]
		\item Todo vetor de $\mathbb{C}^{3}$ é uma combinação linear
		      dos vetores $\left(i,1,0\right)$, $\left(0,1,0\right)$, $\left(0,i,i\right)$.
		      De fato,
		      \[
			      \left(x,y,z\right)=
			      -xi\left(i,1,0\right)+
			      \left(y+xi-z\right)\left(0,1,0\right)-
			      zi\left(0,i,i\right).
		      \]
	\end{enumerate}
\end{example}

\begin{proposition}
Se $S$ é um subconjunto não vazio de um espaço vetorial $V$, então
o conjunto
\[
	W=
	\left\{
	c_{1}v_{1}+\dotsb+c_{n}v_{n}
	\mid v_{i}\in S
	\text{ e }
	c_{i}\in\mathbb{F}
	\right\}
\]
é um subespaço vetorial de $V$.
\end{proposition}

\begin{proof}
	Note que $W$ é não vazio, pois $S\subseteq W$.
	Sejam $u,v\in W$ e $c\in\mathbb{F}$.
	Então
	\begin{align*}
		u+cv
		 & =
		\left(
		c_{1}u_{1}+
		\dotsb+
		c_{n}u_{n}
		\right)+
		c\left(k_{1}v_{1}+\dotsb+k_{m}v_{m}\right)= \\
		 & =
		c_{1}u_{1}+
		\dotsb+
		c_{n}u_{n}+
		\left(ck_{1}\right)v_{1}+
		\dotsb+
		\left(ck_{m}\right)v_{m}
		\in W.
	\end{align*}
\end{proof}

\begin{definition}
	Dado $S$ um subconjunto de um espaço vetorial $V$, chamaremos ao subespaço
	\[
		\left\langle S\right\rangle=
		\bigcap_{S\subseteq W}W,\quad
		\fbox{onde $W$ é subespaço de $V$}
	\]
	de subespaço de $V$ gerado por $S$.
	(Assum $	\left\langle \emptyset\right\rangle=\left\{0\right\}$.)
\end{definition}

\begin{proposition}
	Se $S$ é um subconjunto não vazio de $V$, então
	\[
		\left\langle S\right\rangle=
		\left\{
		c_{1}v_{1}+
		\dotsb+
		c_{n}v_{n}
		\mid v_{i}\in S
		\text{ e }
		c_{i}\in
		\mathbb{F}
		\right\}=
		W
	\]
\end{proposition}

\begin{proof}
	Pela proposição 9, $W$ é um subespaço de $V$ que contém $S$, logo $\left\langle S\right\rangle\subseteq W$ por definição.
	Tome $w\in W$. Então existem vetores $v_{1},\dotsc,v_{n}$ em $S$ e escalares $c_{1},\dotsc,c_{n}\in\mathbb{F}$ tais que
	\[
		W=
		\underbrace{c_{1}v_{1}}_{\in\cap_{S\subseteq U} U}+
		\dotsb+
		\underbrace{c_{n}v_{n}}_{\in\cap_{S\subseteq U} U}\in
		\cap_{S\subseteq U}U,
	\]
	onde a interseção é tomada em todos os subespaços $U$ de $V$ que contém $S$.
	Daí $W\subseteq\left\langle S\right\rangle$.
\end{proof}

\begin{example}
	\begin{enumerate}\leavevmode
		\item Pelo exemplo 8-b, temos que
		      \[
			      \mathbb{C}^{3}=
			      \left\langle
			      \left(i,1,0\right),
			      \left(0,1,0\right),
			      \left(0,i,i\right)
			      \right\rangle;
		      \]
		\item Considere o sistema linear
		      \[
			      \csysteme[xyz]{
				      x+y+z=0,
				      2x+y+2z=0
			      }%L_{2}\to L_{2}-2L_{1}
			      \csysteme[xyz]{
				      x+y+z=0,
				      -y=0
			      }%\implies x=-z \implies y=0
		      \]
		      Daí
		      \begin{math}
			      W=
			      \left\{
			      z\left(-1,0,1\right)\mid
			      z\in\mathbb{R}
			      \right\}
		      \end{math}
		      é solução do sistema dado.
		      Note que
		      \begin{align*}
			      W
			       & =\left\{z\left(-1,0,1\right)\mid z\in\mathbb{R}\right\}
			       & =\left\langle\left(-1,0,1\right)\right\rangle.
		      \end{align*}


		      \begin{align*}
			      W
			       & =
			      \left\{
			      \left(z+y,y,z\right)\mid
			      z,y\in\mathbb{R}
			      \right\}                 \\
			       & =
			      \left\{
			      y\left(1,1,0\right)+
			      z\left(1,0,1\right)\mid
			      y,z\in\mathbb{R}\right\} \\
			       & =
			      \left\langle
			      \left(1,1,0\right),
			      \left(1,0,1\right)
			      \right\rangle
		      \end{align*}
	\end{enumerate}
\end{example}

\[
	\csysteme[xyz]{
		x+y+z=1,
		2x+y+2z=2
	}
	\implies
	\csysteme[xyz]{
		x+y+z=1,
		-y=0
	}% 	 \implies x=1-z \implies y=0
\]

$S=
	\left\{
	\left(1-z,0,z\right)\mid
	z\in\mathbb{R}
	\right\}=
	\left\{
	\left(1,0,0\right)+
	z\left(-1,0,1\right)\mid
	z\in\mathbb{R}
	\right\}
$
