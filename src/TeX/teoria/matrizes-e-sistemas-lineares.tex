\chapter{Matrizes e sistemas lineares\quad$\left(09/01/2021\right)$}\footnote{Aula de reposição.}

\begin{definition}[Matriz reducida por linhas]
  Uma matriz $A={\left(a_{ij}\right)}_{m\times n}$ sobre $\mathbb{F}$
  é deja reduzida por linhas se

  \begin{enumerate}
    \item

          O primeiro elemento não nulo de cada linha não nula é
          igual $1$;

    \item

          cada columna que possui o primeiro elemento não nulo de uma
          linha não possui todos os outros elementos iguais a $0$;
  \end{enumerate}


  Sea além disso, esa matriz $A$ satisfaz

  \begin{enumerate}[resume]

    \item

          todas linhas nulas ocorrem abaixo das linhas não nulas;

    \item

          Se $1,\dotsc,r$ $\left(r\leq m\right)$ são as linhas não
          nulas de $A$ com os primeiros elementos não nunos ocurrendo
          nas colunas $k_{1},k_{2},k_{r}$, respectivamente, então
          $k_{1}<k_{2}<\dotsb<k_{r}$, dizemos que $A$ está na forma
          escada reduzida.
  \end{enumerate}
\end{definition}

Dizemos que $A$ está na forma escada reduzida.

\begin{example}
  \begin{enumerate}
    \item

          As seguintes matrizes estão na forma reduzida:
          \begin{multicols}{2}

            \begin{enumerate}
              \item
                    \begin{math}
                      \begin{pNiceMatrix}
                        0 & 1 & 2 & 3 \\
                        1 & 0 & 1 & 1
                      \end{pNiceMatrix}
                    \end{math},
              \item

                    \begin{math}
                      \begin{pNiceMatrix}
                        0 & 0 & 0 \\
                        1 & 0 & 1 \\
                        0 & 1 & 0
                      \end{pNiceMatrix}
                    \end{math},

              \item

                    \begin{math}
                      \begin{pNiceMatrix}
                        0 & 0 & 0 \\
                        0 & 0 & 0 \\
                        0 & 0 & 0
                      \end{pNiceMatrix}
                    \end{math},

              \item

                    \begin{math}
                      \begin{pNiceMatrix}
                        1 & 0 & 0 & 0 \\
                        0 & 1 & 1 & 2 \\
                        0 & 0 & 0 & 0
                      \end{pNiceMatrix}
                    \end{math}.
            \end{enumerate}
          \end{multicols}

    \item

          As seguientes matrizes estão na forma escada reducida
          \begin{multicols}{2}

            \begin{enumerate}
              \item

                    \begin{math}
                      \begin{pNiceMatrix}
                        1 & 0 & 2 & 0 \\
                        0 & 1 & 3 & 0 \\
                        0 & 0 & 0 & 1
                      \end{pNiceMatrix}
                    \end{math},

              \item

                    \begin{math}
                      \begin{pNiceMatrix}
                        1 & 0 \\
                        0 & 1 \\
                        0 & 0
                      \end{pNiceMatrix}
                    \end{math},


              \item

                    \begin{math}
                      \begin{pNiceMatrix}
                        1 & 2 & 3 & 0 & 1 \\
                        0 & 0 & 0 & 1 & 0
                      \end{pNiceMatrix}
                    \end{math}.
            \end{enumerate}
          \end{multicols}

  \end{enumerate}
\end{example}

\begin{remark}
  \begin{enumerate}
    \item

          Se $A={\left(a_{ij}\right)}_{m\times n}$ está na forma escada reduzida e tem a última linha não nula, então $A=I_{m}$;

    \item Se $AX=0$ e

  \end{enumerate}
\end{remark}

\begin{definition}
  .
\end{definition}

% Obs. Sroam

% X_{1}=X_{2}\text{sougsis oz}AX=0\\
% A\left(X_{1}+c X_{2}\right)
% &=A X_{1}+c A X_{2}=\\
% &=0+c O=0

% 2 x+3 y-z+w=5, \\
% x-y+2 z-2 w=1 \\
% 2 x+y+z+w=3

% 2 & 3 & -1 & 1 & 5 \\
% 1 & -1 & 2 & -2 & 1 \\
% 2 & 1 & 1 & 1 & 3
% %\underbrace{\left(E_{k}\dotsm Z_{2}\right) A}_{A^{-1}}=I_{m}
% $E_{k}\dotsm E_{1}\left(AI_{m}\right)$