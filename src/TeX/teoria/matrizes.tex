\chapter{Matrizes\quad$\left(08/01/2021\right)$}

Podemos denotar uma matriz $A$ sobre um corpo $\mathbb{F}$ de ordem
$m\times n$ por $A={\left(a_{ij}\right)}_{m\times n}$.

Sejam $A={\left(a_{ij}\right)}_{m\times n}$ e
$B={\left(b_{jl}\right)}_{n\times p}$ duas matrizes sobre um corpo
$\mathbb{F}$.
Definimos o producto de $A$ por $B$ como a matriz
$C={\left(c_{il}\right)}_{m\times p}$ dada por
\[
  c_{il}=
  \sum_{j=1}^{n}
  a_{i j}b_{jl}=
  a_{i1}b_{il}+\dotsb+a_{in}b_{nl}
\]
%\section*{Ilustração}
\begin{figure}[H]
  \centering
  \begin{tikzpicture}[
	mymatrix/.style={
			matrix of math nodes,
			outer sep=0pt,
			nodes={
					draw,
					text width=2.5em,
					align=center,
					minimum height=2.5em,
					text=gray
				},
			nodes in empty cells,
			column sep=-\pgflinewidth,
			row sep=-\pgflinewidth,
			left delimiter=[,
					right delimiter=],
		},
	mycircle/.style 2 args={
			draw=#1,
			circle,
			fill=#2,
			line width=2pt,
			inner sep=5pt
		},
	arr/.style={
	line width=4pt,
	-{Triangle[angle=60:1.5pt 3]},
	#1,
	shorten >= 3pt,
	shorten <= 3pt
	}
	]
	%the matrices
	\matrix[mymatrix] (A)
	{
	|[text=black]|a_{11} & |[text=black]|a_{12} \\
	a_{21} & a_{22} \\
	|[text=black]|a_{31} & |[text=black]|a_{32} \\
	a_{41} & a_{42} \\
	};
	\matrix[mymatrix,right=of A.north east,anchor=north west] (prod)
	{
		 &  & \\
		 &  & \\
		 &  & \\
		 &  & \\
	};
	\matrix[mymatrix,above=of prod.north west,anchor=south west] (B)
	{
	b_{11} & |[text=black]|b_{12} & |[text=black]|b_{13} \\
	b_{21} & |[text=black]|b_{22} & |[text=black]|b_{23} \\
	};

	%the labels for the matrices
	\node[font=\huge,left=10pt of A] {$A$};
	\node[font=\huge,above=2pt of B] {$B$};

	%the frames in both matrices
	\draw[myyellow,line width=2pt]
	([shift={(1.2pt,-1.2pt)}]A-1-1.north west)
	rectangle
	([shift={(-1.2pt,1.2pt)}]A-1-2.south east);
	\draw[myyellow,line width=2pt]
	([shift={(1.2pt,-1.2pt)}]B-1-2.north west)
	rectangle
	([shift={(-1.2pt,1.2pt)}]B-2-2.south east);
	\draw[mygreen,line width=2pt]
	([shift={(1.2pt,-1.2pt)}]A-3-1.north west)
	rectangle
	([shift={(-1.2pt,1.2pt)}]A-3-2.south east);
	\draw[mygreen,line width=2pt]
	([shift={(1.2pt,-1.2pt)}]B-1-3.north west)
	rectangle
	([shift={(-1.2pt,1.2pt)}]B-2-3.south east);

	%the filled circles in the product
	\node[mycircle={myblue}{mygreen}]
	at (prod-3-3) (prod33) {};
	\node[mycircle={myred}{myyellow}]
	at (prod-1-2) (prod12) {};

	%the arrows
	\draw[arr=myred]
	(A-1-2.east) -- (prod12);
	\draw[arr=myred]
	(B-2-2.south) -- (prod12);
	\draw[arr=myblue]
	(A-3-2.east) -- (prod33);
	\draw[arr=myblue]
	(B-2-3.south) -- (prod33);

	%the legend
	\matrix[
	matrix of math nodes,
	nodes in empty cells,
	column sep=10pt,
	anchor=north,
	nodes={
			minimum height=2.2em,
			minimum width=2em,
			anchor=north east
		},
	right=30pt of current bounding box.east
	]
	(legend)
	{
	& a_{11}b_{12} + a_{12}b_{22} \\
	& a_{31}b_{13} + a_{32}b_{23} \\
	};
	\node[mycircle={myblue}{mygreen}]
	at (legend-2-1) {};
	\node[mycircle={myred}{myyellow}]
	at (legend-1-1) {};
\end{tikzpicture}
  \caption{Ilustração.}
\end{figure}

\[
  \begin{pNiceMatrix}
    a_{11} & \cdots & a_{1n} \\
    \vdots & \vdots & \vdots \\
    a_{m1} & \cdots & a_{mn}
  \end{pNiceMatrix}
  \begin{pNiceMatrix}
    b_{11} & \cdots & b_{1p} \\
    \vdots & \vdots & \vdots \\
    b_{n1} & \cdots & b_{np}
  \end{pNiceMatrix}=
  \begin{pNiceMatrix}
    c_{11} & \cdots & c_{1p} \\
    \vdots & \vdots & \vdots \\
    c_{m1} & \cdots & c_{mp}
  \end{pNiceMatrix}.
\]

\begin{example}
  Temos que
  \[
    {\begin{pNiceMatrix}
          1 & 2  \\
          2 & -1
        \end{pNiceMatrix}}_{2\times2}
      {\begin{pNiceMatrix}
          1 & 0 & 4 \\
          3 & 2 & 5
        \end{pNiceMatrix}}_{2\times3}=
    {\begin{pNiceMatrix}
      7  & 4  & 14 \\
      -1 & -2 & 3
    \end{pNiceMatrix}}_{2\times3}.
  \]
\end{example}

\begin{proposition}
  Sejam matrices $A={\left(a_{ij}\right)}_{m\times n}$,
  $B={\left(a_{jl}\right)}_{n\times p}$ e
  $C={\left(a_{lk}\right)}_{p\times q}$ matrizes sobre um corpo
  $\mathbb{F}$.
  Então $\left(AB\right)C=A\left(BC\right)$.
\end{proposition}

\begin{proof}
  Veja que
  \begin{math}
    \left(AB\right)C=
    {\left(\alpha_{ik}\right)}_{m\times q}
  \end{math},
  \begin{math}
    AB=
    {\left(d_{il}\right)}_{m\times p}
  \end{math}
  onde
  \begin{align*}
    d_{il}
     & =\sum_{l=1}^{n}
    a_{ij}b_{jl}
    \shortintertext{e}
    \alpha_{ik}
     & =\sum_{l=1}^{p}
    d_{il}c_{lk}=
    \sum_{l=1}^{p}
    \left(\sum_{j=1}^{n}a_{ij}b_{jl}\right)c_{lk}= \\
     & =\sum_{l=1}^{p}
    \left(\sum_{j=1}^{n}a_{ij}b_{jl}c_{lk}\right)= \\
     & =\sum_{j=1}^{n}
    a_{ij}\left(\sum_{l=1}^{p}b_{jl}c_{lk}\right)=\beta_{ik}
  \end{align*}
  com $A\left(BC\right)={\left(\beta_{ik}\right)}_{m\times q}$.
\end{proof}

Chamaremos a matriz quadrada
$I_{m}={\left(\delta_{ij}\right)}_{m\times m}$ definida por
\[
  \delta_{ij}=
  \begin{cases}
    1 & , \text{se }i=j,     \\
    0 & , \text{se }i\neq j,
  \end{cases}
\]
de matriz identidade de ordem $m\times m$.

Note que se $A={\left(a_{jl}\right)}_{m\times n}$, então $I_{m}A=A$, e se $B={\left(b_{li}\right)}_{n\times m}$, então $BI_{m}=B$.


$I_{m}A={\left(c_{il}\right)}_{m\times m}$ é tal que
$c_{il}=\sum_{j=1}^{m}\delta_{ij}a_{jl}=a_{il}$ con $1\leq i\leq m$,
e $I_{m}A=A$.

\begin{example}
  Se $m=3$, então
  \[
    I_{3}=
    \begin{pNiceMatrix}
      1 & 0 & 0 \\
      0 & 1 & 0 \\
      0 & 0 & 1
    \end{pNiceMatrix}.
  \]
\end{example}
Dizemos que uma matriz quadrada $A={\left(a_{ij}\right)}_{m\times m}$
tem inversa se existe uma matriz
$B={\left(b_{ij}\right)}_{m\times m}$ tal que $AB=BA=I_{m}$.

Denotaremos a matriz $B$ por $A^{-1}$.

\begin{definition}
  Seja $c\in\mathbb{F}\setminus\left\{0\right\}$.
  Uma matriz quadrada de ordem $m\times m$ $E$ é dita elementar se
  $E$ é de uma das formas
  \begin{enumerate}
    \item

          $E_{1}={\left(e_{ij}\right)}_{m\times m}$, onde
          \[
            e_{ij}=
            \begin{cases}
              \delta_{ij}, & \text{se }i\neq k \\
              \delta_{ij}, & \text{se }i=k
            \end{cases}\qquad
            \fbox{\begin{varwidth}{\textwidth}
                $m=3$, $k=2$\\[\baselineskip]
                \begin{math}
                  E_{1}=
                  \begin{pNiceMatrix}
                    1 & 0 & 0 \\
                    0 & c & 0 \\
                    0 & 0 & 1
                  \end{pNiceMatrix}
                \end{math}
              \end{varwidth}}
          \]
          con $k$ um inteiro fixo entre $1$ e $m$;
    \item

          $E_{2}={\left(e_{ij}\right)}_{m\times m}$, onde
          \[
            e_{ij}=
            \begin{cases}
              \delta_{ij}, & \text{se }i\neq l\text{ e }i\neq k \\
              \delta_{lj}, & \text{se }i=k                      \\
              \delta_{kj}, & \text{se }i=l
            \end{cases}\qquad
            \fbox{\begin{varwidth}{\textwidth}
                $m=3$, $k=2$, $l=3$\\[\baselineskip]
                \begin{math}
                  \begin{pNiceMatrix}
                    1 & 0 & 0 \\
                    0 & 0 & 1 \\
                    0 & 1 & 0
                  \end{pNiceMatrix}
                \end{math}
              \end{varwidth}}
          \]
          con $k<l$ inteiros fixos entre $1$ e $m$;
    \item

          $E_{3}={\left(e_{ij}\right)}_{m\times m}$, onde
          \[
            e_{ij}=
            \begin{cases}
              \delta_{i j},                  & i\neq k \\
              \delta_{kj}+c\cdot\delta_{lj}, & i=k
            \end{cases}\qquad
            \fbox{\begin{varwidth}{\textwidth}
                $m=3$, $k=2$, $l=3$\\[\baselineskip]
                \begin{math}
                  \begin{pNiceMatrix}
                    1 & 0 & 0 \\
                    0 & 1 & c \\
                    0 & 0 & 1
                  \end{pNiceMatrix}
                \end{math}
              \end{varwidth}}
          \]
  \end{enumerate}
\end{definition}

\begin{example}
  Calcule
  \[
    \begin{pNiceMatrix}
      1 & 0 & 0 \\
      0 & 1 & 2 \\
      0 & 0 & 1
    \end{pNiceMatrix}
    \begin{pNiceMatrix}
      1 & 2 & 3  & -1 \\
      2 & 2 & 1  & 1  \\
      1 & 1 & -1 & 2  \\
    \end{pNiceMatrix}=
    \begin{pNiceMatrix}
      1 & 2 & 3  & -1
      4 & 4 & -1 & 5
      1 & 1 & -1 & 2
    \end{pNiceMatrix}
  \]
\end{example}
Dada uma matriz $A={\left(a_{ij}\right)}_{m\times m}$ o efeito de multiplicar uma matriz elementar $E$ por $A$ pode ser colocado como:
\begin{enumerate}
  \item

        $E_{1}A$: multiplica uma linha $k$ de $A$ por um escalar $c$;

  \item

        $E_{2}A$: troca duas linhas $l$ e $k$ de posições
        $\left(k<l\right)$;

  \item

        $E_{3}A$: soma uma linha $k$ com outra linha $l$ multiplicada por um escalar $c\in\mathbb{F}$.
\end{enumerate}

% \[
%   E_{1}=
%   \left(e_{ij}\right)_{m\times m}\equiv
%   A=\left(a_{jl}\right)_{m\times n} \\
%   E_{1}A=
%   \left(c_{il}\right)_{m\times n}\text{com}
%   c_{il}=
%   \sum_{j=1}^{n}e_{ij}a_{je}=
%   a_{il}, i\neq k,
%   ca_{il},i=k
%   \underbrace{
%   a_{11} & \cdots & a_{14} \\
%   a_{m1} & \vdots & a_{mn}
%     _{A}
%   \underbrace{
%     x_{2} \\
%     \vdots \\
%     x_{n}
%   }_{X}=
%   \underbrace{
%     \gamma_{1} \\
%     \vdots \\
%     y_{m}
%   }_{y}
%   a_{1} x_{2}+\dotsb+a_{n} x_{n}=y_{2} \\
%   \vdots \\
%   a_{m 1} x_{2}+\dotsb+a_{w_{x}} x_{n}=y_{m}
% \]