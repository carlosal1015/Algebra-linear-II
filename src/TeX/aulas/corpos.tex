\chapter{Corpos e Sistemas Lineares\quad$\left(06/01/2021\right)$}

\begin{definition}[Corpo]

	Um \index{corpo}\emph{corpo} é um conjunto não vazio $\mathbb{F}$
	munido de duas operações: adição mais e multiplicação.

	\[
		\begin{aligned}
			+\colon\mathbb{F}\times\mathbb{F} & \longrightarrow\mathbb{F} \\
			\left(x,y\right)                  & \longmapsto x+y
		\end{aligned}\quad
		\begin{aligned}
			\cdot\colon\mathbb{F}\times\mathbb{F} & \longrightarrow\mathbb{F} \\
			\left(x,y\right)                      & \longmapsto x\cdot y
		\end{aligned}
	\]

	e tais que en $\left(\mathbb{F},+\right)$

	\begin{enumerate}[label={A\arabic*.},leftmargin=0em,itemindent=*]
		\item\label{adição:1}

		      (Asociatividade na adição)
		      $\left(x+y\right)+z=x+\left(y+z\right)$,
		      $\forall x,y,z\in\mathbb{F}$;

		\item\label{adição:2}

		      (Existênza de neutro aditivo)
		      $\exists0\in\mathbb{F}$ tal que $x+0=0+x=x$,
		      $\forall x\in\mathbb{F}$;

		\item\label{adição:3}

		      (Existênza de elemento oposto o inverso aditivo)
		      Dado $x\in\mathbb{F}$, existe $-x\in\mathbb{F}$ tal que
		      $x+\left(-x\right)=\left(-x\right)+x=0$.

		\item\label{adição:4}

		      (Conmutatividade na adição)
		      $x+y=y+x$, $\forall x,y\in\mathbb{F}$;

		\item\label{adição:5}

		      (Associatividade na multiplicação)
		      $\left(x\cdot y\right)z=x\cdot\left(y\cdot z\right)$,
		      $\forall x,y,z\in\mathbb{F}$;
	\end{enumerate}

	e $\left(\mathbb{F}\setminus\left\{0\right\},\cdot\right)$

	\begin{enumerate}[label={M\arabic*.},leftmargin=0em,itemindent=*]
		\item\label{multiplicação:1}

		      (Existênza do elemento neutro na multiplicação)
		      $\exists 1\in\mathbb{F}$ tal que $x\cdot 1=1\cdot x=x$, $\forall x\in\mathbb{F}$;

		\item\label{multiplicação:2}

		      (Existênza inverso multiplicativo)
		      Dado $x\in\mathbb{F}\setminus\left\{0\right\}$,
		      existe $x^{-1}\in\mathbb{F}$ tal que
		      $x\cdot x^{-1}=x^{-1}\cdot x=1$;

		\item\label{multiplicação:3}

		      (Conmutatividade na multiplicação)
		      $x\cdot y=y\cdot x$, $\forall x,y\in\mathbb{F}$.

		\item\label{multiplicação:4}

		      (Distributiva)
		      $x\cdot\left(y+z\right)=xy+xz$,
		      $\forall x,y,z\in\mathbb{F}$.
	\end{enumerate}
\end{definition}

\begin{proposition}
	$x\cdot0=0$, $\forall x\in\mathbb{F}$.
\end{proposition}

\begin{proof}
	\begin{math}
		x\cdot0=
		\overset{A2}{=}
		x\cdot\left(0+0\right)
		\overset{D}{=}
		x\cdot0+x\cdot0
	\end{math}.
	Assim
	\begin{align*}
		x\cdot0+
		\underbrace{x\cdot0+\left(-x\cdot0\right)}_{=0}
		        & =
		\underbrace{x\cdot0+\left(-x\cdot0\right)}_{=0} \\
		x\cdot0+
		0       & \overset{A3}{=}
		0                                               \\
		x\cdot0 & \overset{A2}{=}
		0.
	\end{align*}
\end{proof}

\begin{example}
	\begin{enumerate}\leavevmode
		\item

		      $\left(\mathbb{Z},+,\cdot\right)$ não é um corpo.
		      De fato não existe o inverso multiplicativo de $2$ em
		      $\mathbb{Z}$, ou seja, a equação $2\cdot x=1$ não se
		      resolue em $\mathbb{Z}$.

		\item

		      $\left(\mathbb{Q},+,\cdot\right)$ é um corpo, onde
		      \begin{math}
			      \mathbb{Q}=
			      \left\{
			      \frac{a}{b}\mid a,b\in\mathbb{Z},b\neq0
			      \right\}
		      \end{math}
		      e $\frac{a}{b}+\frac{c}{d}=\frac{ad+bc}{bd}$ e
		      $\frac{a}{b}\cdot\frac{c}{d}=\frac{ac}{bd}$.

		\item

		      $\left(\mathbb{R},+,\cdot\right)$ é um corpo (conjunto dos
		      números reais);

		\item

		      $\left(\mathbb{C},+,\cdot\right)$ é um corpo, onde
		      \begin{math}
			      \mathbb{C}=
			      \left\{
			      a+bi\mid a,b\in\mathbb{R},\text{ e }i^{2}=1
			      \right\}
		      \end{math},

		      \begin{math}
			      +\colon
			      \left(a+bi\right)+
			      \left(c+di\right)=
			      \left(a+c\right)+
			      \left(b+d\right)i
		      \end{math}.

		      \begin{math}
			      \cdot\colon
			      \left(a+bi\right)\cdot
			      \left(c+di\right)=
			      \left(ac-bd\right)+\left(ad+bc\right)i
		      \end{math}.

		      \begin{align*}
			      \left(a+bi\right)\left(c+di\right)
			       & =ac+adi+bci+bdi^{2}=                       \\
			       & =ac+\left(-1\right)bd+\left(ad+bc\right)i= \\
			       & =\left(ac-bd\right)+\left(ad+bc\right)i
		      \end{align*}
		      $\mathbb{C}$ é chamado del conjunto nos números complexos.
		      Tome $a+bi\in\mathbb{C}\setminus\left\{0\right\}$
		      $(0=0+0i)$.
		      Assim
		      \begin{align*}
			      \left(a+bi\right)\left(a-bi\right)
			       & =a^{2}+b^{2}+\left(ab-ba\right)i= \\
			       & =a^{2}+b^{2} \neq 0               \\
			      \left(a+bi\right)\left(a-bi\right)
			      \left(a^{2}+b^{2}\right)^{-1}
			       & =1
		      \end{align*}
		      Logo
		      $$
			      \left(a+bi\right)^{-1}=
			      \frac{a}{a^{2}+b^{2}}-
			      \frac{b}{a^{2}+b^{2}}i
		      $$

		\item

		      $\left(\mathbb{Z}/p\mathbb{Z},+,\cdot\right)$ é um corpo, onde $\mathbb{Z}/p\mathbb{Z}=\left\{\overline{a}\mid \overline{a}\in\mathbb{Z}\right\}$.

		      \[
			      \mathbb{Z}/p\mathbb{Z}=
			      \left\{\overline{a}/a\in\mathbb{Z}\right\}
			      \overline{a}=
			      \left\{a+pu/u\in\mathbb{Z}\right\}
			      \neq0\leq a\leq p-1
		      \]

	\end{enumerate}
\end{example}

Defina:

$$
	F=
	\left\{
	\overline{a}+\overline{b}\imath\mid
	\overline{a},\overline{b}\in\mathbb{Z}/3\mathbb{Z}
	\text{ e }\imath^{2}=\overline{2}
	\right\}.
	+\colon
	\left(\overline{a}+\overline{b}\imath\right)+
	\left(\overline{c}+\overline{d}\imath\right)=
	\left(\overline{a}+\overline{c}\right)+
	\left(\overline{b}+\overline{d}\right)i.
	\cdot\colon
	\left(\overline{a}+\overline{b}i\right)
	\left(\overline{c}+\overline{d}i\right)=
	\left(\overline{a}\cdot\overline{c}+\overline{2}\overline{b}d\right)+
	\left(\overline{a}\overline{d}+\overline{b}\overline{c}\right)i
$$
%https://www.learn-portuguese-with-rafa.com/portuguese-keyboard-characters.html

$\underbrace{1+\cdots+1}_{n}=0$

\[
	\begin{array}{l}
		\left\{\begin{array}{l}
			2 x+3 y=1 \\
			x+4 y=2 .
		\end{array}\right.                                                      \\
		+\left\{\begin{array}{l}
			2 x+3 y=1 \\
			-2 x-8 y=-4
		\end{array} \Rightarrow\left\{\begin{array}{l}
			2 x+3 y=1 \\
			-5 y=-3 \Rightarrow y=\frac{3}{5}
		\end{array}\right.\right. \\
		\begin{array}{l}
			2 x+3 \cdot \frac{3}{5}=1 \\
			2 x+\frac{9}{5}=1 \Rightarrow 2 x=-\frac{4}{5} \Rightarrow x=-\frac{2}{5}
		\end{array}
	\end{array}
\]

\[
	\left\{\begin{array}{l}
		\overline{2} x+\overline{2} y=\overline{1} \\
		\overline{2} x+y=\overline{0}
	\end{array}\right.
\]

\[
	\left\{\begin{aligned}
		\overline{2} x+\overline{2} y & =\overline{1} \\
		r_{y}                         & =\bar{I}
	\end{aligned}\right.
\]

\[
	\begin{aligned}
		\overline{2} x+\overline{2} \cdot I=1 \Rightarrow & 2 x=\overline{1}-\overline{2} \\
		                                                  & \overline{2} x=-\overline{1}  \\
		                                                  & \overline{2} x=\overline{2}   \\
		                                                  & x=\overline{1}
	\end{aligned}
\]
