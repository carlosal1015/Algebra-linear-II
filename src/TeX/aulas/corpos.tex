\chapter{Corpos}

\begin{definition}[Corpo]
	Um \emph{corpo} é um conjunto não vazio $\mathbb{F}$ munido de duas
	operacões: adição mais e multiplicação
	\[ +\colon\mathbb{F}\times\mathbb{F}\left(x,y\right)\mapsto x+y \]
	\[ \cdot\colon\mathbb{F}\times\mathbb{F}\left(x,y\right)\mapsto x\cdot y \]
	e tais que
	\begin{enumerate}
		\item[Asociatividade na adição]

		      $\left(x+y\right)+z=x+\left(y+z\right)$, $\forall x,y,z\mathbb{F}$;

		\item[Existênza de neutro aditivo]

		      $\exists0\in\mathbb{F}$ tal que $x+0=0+x=x$, $\forall x\in\mathbb{F}$;

		\item[Existênza de elemento oposto o inverso aditivo]

		      Dado $x\in\mathbb{F}$, existe $-x\in\mathbb{F}$ tal que $x+\left(-x\right)=\left(-x\right)+x=0$.

		\item[Conmutatividade na adição]

		      $x+y=y+x$, $\forall x,y\in\mathbb{F}$;

		\item[Associatividade na multiplicação]

		      $\left(x\cdot y\right)z=x\cdot\left(y\cdot z\right)$, $\forall x,y,z\in\mathbb{F}$;

		\item[Existênza do elemento neutro na multiplicação]

		      $\exists 1\in\mathbb{F}$ tal que $x\cdot 1=1\cdot x=x$, $\forall x\in\mathbb{F}$;

		\item[Existênza inverso multiplicativo]

		      Dado $x\in\mathbb{F}\setminus\left\{0\right\}$, existe $x^{-1}\in\mathbb{F}$ tal que $x\cdot x^{-1}=x^{-1}\cdot x=1$;

		\item[Conmutatividade na multiplicação]

		      $x\cdot y=y\cdot x$, $\forall x,y\in\mathbb{F}$.

		\item[Distributiva]

		      $x\cdot\left(y+z\right)=xy+xz$, $\forall x,y,z\in\mathbb{F}$.
	\end{enumerate}
\end{definition}

\begin{proposition}
	$x\cot0=0$, $\forall x\in\mathbb{F}$.
\end{proposition}

\begin{example}
	\begin{enumerate}
		\item

		      $\left(\mathbb{Z},+,\cdot\right)$ não é um corpo.

		      O seja el conjunto.

		\item

		      $\left(\mathbb{Q},+,\cdot\right)$ é um corpo, onde
		      $\mathbb{Q}=\left\{\frac{a}{b}\mid a,b\in\mathbb{Z},b\neq0\right\}$.

		\item

		      $\left(\mathbb{C},+,\cdot\right)$ é um corpo, onde
		      $\mathbb{C}=\left\{a+b\imath\mid a,b\in\mathbb{R},\text{ e }\imath^{2}=1\right\}$,
		      $+\colon\left(a+b\imath\right)+\left(c+d\imath\right)=\left(a+c\right)+\left(b+d\right)\imath$
		      $\cdot\colon\left(a+b\imath\right)\cdot\left(c+d\imath\right)=\left(ac-bd\right)+\left(ad+bc\right)\imath$.

		\item

		      $\left(\mathbb{Z}/p\mathbb{Z},+,\cdot\right)$ é um corpo, onde $\mathbb{Z}/p\mathbb{Z}=\left\{\overline{a}\mid \overline{a}\in\mathbb{Z}\right\}$.
	\end{enumerate}
\end{example}

Defina:

$$
	F=
	\left\{
	\overline{a}+\overline{b}\imath\mid
	\overline{a},\overline{b}\in\mathbb{Z}/3\mathbb{Z}
	\text{ e }\imath^{2}=\overline{2}
	\right\}.
$$
%https://www.learn-portuguese-with-rafa.com/portuguese-keyboard-characters.html

