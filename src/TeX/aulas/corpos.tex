\chapter{Corpos e Sistemas Lineares\quad$\left(06/01/2021\right)$}

\begin{definition}[Corpo]

	Um \index{corpo}\emph{corpo} é um conjunto não vazio $\mathbb{F}$
	munido de duas operações: adição mais e multiplicação.

	\[
		\begin{aligned}
			+\colon\mathbb{F}\times\mathbb{F} & \longrightarrow\mathbb{F} \\
			\left(x,y\right)                  & \longmapsto x+y
		\end{aligned}\qquad
		\begin{aligned}
			\cdot\colon\mathbb{F}\times\mathbb{F} & \longrightarrow\mathbb{F} \\
			\left(x,y\right)                      & \longmapsto x\cdot y
		\end{aligned}
	\]

	e tais que en $\left(\mathbb{F},+\right)$

	\begin{enumerate}[label={(A\arabic*)},leftmargin=0em,itemindent=*]
		\item\label{adição:1}

		      (Asociatividade na adição)
		      $\left(x+y\right)+z=x+\left(y+z\right)$,
		      $\forall x,y,z\in\mathbb{F}$;

		\item\label{adição:2}

		      (Existênza de neutro aditivo)
		      $\exists0\in\mathbb{F}$ tal que $x+0=0+x=x$,
		      $\forall x\in\mathbb{F}$;

		\item\label{adição:3}

		      (Existênza de elemento oposto o inverso aditivo)
		      Dado $x\in\mathbb{F}$, existe $-x\in\mathbb{F}$ tal que
		      $x+\left(-x\right)=\left(-x\right)+x=0$;

		\item\label{adição:4}

		      (Conmutatividade na adição)
		      $x+y=y+x$, $\forall x,y\in\mathbb{F}$;
	\end{enumerate}

	e $\left(\mathbb{F}\setminus\left\{0\right\},\cdot\right)$

	\begin{enumerate}[label={(M\arabic*)},leftmargin=0em,itemindent=*]
		\item\label{multiplicação:1}

		      (Associatividade na multiplicação)
		      $\left(x\cdot y\right)\cdot z=x\cdot\left(y\cdot z\right)$,
		      $\forall x,y,z\in\mathbb{F}$;


		\item\label{multiplicação:2}

		      (Existênza do elemento neutro na multiplicação)
		      $\exists 1\in\mathbb{F}$ tal que $x\cdot 1=1\cdot x=x$, $\forall x\in\mathbb{F}$;

		\item\label{multiplicação:3}

		      (Existênza inverso multiplicativo)
		      Dado $x\in\mathbb{F}\setminus\left\{0\right\}$,
		      existe $x^{-1}\in\mathbb{F}$ tal que
		      $x\cdot x^{-1}=x^{-1}\cdot x=1$;

		\item\label{multiplicação:4}

		      (Conmutatividade na multiplicação)
		      $x\cdot y=y\cdot x$, $\forall x,y\in\mathbb{F}$;
	\end{enumerate}

	\begin{enumerate}[label={(D)},leftmargin=0em,itemindent=*]
		\item\label{distributiva}

		      (Distributiva)
		      $x\cdot\left(y+z\right)=x\cdot y+x\cdot z$,
		      $\forall x,y,z\in\mathbb{F}$.
	\end{enumerate}
\end{definition}

\begin{proposition}
	$x\cdot0=0$, $\forall x\in\mathbb{F}$.
\end{proposition}

\begin{proof}
	\begin{math}
		x\cdot0=
		\overset{A2}{=}
		x\cdot\left(0+0\right)
		\overset{D}{=}
		x\cdot0+x\cdot0
	\end{math}.
	Assim
	\begin{align*}
		x\cdot0+
		\underbrace{x\cdot0+\left(-x\cdot0\right)}_{=0}
		        & =
		\underbrace{x\cdot0+\left(-x\cdot0\right)}_{=0} \\
		x\cdot0+
		0       & \overset{A3}{=}
		0                                               \\
		x\cdot0 & \overset{A2}{=}
		0.
	\end{align*}
\end{proof}

\begin{example}
	\begin{enumerate}[a)]\leavevmode
		\item

		      $\left(\mathbb{Z},+,\cdot\right)$ não é um corpo.
		      De fato não existe o inverso multiplicativo de $2$ em
		      $\mathbb{Z}$, ou seja, a equação $2\cdot x=1$ não se
		      resolue em $\mathbb{Z}$;

		\item

		      $\left(\mathbb{Q},+,\cdot\right)$ é um corpo, onde
		      \begin{math}
			      \mathbb{Q}=
			      \left\{
			      \frac{a}{b}\mid a,b\in\mathbb{Z},b\neq0
			      \right\}
		      \end{math}
		      e $\frac{a}{b}+\frac{c}{d}=\frac{ad+bc}{bd}$ e
		      $\frac{a}{b}\cdot\frac{c}{d}=\frac{ac}{bd}$.

		\item

		      $\left(\mathbb{R},+,\cdot\right)$ é um corpo (conjunto dos
		      números reais);

		\item

		      $\left(\mathbb{C},+,\cdot\right)$ é um corpo, onde
		      \begin{math}
			      \mathbb{C}=
			      \left\{
			      a+bi\mid a,b\in\mathbb{R},\text{ e }i^{2}=1
			      \right\}
		      \end{math},

		      \[
			      \begin{aligned}
				      +\colon\mathbb{C}\times\mathbb{C}                & \longrightarrow\mathbb{C} \\
				      \left(\left(a+bi\right),\left(c+di\right)\right) & \longmapsto
				      \left(a+c\right)+
				      \left(b+d\right)i
			      \end{aligned}\qquad
			      \begin{aligned}
				      \cdot\colon\mathbb{C}\times\mathbb{C}            & \longrightarrow\mathbb{C} \\
				      \left(\left(a+bi\right),\left(c+di\right)\right) & \longmapsto
				      \left(ac-bd\right)+
				      \left(ad+bc\right)i
			      \end{aligned}
		      \]

		      \begin{align*}
			      \left(a+bi\right)\left(c+di\right)
			       & =ac+adi+bci+bdi^{2}=                       \\
			       & =ac+\left(-1\right)bd+\left(ad+bc\right)i= \\
			       & =\left(ac-bd\right)+\left(ad+bc\right)i.
		      \end{align*}

		      $\mathbb{C}$ é chamado del conjunto nos números complexos.
		      Tome $a+bi\in\mathbb{C}\setminus\left\{0\right\}$
		      $(0=0+0i)$.

		      Assim

		      \begin{align*}
			      \left(a+bi\right)\left(a-bi\right)
			       & =a^{2}+b^{2}+\left(ab-ba\right)i= \\
			       & =a^{2}+b^{2} \neq 0               \\
			      \left(a+bi\right)
			      \underbrace{
				      \left(a-bi\right)
				      {\left(a^{2}+b^{2}\right)}^{-1}
			      }_{}
			       & =1.
		      \end{align*}

		      Logo
		      \[
			      {\left(a+bi\right)}^{-1}=
			      \frac{a}{a^{2}+b^{2}}-
			      \frac{b}{a^{2}+b^{2}}i.
		      \]

		\item

		      $\left(\mathbb{Z}/p\mathbb{Z},+,\cdot\right)$ é um corpo,
		      onde $p$ é primo e
		      \begin{math}
			      \mathbb{Z}/p\mathbb{Z}=
			      \left\{
			      \overline{a}\mid a\in\mathbb{Z}
			      \right\},
			      \overline{a}=
			      \left\{
			      a+pn\mid n\in\mathbb{Z}
			      \right\}
		      \end{math}
		      e $0\leq a\leq p-1$.

		      \[
			      \begin{aligned}
				      +\colon\mathbb{Z}/p\mathbb{Z}\times\mathbb{Z}/p\mathbb{Z} & \longrightarrow\mathbb{Z}/p\mathbb{Z} \\
				      \left(\overline{a},\overline{b}\right)                    & \longmapsto
				      \overline{a+b}
			      \end{aligned}\qquad
			      \begin{aligned}
				      \cdot\colon\mathbb{Z}/p\mathbb{Z}\times\mathbb{Z}/p\mathbb{Z} & \longrightarrow\mathbb{Z}/p\mathbb{Z} \\
				      \left(\overline{a},\overline{b}\right)                        & \longmapsto
				      \overline{a\cdot b}
			      \end{aligned}
		      \]
		      Tome $p=3$.
		      Assim
		      \begin{math}
			      \mathbb{Z}/3\mathbb{Z}=
			      \left\{
			      \overline{0},
			      \overline{1},
			      \overline{2}
			      \right\}
		      \end{math}.

		      \begin{tabular}{|>{$}c<{$}|>{$}c<{$}|>{$}c<{$}|>{$}c<{$}|}
			      \hline
			      +            & \overline{0} & \overline{1} & \overline{2} \\
			      \hline
			      \overline{0} & \overline{0} & \overline{1} & \overline{2} \\
			      \hline
			      \overline{1} & \overline{1} & \overline{2} & \overline{0} \\
			      \hline
			      \overline{2} & \overline{2} & \overline{0} & \overline{1} \\
			      \hline
		      \end{tabular}

		      \begin{tabular}{|>{$}c<{$}|>{$}c<{$}|>{$}c<{$}|}
			      \hline
			      \cdot        & \overline{1} & \overline{2} \\
			      \hline
			      \overline{1} & \overline{1} & \overline{2} \\
			      \hline
			      \overline{2} & \overline{2} & \overline{1} \\
			      \hline
		      \end{tabular}
		      $\overline{2}+\overline{2}=\overline{2+2}=\overline{4}=\overline{3\cdot1+1}=\overline{1}$.
	\end{enumerate}
\end{example}

Note que a equação $x^{2}+\overline{1}=\overline{0}$ não tem solução
em $\left(\mathbb{Z}/p\mathbb{Z},+,\cdot\right)$.

Defina:
\begin{math}
	F=
	\left\{
	\overline{a}+
	\overline{b}i\mid
	\overline{a},\overline{b}\in\mathbb{Z}/3\mathbb{Z}
	\text{ e }
	i^{2}=
	\overline{2}
	\right\}
\end{math}.

\[
	\begin{aligned}
		+\colon\mathbb{F}\times\mathbb{F} & \longrightarrow\mathbb{F} \\
		\left(
		\overline{a}+\overline{b}i,
		\overline{c}+\overline{d}i
		\right)                           & \longmapsto
		\left(\overline{a}+\overline{c}\right)+
		\left(\overline{b}+\overline{d}\right)i
	\end{aligned}\qquad
	\begin{aligned}
		\cdot\colon\mathbb{F}\times\mathbb{F} & \longrightarrow\mathbb{F} \\
		\left(
		\overline{a}+\overline{b}i,
		\overline{c}+\overline{d}i
		\right)                               & \longmapsto
		\left(
		\overline{a}\cdot\overline{c}+
		2\overline{b}\cdot\overline{d}
		\right)+
		\left(
		\overline{a}\cdot\overline{d}+
		\overline{b}\cdot\overline{c}
		\right)i
	\end{aligned}
\]
Mostre que $\left(\mathbb{F},+,\cdot\right)$ é um corpo com $9$ elementos.
%https://www.learn-portuguese-with-rafa.com/portuguese-keyboard-characters.html

\begin{definition}
	A característica de um corpo $\mathbb{F}$ é o menor inteiro positivo $n$ (se existir) tal que $\underbrace{1+\cdots+1}_{n}=0$.

	Se tal $n$ não existe, diremos que $F$ tem característica $0$.
\end{definition}

\begin{proposition}
	Seja $\mathbb{F}$ um corpo.
	Sea característica de $F$ é um inteiro positivo $n$, então $n$ é primo.
\end{proposition}

\begin{proof}
	Exercízio.
\end{proof}

\begin{example}
	\begin{enumerate}[a)]\leavevmode
		\item

		      Resolva em $\mathbb{Q}$ o sistema

		      \[
			      \begin{cases}
				      2 x+3 y & =1, \\
				      x+4 y   & =2.
			      \end{cases}
		      \]

		      % 2 x+3 y=1 \\
		      % 		-2 x-8 y=-4
		      % 	\Rightarrow
		      % 		2 x+3 y=1 \\
		      % 		-5 y=-3 \Rightarrow y=\frac{3}{5}
		      % 		2 x+3 \cdot \frac{3}{5}=1 \\
		      % 		2 x+\frac{9}{5}=1 \Rightarrow 2 x=-\frac{4}{5} \Rightarrow x=-\frac{2}{5}

		      \[
			      \begin{cases}
				      \overline{2}x+\overline{2}y= & \overline{1} \\
				      \overline{2}x+y=             & \overline{0}
			      \end{cases}
		      \]

		      \[
			      \begin{cases}
				      \overline{2}x+\overline{2}y & =\overline{1} \\
				      r_{y}                       & =\bar{I}
			      \end{cases}
		      \]
		\item Resolva em $\mathbb{Z}/3\mathbb{Z}$ o sistema
		      \[
			      \begin{aligned}
				      \overline{2} x+\overline{2} \cdot I=1 \Rightarrow & 2 x=\overline{1}-\overline{2} \\
				                                                        & \overline{2} x=-\overline{1}  \\
				                                                        & \overline{2} x=\overline{2}   \\
				                                                        & x=\overline{1}
			      \end{aligned}
		      \]
		      Daí $\left(\overline{1},\overline{1}\right)$ é solução do sistema.
	\end{enumerate}
\end{example}