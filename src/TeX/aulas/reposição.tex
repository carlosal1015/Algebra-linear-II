\chapter{Aula de reposição\quad$\left(09/01/2021\right)$}

\begin{definition}[Matriz reducida por linhas]
  Uma matriz $A={\left(a_{ij}\right)}_{m\times n}$ sobre $\mathbb{F}$
  é deja reduzida por linhas se

  \begin{enumerate}
    \item

          O primeiro elemento não nulo de cada linha não nula é
          igual $1$;

    \item

          cada columna que possui o primeiro elemento não nulo de

    \item

          uma linha não possui todos os outros elementos iguais a
          $0$;
  \end{enumerate}


  Sea além disso, esa matriz $A$ satisfaz

  \begin{enumerate}

    \item

          Todas linhas nulas ocorrem abaixo das linhas não nulas;

    \item

          Se $1,\ldots,r$ $\left(r\leq m\right)$ são as linhas não
          nulas de $A$ com os primeiros elementos não nunos ocurrendo
          nas colunas $k_{1},k_{2},k_{r}$, respectivamente, então
          $k_{1}<k_{2}<\cdots<k_{r}$, dizemos que $A$ está na forma
          escada reduzida.
  \end{enumerate}
\end{definition}

\begin{example}
  \begin{enumerate}
    \item As seguintes matrizes estão na forma reduzida:
          \begin{enumerate}
            \item
                  \begin{math}
                    \begin{pmatrix}
                      0 & 1 & 2 & 3 \\
                      1 & 0 & 1 & 1
                    \end{pmatrix},
                  \end{math}.
            \item

                  \begin{math}
                    \begin{pmatrix}
                      0 & 0 & 0 \\
                      1 & 0 & 1 \\
                      0 & 1 & 0
                    \end{pmatrix},
                  \end{math}.

            \item

                  \begin{math}
                    \begin{pmatrix}
                      0 & 0 & 0 \\
                      0 & 0 & 0 \\
                      0 & 0 & 0
                    \end{pmatrix},
                  \end{math}.

            \item

                  \begin{math}
                    \begin{pmatrix}
                      1 & 0 & 0 & 0 \\
                      0 & 1 & 1 & 2 \\
                      0 & 0 & 0 & 0
                    \end{pmatrix},
                  \end{math}.
          \end{enumerate}

    \item As seguientes matrizes estão na forma escada reducida

          \begin{enumerate}
            \item

                  \begin{math}
                    \begin{pmatrix}
                      1 & 0 & 2 & 0 \\
                      0 & 1 & 3 & 0 \\
                      0 & 0 & 0 & 1
                    \end{pmatrix}
                  \end{math}.

            \item

                  \begin{math}
                    \begin{pmatrix}
                      1 & 0 \\
                      0 & 1 \\
                      0 & 0
                    \end{pmatrix}
                  \end{math}.


            \item

                  \begin{math}
                    \begin{pmatrix}
                      1 & 2 & 3 & 0 & 1 \\
                      0 & 0 & 0 & 1 & 0
                    \end{pmatrix}
                  \end{math}.
          \end{enumerate}

  \end{enumerate}
\end{example}

\begin{remark}
  .
\end{remark}