\chapter*{Introdução ao curso\quad$\left(04/01/2021\right)$}

O professor \href{http://buscatextual.cnpq.br/buscatextual/visualizacv.do?id=K4463841D3}{Alex Carrazedo Dantas} é especialista no \href{https://pt.wikipedia.org/wiki/Teoria_dos_grupos}{
  \emph{Teoria dos grupos}}.
Em um curso presencial você pode discutir mais, enquanto em um curso remoto, cada aula tem um pdf \href{https://moodle.mat.unb.br/20201}{
  Moodle MAT} e uma gravação da sessão.
Se você tiver dúvidas sobre o moodle, peça ajuda a
\href{http://buscatextual.cnpq.br/buscatextual/visualizacv.do?id=K8136975Z1}{Carol Lafetá}\footnote{\href{mailto:lafeta.carol@gmail.com}{lafeta.carol\MVAt gmail.com}}.
\section*{Ementa}

\begin{enumerate}\bfseries
  \begin{multicols}{2}
    \item

    Sistemas lineares e matrizes.

    \item

    Espaços vetoriais e transformações lineares.

    \item

    Polinômios e determinantes

    \item

    Decomposicões primárias e formas racionais e de Jordan.

    \item

    Produto interno e teorema espectral.

    \item

    Formas multilineares.
  \end{multicols}
\end{enumerate}

\section*{Critério de avaliação}
%https://introducaocomunicacao.wordpress.com/avaliacao/
\begin{table}[ht!]
  \centering
  \begin{tabular}{c|c}
    \textbf{Menção em disciplina} & \textbf{Equivalência numérica} \\
    \hline
    Superior (SS)                 & $9-10$                         \\
    Média Superior (MS)           & $7-8.9$                        \\
    Média (MM)                    & $5-6.9$
  \end{tabular}
\end{table}
Serão aplicadas $2$ provas, de acordo com o cronograma abaixo, as
quais serão atribuı́das as notas $x$ e $y$.
\[
  \text{MF}=\frac{x+3y}{4}.
\]
O aluno deverá obter média final igual ou superior a $5$ pontos e
$75\%$ de frequência para ser aprovado.
\section*{Tutores}

\begin{itemize}
  \begin{multicols}{3}
    \item\href{https://www.escavador.com/sobre/7335541/sara-raissa-silva-rodrigues}{Sara Raissa Silva Rodrigues}.
    \item \href{https://www.escavador.com/sobre/5504138/geraldo-herbert-beltrao-de-souza}{Geraldo Herbert Beltrão de Souza}.
    \item\href{https://www.escavador.com/sobre/5634752/mattheus-pereira-da-silva-aguiar}{Mattheus Pereira da Silva Aguiar}.
  \end{multicols}
\end{itemize}

\vfill
\nocite{*}
\printbibliography[
  title={\textcolor{DarkBlue}{Referências bibliográficas}{\fontspec[Renderer=Harfbuzz]{NotoColorEmoji.ttf}📚}},
  heading=bibintoc
]