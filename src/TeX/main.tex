% arara: clean: {
% arara: --> extensions:
% arara: --> ['aux','log','out','blg','idx','ind',
% arara: --> 'toc','bbl', 'bcf','run.xml','ilg','pdf']
% arara: --> }
% arara: lualatex: {
% arara: --> shell: 1,
% arara: --> interaction: batchmode
% arara: --> }
% arara: biber
% arara: makeindex
% arara: lualatex: {
% arara: --> shell: 1,
% arara: --> interaction: batchmode
% arara: --> }
% arara: lualatex: {
% arara: --> shell: 1,
% arara: --> interaction: batchmode
% arara: --> }
% arara: clean: {
% arara: --> extensions:
% arara: --> ['aux','log','out','blg','idx','ind',
% arara: --> 'toc','bbl', 'bcf','run.xml','ilg']
% arara: --> }

\RequirePackage[leqno]{mathtools}%fleqn
\documentclass[
  fontsize=12pt,
  answers,
  addpoints,
  pagesize,
  paper=landscape,
  a5,
  DIV=calc,
  headsepline,
  footsepline,
  % abstract=true,
  bibliography=totoc
]{kommaexam}

\usepackage[margin={.43cm}]{geometry}%,.43cm
\usepackage[brazilian]{babel}
\usepackage{fontspec}%unicode-math,mathpazo,pdfpages
\usepackage{newunicodechar}
\usepackage{csquotes}
\usepackage[useregional]{datetime2}
\usepackage[svgnames]{xcolor}
\usepackage{graphicx,svg}
\usepackage{caption,subcaption}
\usepackage[shortlabels]{enumitem}
\usepackage{systeme,amssymb,amsthm}
\usepackage[lite]{mtpro2}
\usepackage[chapter]{minted}
\usepackage{background}
\usepackage{tikz}
\usepackage{varwidth}
\usepackage{booktabs}
\usepackage{nicematrix}
\usepackage{bookmark}
\usepackage{bbm}

\usepackage{imakeidx}
\makeindex[
  columns=2,
  title=Índice,
  options= -s header.ist,
  intoc
]

\usepackage[
  citestyle=numeric,
  style=numeric,
  backend=biber,
  maxnames=5,
  minnames=3
]{biblatex}
\usepackage[usefamily={julia,juliacon}]{pythontex}
\usepackage{hyperref}

\makeatletter
\patchcmd{\PEX@}{\dp\Pbox@>\dp\z@}{\ht\Pbox@>\dp\z@}{}{}
\patchcmd{\SQEX@}{\dp\Sbox@>\dp0}{\ht\Sbox@>\dp0}{}{}
\makeatother

\makeatletter
% Fix weird space
\patchcmd{\LEFTRIGHT}
{\kern-2\nulldelimiterspace\mskip-\thinmuskip}
{\kern-2\nulldelimiterspace}
{}{}
% Two new boxes
\newsavebox{\mtp@matrixbox}
\newsavebox{\mtp@casesbox}
% Round parentheses should always be used by default
% `pmatrix' from `amsmath'
\renewenvironment{pmatrix}{%
  \matrix@check\pmatrix\setbox\mtp@matrixbox=\hbox\bgroup$\env@matrix
}{%
  \endmatrix$\egroup\PARENS{\copy\mtp@matrixbox}%
}
% Curly braces are used only if `curlybraces' is set
% From `mtpro2.sty': \DeclareOption{curlybraces}{\let\mtp@br=c}
\ifx\mtp@br c
  % `Bmatrix' from `amsmath'
  \renewenvironment{Bmatrix}{%
    \setbox\mtp@matrixbox=\hbox\bgroup$\env@matrix
  }{%
    \endmatrix$\egroup\LEFTRIGHT\lbrace\rbrace{\copy\mtp@matrixbox}%
  }
  % `cases' from `amsmath'
  \renewcommand*\env@cases{%
  \let\@ifnextchar\new@ifnextchar
  \setbox\mtp@casesbox=\hbox\bgroup$%
    \def\arraystretch{1.2}%
    \array{@{}l@{\quad}l@{}}%
    }
    \renewenvironment{cases}{%
    \matrix@check\cases\env@cases
    }{%
    \endarray$\egroup\LEFTRIGHT\lbrace.{\copy\mtp@casesbox}%
  }
\fi
% Now, the matrices from `mathtools'
\MHInternalSyntaxOn
\MaybeMHPrecedingSpacesOff
% `pmatrix*' from `mathtools'
\renewenvironment{pmatrix*}[1][c]
{\setbox\mtp@matrixbox=\hbox\bgroup$\MT_matrix_begin:N #1}
{\MT_matrix_end:$\egroup\PARENS{\copy\mtp@matrixbox}}
\MH_if_meaning:NN \mtp@br c
% `Bmatrix*' from `mathtools'
\renewenvironment{Bmatrix*}[1][c]
{\setbox\mtp@matrixbox=\hbox\bgroup$\MT_matrix_begin:N #1}
{\MT_matrix_end:$\egroup\LEFTRIGHT\lbrace\rbrace{\copy\mtp@matrixbox}}
\MH_fi:
\MHPrecedingSpacesOn
\MHInternalSyntaxOff
\makeatother
% Patches end

\NewDocumentCommand{\csysteme}{som}{%
  \LEFTRIGHT\{.{%
      \sysdelim..
      \IfBooleanTF{#1}
      {\IfNoValueTF{#2}{\systeme*{#3}}{\systeme*[#2]{#3}}}
      {\IfNoValueTF{#2}{\systeme{#3}}{\systeme[#2]{#3}}}%
    }%
}

%\setlength\columnsep{10pt} % This is the default columnsep for all pages
\setmainfont{TeX Gyre Pagella}
\setmonofont{Fira Code}[Contextuals=Alternate]

\setkomafont{title}{\bfseries\sffamily\Huge\color{DarkBlue}}
\setkomafont{subtitle}{\bfseries\sffamily\Large\color{DarkBlue}}
\setkomafont{author}{\bfseries\sffamily\large\color{DarkBlue}}
\setkomafont{date}{\bfseries\sffamily\color{DarkBlue}}
\setkomafont{chapter}{\LARGE\color{DarkBlue}}
\setkomafont{section}{\Large\color{DarkBlue}}
\setkomafont{subsection}{\large\color{DarkBlue}}

\urlstyle{rm}

\newunicodechar{∈}{\makebox[\fontcharwd\font`a]{\(\in\)}}
\newunicodechar{⊗}{\makebox[\fontcharwd\font`a]{\(\otimes\)}}

\usetikzlibrary{matrix,arrows.meta,positioning}

\definecolor{myyellow}{RGB}{240,217,1}
\definecolor{mygreen}{RGB}{143,188,103}
\definecolor{myred}{RGB}{234,38,40}
\definecolor{myblue}{RGB}{53,101,167}

\newcommand{\MVAt}{{\usefont{U}{mvs}{m}{n}\symbol{`@}}}
\renewcommand{\qedsymbol}{\(\blacksquare\)}

\newcounter{tmp}

\newcommand\tikzmark[1]{%
  \tikz[remember picture,baseline=-0.65ex]
  \node[inner sep=0,outer sep=0] (#1){};%
}

\newcommand\mess[4][25pt]{%
  \stepcounter{tmp}%
  \begin{tikzpicture}[remember picture,overlay,>=latex,xshift=#1,cyan]
    \node[cyan,inner sep=2pt] at ([xshift=#1]#2) (a\thetmp) {$#4$};%circle,draw,
    \draw[->] (a\thetmp.south) |- (#3);
  \end{tikzpicture}%
}

\renewcommand{\solutiontitle}{\noindent\textbf{Solução}\par\noindent}
\definecolor{SolutionColor}{rgb}{0.2,0.9,1}
\colorsolutionboxes
\definecolor{SolutionBoxColor}{rgb}{0.2,0.9,1}
\vqword{Questão}
\vpword{Pontos}
\vsword{Pontuaçõe}

\theoremstyle{definition}
\newtheorem{definition}{Definição}[chapter]
\newtheorem{theorem}[definition]{Teorema}
\newtheorem{proposition}[definition]{Proposição}
\newtheorem{example}[definition]{Exemplo}
\newtheorem{exercise}[definition]{Exercício}
\newtheorem{remark}[definition]{Observação}

\graphicspath{ {img/} }

% \newmintedfile[juliacode]{julia}{
%   breaklines,
%   breakanywhere,
%   tabsize=4,
%   samepage=false,
%   showspaces=false,
%   showtabs =false
% }

\backgroundsetup{
  scale=1,
  angle=0,
  opacity=.1,
  contents={
      \begin{tikzpicture}[remember picture,overlay]
        \node at ([yshift=11pt,xshift=5pt]current page.center) {\includegraphics[width=8.5cm]{unb}};
      \end{tikzpicture}}
}

\hypersetup{
  pdfinfo={
      Title={Álgebra linear~Verão 2021~Notas das aulas na escola},
      Author={Alex Dantas et al.},
      Keywords={corpos, sistemas lineares},
      Subject={Álgebra linear},
      Producer={TeXstudio 3.0.4 Using Qt Version 5.15.2},
      Creator={LuaHBTeX, Version 1.12.0 (TeX Live 2020/Arch Linux)},
    }
  %	hyperref,
  pdfencoding=auto,
  linktocpage=true,
  colorlinks=true,
  linkcolor=NavyBlue,
  urlcolor=magenta,
  linktoc=all,
  pdfpagelabels,
  bookmarks=true,
  unicode,
  pdfborder = {0 0 0}
  %	filecolor = red,
}

\addbibresource{referências.bib}


\title{Álgebra Linear II}

\subtitle{
  \href{https://www.mat.unb.br/verao2021/verao/verao_pt.html}{
    XLIX Escola de Verão en Matemática da UnB
  }
}

\titlehead{\Huge
\[
  J_{j}=
  \begin{pNiceMatrix}
    \lambda_{j} & 1           &        &             & 0           \\
                & \lambda_{j} & 1      &             &             \\
                &             & \ddots & \ddots      &             \\
                &             &        & \lambda_{j} & 1           \\
    0           &             &        &             & \lambda_{j}
  \end{pNiceMatrix}
  \in\mathbb{C}^{s_{j}\times s_{j}}.
\]
}

\author{Aulas do professor Alex Carrazedo Dantas\thanks{alex\MVAt mat.unb.br}}

\date{
  Última modificação: \today{} às $\DTMcurrenttime$.
  \\[\baselineskip]
  \url{https://carlosal1015.github.io/Algebra-linear-II/pdf/main.pdf}
}

\begin{document}

\maketitle
\tableofcontents

\chapter*{Introdução ao curso}

El profesor Alex Dantas se especializa en la rama de las matemáticas
llamada \href{https://pt.wikipedia.org/wiki/Teoria_dos_grupos}{
  \emph{Teoria dos grupos}}.

En un curso presencial se puede discutir mais, en cambio en un curso remote cada aula un pdf \href{https://moodle.mat.unb.br/20201}{
  Moodle MAT} y sesión grabado.

\vfill
\nocite{*}
\printbibliography[
  title={\textcolor{DarkBlue}{Referências bibliográficas}{\fontspec[Renderer=Harfbuzz]{NotoColorEmoji.ttf}📚}},
  heading=bibintoc
]

\setpartpreamble{\Huge

\

\[
  J_{j}=
  \begin{pmatrix}
    \lambda_{j} & 1           &        &             & 0           \\
                & \lambda_{j} & 1      &             &             \\
                &             & \ddots & \ddots      &             \\
                &             &        & \lambda_{j} & 1           \\
    0           &             &        &             & \lambda_{j}
  \end{pmatrix}
  \in\mathbb{C}^{s_{j}\times s_{j}}.
\]
}
\part{Teoria}

\chapter{Corpos e Sistemas Lineares\quad$\left(06/01/2021\right)$}

\begin{definition}[Corpo]

	Um \index{corpo}\emph{corpo} é um conjunto não vazio $\mathbb{F}$
	munido de duas operações: adição mais e multiplicação.

	\[
		\begin{aligned}
			+\colon\mathbb{F}\times\mathbb{F} & \longrightarrow\mathbb{F} \\
			\left(x,y\right)                  & \longmapsto x+y
		\end{aligned}\qquad
		\begin{aligned}
			\cdot\colon\mathbb{F}\times\mathbb{F} & \longrightarrow\mathbb{F} \\
			\left(x,y\right)                      & \longmapsto x\cdot y
		\end{aligned}
	\]

	e tais que en $\left(\mathbb{F},+\right)$

	\begin{enumerate}[label={(A\arabic*)},leftmargin=0em,itemindent=*]
		\item\label{adição:1}

		      (Asociatividade na adição)
		      $\left(x+y\right)+z=x+\left(y+z\right)$,
		      $\forall x,y,z\in\mathbb{F}$;

		\item\label{adição:2}

		      (Existênza de neutro aditivo)
		      $\exists0\in\mathbb{F}$ tal que $x+0=0+x=x$,
		      $\forall x\in\mathbb{F}$;

		\item\label{adição:3}

		      (Existênza de elemento oposto o inverso aditivo)
		      Dado $x\in\mathbb{F}$, existe $-x\in\mathbb{F}$ tal que
		      $x+\left(-x\right)=\left(-x\right)+x=0$;

		\item\label{adição:4}

		      (Conmutatividade na adição)
		      $x+y=y+x$, $\forall x,y\in\mathbb{F}$;
	\end{enumerate}

	e $\left(\mathbb{F}\setminus\left\{0\right\},\cdot\right)$

	\begin{enumerate}[label={(M\arabic*)},leftmargin=0em,itemindent=*]
		\item\label{multiplicação:1}

		      (Associatividade na multiplicação)
		      $\left(x\cdot y\right)\cdot z=x\cdot\left(y\cdot z\right)$,
		      $\forall x,y,z\in\mathbb{F}$;


		\item\label{multiplicação:2}

		      (Existênza do elemento neutro na multiplicação)
		      $\exists 1\in\mathbb{F}$ tal que $x\cdot 1=1\cdot x=x$, $\forall x\in\mathbb{F}$;

		\item\label{multiplicação:3}

		      (Existênza inverso multiplicativo)
		      Dado $x\in\mathbb{F}\setminus\left\{0\right\}$,
		      existe $x^{-1}\in\mathbb{F}$ tal que
		      $x\cdot x^{-1}=x^{-1}\cdot x=1$;

		\item\label{multiplicação:4}

		      (Conmutatividade na multiplicação)
		      $x\cdot y=y\cdot x$, $\forall x,y\in\mathbb{F}$;
	\end{enumerate}

	\begin{enumerate}[label={(D)},leftmargin=0em,itemindent=*]
		\item\label{distributiva}

		      (Distributiva)
		      $x\cdot\left(y+z\right)=x\cdot y+x\cdot z$,
		      $\forall x,y,z\in\mathbb{F}$.
	\end{enumerate}
\end{definition}

\begin{proposition}
	$x\cdot0=0$, $\forall x\in\mathbb{F}$.
\end{proposition}

\begin{proof}
	\begin{math}
		x\cdot0=
		\overset{A2}{=}
		x\cdot\left(0+0\right)
		\overset{D}{=}
		x\cdot0+x\cdot0
	\end{math}.
	Assim
	\begin{align*}
		x\cdot0+
		\underbrace{x\cdot0+\left(-x\cdot0\right)}_{=0}
		        & =
		\underbrace{x\cdot0+\left(-x\cdot0\right)}_{=0} \\
		x\cdot0+
		0       & \overset{A3}{=}
		0                                               \\
		x\cdot0 & \overset{A2}{=}
		0.
	\end{align*}
\end{proof}

\begin{example}
	\begin{enumerate}[a)]\leavevmode
		\item

		      $\left(\mathbb{Z},+,\cdot\right)$ não é um corpo.
		      De fato não existe o inverso multiplicativo de $2$ em
		      $\mathbb{Z}$, ou seja, a equação $2\cdot x=1$ não se
		      resolue em $\mathbb{Z}$;

		\item

		      $\left(\mathbb{Q},+,\cdot\right)$ é um corpo, onde
		      \begin{math}
			      \mathbb{Q}=
			      \left\{
			      \frac{a}{b}\mid a,b\in\mathbb{Z},b\neq0
			      \right\}
		      \end{math}
		      e $\frac{a}{b}+\frac{c}{d}=\frac{ad+bc}{bd}$ e
		      $\frac{a}{b}\cdot\frac{c}{d}=\frac{ac}{bd}$.

		\item

		      $\left(\mathbb{R},+,\cdot\right)$ é um corpo (conjunto dos
		      números reais);

		\item

		      $\left(\mathbb{C},+,\cdot\right)$ é um corpo, onde
		      \begin{math}
			      \mathbb{C}=
			      \left\{
			      a+bi\mid a,b\in\mathbb{R},\text{ e }i^{2}=1
			      \right\}
		      \end{math},

		      \[
			      \begin{aligned}
				      +\colon\mathbb{C}\times\mathbb{C}                & \longrightarrow\mathbb{C} \\
				      \left(\left(a+bi\right),\left(c+di\right)\right) & \longmapsto
				      \left(a+c\right)+
				      \left(b+d\right)i
			      \end{aligned}\qquad
			      \begin{aligned}
				      \cdot\colon\mathbb{C}\times\mathbb{C}            & \longrightarrow\mathbb{C} \\
				      \left(\left(a+bi\right),\left(c+di\right)\right) & \longmapsto
				      \left(ac-bd\right)+
				      \left(ad+bc\right)i
			      \end{aligned}
		      \]

		      \begin{align*}
			      \left(a+bi\right)\left(c+di\right)
			       & =ac+adi+bci+bdi^{2}=                       \\
			       & =ac+\left(-1\right)bd+\left(ad+bc\right)i= \\
			       & =\left(ac-bd\right)+\left(ad+bc\right)i.
		      \end{align*}

		      $\mathbb{C}$ é chamado del conjunto nos números complexos.
		      Tome $a+bi\in\mathbb{C}\setminus\left\{0\right\}$
		      $(0=0+0i)$.

		      Assim

		      \begin{align*}
			      \left(a+bi\right)\left(a-bi\right)
			       & =a^{2}+b^{2}+\left(ab-ba\right)i= \\
			       & =a^{2}+b^{2} \neq 0               \\
			      \left(a+bi\right)
			      \underbrace{
				      \left(a-bi\right)
				      {\left(a^{2}+b^{2}\right)}^{-1}
			      }_{}
			       & =1.
		      \end{align*}

		      Logo
		      \[
			      {\left(a+bi\right)}^{-1}=
			      \frac{a}{a^{2}+b^{2}}-
			      \frac{b}{a^{2}+b^{2}}i.
		      \]

		\item

		      $\left(\mathbb{Z}/p\mathbb{Z},+,\cdot\right)$ é um corpo,
		      onde $p$ é primo e
		      \begin{math}
			      \mathbb{Z}/p\mathbb{Z}=
			      \left\{
			      \overline{a}\mid a\in\mathbb{Z}
			      \right\},
			      \overline{a}=
			      \left\{
			      a+pn\mid n\in\mathbb{Z}
			      \right\}
		      \end{math}
		      e $0\leq a\leq p-1$.

		      \[
			      \begin{aligned}
				      +\colon\mathbb{Z}/p\mathbb{Z}\times\mathbb{Z}/p\mathbb{Z} & \longrightarrow\mathbb{Z}/p\mathbb{Z} \\
				      \left(\overline{a},\overline{b}\right)                    & \longmapsto
				      \overline{a+b}
			      \end{aligned}\qquad
			      \begin{aligned}
				      \cdot\colon\mathbb{Z}/p\mathbb{Z}\times\mathbb{Z}/p\mathbb{Z} & \longrightarrow\mathbb{Z}/p\mathbb{Z} \\
				      \left(\overline{a},\overline{b}\right)                        & \longmapsto
				      \overline{a\cdot b}
			      \end{aligned}
		      \]
		      Tome $p=3$.
		      Assim
		      \begin{math}
			      \mathbb{Z}/3\mathbb{Z}=
			      \left\{
			      \overline{0},
			      \overline{1},
			      \overline{2}
			      \right\}
		      \end{math}.

		      \begin{tabular}{|>{$}c<{$}|>{$}c<{$}|>{$}c<{$}|>{$}c<{$}|}
			      \hline
			      +            & \overline{0} & \overline{1} & \overline{2} \\
			      \hline
			      \overline{0} & \overline{0} & \overline{1} & \overline{2} \\
			      \hline
			      \overline{1} & \overline{1} & \overline{2} & \overline{0} \\
			      \hline
			      \overline{2} & \overline{2} & \overline{0} & \overline{1} \\
			      \hline
		      \end{tabular}

		      \begin{tabular}{|>{$}c<{$}|>{$}c<{$}|>{$}c<{$}|}
			      \hline
			      \cdot        & \overline{1} & \overline{2} \\
			      \hline
			      \overline{1} & \overline{1} & \overline{2} \\
			      \hline
			      \overline{2} & \overline{2} & \overline{1} \\
			      \hline
		      \end{tabular}
		      $\overline{2}+\overline{2}=\overline{2+2}=\overline{4}=\overline{3\cdot1+1}=\overline{1}$.
	\end{enumerate}
\end{example}

Note que a equação $x^{2}+\overline{1}=\overline{0}$ não tem solução
em $\left(\mathbb{Z}/p\mathbb{Z},+,\cdot\right)$.

Defina:
\begin{math}
	F=
	\left\{
	\overline{a}+
	\overline{b}i\mid
	\overline{a},\overline{b}\in\mathbb{Z}/3\mathbb{Z}
	\text{ e }
	i^{2}=
	\overline{2}
	\right\}
\end{math}.

\[
	\begin{aligned}
		+\colon\mathbb{F}\times\mathbb{F} & \longrightarrow\mathbb{F} \\
		\left(
		\overline{a}+\overline{b}i,
		\overline{c}+\overline{d}i
		\right)                           & \longmapsto
		\left(\overline{a}+\overline{c}\right)+
		\left(\overline{b}+\overline{d}\right)i
	\end{aligned}\qquad
	\begin{aligned}
		\cdot\colon\mathbb{F}\times\mathbb{F} & \longrightarrow\mathbb{F} \\
		\left(
		\overline{a}+\overline{b}i,
		\overline{c}+\overline{d}i
		\right)                               & \longmapsto
		\left(
		\overline{a}\cdot\overline{c}+
		2\overline{b}\cdot\overline{d}
		\right)+
		\left(
		\overline{a}\cdot\overline{d}+
		\overline{b}\cdot\overline{c}
		\right)i
	\end{aligned}
\]
Mostre que $\left(\mathbb{F},+,\cdot\right)$ é um corpo com $9$ elementos.
%https://www.learn-portuguese-with-rafa.com/portuguese-keyboard-characters.html

\begin{definition}
	A característica de um corpo $\mathbb{F}$ é o menor inteiro positivo $n$ (se existir) tal que $\underbrace{1+\cdots+1}_{n}=0$.

	Se tal $n$ não existe, diremos que $F$ tem característica $0$.
\end{definition}

\begin{proposition}
	Seja $\mathbb{F}$ um corpo.
	Sea característica de $F$ é um inteiro positivo $n$, então $n$ é primo.
\end{proposition}

\begin{proof}
	Exercízio.
\end{proof}

\begin{example}
	\begin{enumerate}[a)]\leavevmode
		\item

		      Resolva em $\mathbb{Q}$ o sistema

		      \[
			      \begin{cases}
				      2 x+3 y & =1, \\
				      x+4 y   & =2.
			      \end{cases}
		      \]

		      % 2 x+3 y=1 \\
		      % 		-2 x-8 y=-4
		      % 	\Rightarrow
		      % 		2 x+3 y=1 \\
		      % 		-5 y=-3 \Rightarrow y=\frac{3}{5}
		      % 		2 x+3 \cdot \frac{3}{5}=1 \\
		      % 		2 x+\frac{9}{5}=1 \Rightarrow 2 x=-\frac{4}{5} \Rightarrow x=-\frac{2}{5}

		      \[
			      \begin{cases}
				      \overline{2}x+\overline{2}y= & \overline{1} \\
				      \overline{2}x+y=             & \overline{0}
			      \end{cases}
		      \]

		      \[
			      \begin{cases}
				      \overline{2}x+\overline{2}y & =\overline{1} \\
				      r_{y}                       & =\bar{I}
			      \end{cases}
		      \]
		\item Resolva em $\mathbb{Z}/3\mathbb{Z}$ o sistema
		      \[
			      \begin{aligned}
				      \overline{2} x+\overline{2} \cdot I=1 \Rightarrow & 2 x=\overline{1}-\overline{2} \\
				                                                        & \overline{2} x=-\overline{1}  \\
				                                                        & \overline{2} x=\overline{2}   \\
				                                                        & x=\overline{1}
			      \end{aligned}
		      \]
		      Daí $\left(\overline{1},\overline{1}\right)$ é solução do sistema.
	\end{enumerate}
\end{example}
\chapter{Sistemas lineares}

\begin{definition}[Sistema linear]
  Um corpo é.
\end{definition}
\chapter{Matrizes\quad$\left(08/01/2021\right)$}

Podemos denotar uma matriz $A$ sobre um corpo $\mathbb{F}$ de ordem
$m\times n$ por $A={\left(a_{ij}\right)}_{m\times n}$.

Sejam $A={\left(a_{ij}\right)}_{m\times n}$ e
$B={\left(b_{jl}\right)}_{n\times p}$ duas matrizes sobre um corpo
$\mathbb{F}$.
Definimos o producto de $A$ por $B$ como a matriz
$C={\left(c_{il}\right)}_{m\times p}$ dada por
\[
  c_{il}=
  \sum_{j=1}^{n}
  a_{i j}b_{jl}=
  a_{i1}b_{il}+\cdots+a_{in}b_{nl}
\]

\section*{Ilustração}

\[
  \begin{pmatrix}
    a_{11} & \cdots & a_{1n} \\
    \vdots & \vdots & \vdots \\
    a_{m1} & \cdots & a_{mn}
  \end{pmatrix}
  \begin{pmatrix}
    b_{11} & \cdots & b_{1p} \\
    \vdots & \vdots & \vdots \\
    b_{n1} & \cdots & b_{np}
  \end{pmatrix}=
  \begin{pmatrix}
    c_{11} & \cdots & c_{1p} \\
    \vdots & \vdots & \vdots \\
    c_{m1} & \cdots & c_{mp}
  \end{pmatrix}.
\]

\begin{example}
  Temos que
  \[
    {\begin{pmatrix}
          1 & 2  \\
          2 & -1
        \end{pmatrix}}_{2\times2}
      {\begin{pmatrix}
          1 & 0 & 4 \\
          3 & 2 & 5
        \end{pmatrix}}_{2\times3}=
    {\begin{pmatrix}
      7  & 4  & 14 \\
      -1 & -2 & 3
    \end{pmatrix}}_{2\times3}.
  \]
\end{example}

\begin{proposition}
  Sejam matrices $A={\left(a_{ij}\right)}_{m\times n}$,
  $B={\left(a_{jl}\right)}_{n\times p}$ e
  $C={\left(a_{lk}\right)}_{p\times q}$ matrizes sobre um corpo
  $\mathbb{F}$.
  Então $\left(AB\right)C=A\left(BC\right)$.
\end{proposition}

\begin{proof}
  Veja que
  \begin{math}
    \left(AB\right)C=
    {\left(\alpha_{ik}\right)}_{m\times q}
  \end{math},
  \begin{math}
    AB=
    {\left(d_{il}\right)}_{m\times p}
  \end{math}
  onde
  \begin{align*}
    d_{il}
     & =\sum_{l=1}^{n}
    a_{ij}b_{jl}
    \shortintertext{e}
    \alpha_{ik}
     & =\sum_{l=1}^{p}
    d_{il}c_{lk}=
    \sum_{l=1}^{p}
    \left(\sum_{j=1}^{n}a_{ij}b_{jl}\right)c_{lk}= \\
     & =\sum_{l=1}^{p}
    \left(\sum_{j=1}^{n}a_{ij}b_{jl}c_{lk}\right)= \\
     & =\sum_{j=1}^{n}
    a_{ij}\left(\sum_{l=1}^{p}b_{jl}c_{lk}\right)=\beta_{ik}
  \end{align*}
  com $A\left(BC\right)={\left(\beta_{ik}\right)}_{m\times q}$.
\end{proof}

Chamaremos a matriz quadrada
$I_{m}={\left(\delta_{ij}\right)}_{m\times m}$ definida por
\[
  \delta_{ij}=
  \begin{cases}
    1 & , \text{se }i=j,     \\
    0 & , \text{se }i\neq j,
  \end{cases}
\]
de matriz identidade de ordem $m\times m$.

Note que se $A={\left(a_{jl}\right)}_{m\times n}$, então $I_{m}A=A$, e se $B={\left(b_{li}\right)}_{n\times m}$, então $BI_{m}=B$.


$I_{m}A={\left(c_{il}\right)}_{m\times m}$ é tal que
$c_{il}=\sum_{j=1}^{m}\delta_{ij}a_{jl}=a_{il}$ con $1\leq i\leq m$,
e $I_{m}A=A$.

\begin{example}
  Se $m=3$, então
  \[
    I_{3}=
    \begin{pmatrix}
      1 & 0 & 0 \\
      0 & 1 & 0 \\
      0 & 0 & 1
    \end{pmatrix}.
  \]
\end{example}
Dizemos que uma matriz quadrada $A={\left(a_{ij}\right)}_{m\times m}$
tem inversa se existe uma matriz
$B={\left(b_{ij}\right)}_{m\times m}$ tal que $AB=BA=I_{m}$.

Denotaremos a matriz $B$ por $A^{-1}$.

\begin{definition}
  Seja $c\in\mathbb{F}\setminus\left\{0\right\}$.
  Uma matriz quadrada de ordem $m\times m$ $E$ é dita elementar se
  $E$ é de uma das formas
  \begin{enumerate}
    \item

          $E_{1}={\left(e_{ij}\right)}_{m\times m}$, onde
          \[
            e_{ij}=
            \begin{cases}
              \delta_{ij}, & \text{se }i\neq k \\
              \delta_{ij}, & \text{se }i=k
            \end{cases}\qquad
            \fbox{\begin{varwidth}{\textwidth}
                $m=3$, $k=2$\\[\baselineskip]
                \begin{math}
                  E_{1}=
                  \begin{pmatrix}
                    1 & 0 & 0 \\
                    0 & c & 0 \\
                    0 & 0 & 1
                  \end{pmatrix}
                \end{math}
              \end{varwidth}}
          \]
          con $k$ um inteiro fixo entre $1$ e $m$;
    \item

          $E_{2}={\left(e_{ij}\right)}_{m\times m}$, onde
          \[
            e_{ij}=
            \begin{cases}
              \delta_{ij}, & \text{se }i\neq l\text{ e }i\neq k \\
              \delta_{lj}, & \text{se }i=k                      \\
              \delta_{kj}, & \text{se }i=l
            \end{cases}\qquad
            \fbox{\begin{varwidth}{\textwidth}
                $m=3$, $k=2$, $l=3$\\[\baselineskip]
                \begin{math}
                  \begin{pmatrix}
                    1 & 0 & 0 \\
                    0 & 0 & 1 \\
                    0 & 1 & 0
                  \end{pmatrix}
                \end{math}
              \end{varwidth}}
          \]
          con $k<l$ inteiros fixos entre $1$ e $m$;
    \item

          $E_{3}={\left(e_{ij}\right)}_{m\times m}$, onde
          \[
            e_{ij}=
            \begin{cases}
              \delta_{i j},                  & i\neq k \\
              \delta_{kj}+c\cdot\delta_{lj}, & i=k
            \end{cases}\qquad
            \fbox{\begin{varwidth}{\textwidth}
                $m=3$, $k=2$, $l=3$\\[\baselineskip]
                \begin{math}
                  \begin{pmatrix}
                    1 & 0 & 0 \\
                    0 & 1 & c \\
                    0 & 0 & 1
                  \end{pmatrix}
                \end{math}
              \end{varwidth}}
          \]
  \end{enumerate}
\end{definition}

\begin{example}
  Calcule
  \[
    \begin{pmatrix}
      1 & 0 & 0 \\
      0 & 1 & 2 \\
      0 & 0 & 1
    \end{pmatrix}
    \begin{pmatrix}
      1 & 2 & 3  & -1 \\
      2 & 2 & 1  & 1  \\
      1 & 1 & -1 & 2  \\
    \end{pmatrix}=
    \begin{pmatrix}
      1 & 2 & 3  & -1
      4 & 4 & -1 & 5
      1 & 1 & -1 & 2
    \end{pmatrix}
  \]
\end{example}
Dada uma matriz $A={\left(a_{ij}\right)}_{m\times m}$ o efeito de multiplicar uma matriz elementar $E$ por $A$ pode ser colocado como:
\begin{enumerate}
  \item

        $E_{1}A$: multiplica uma linha $k$ de $A$ por um escalar $c$;

  \item

        $E_{2}A$: troca duas linhas $l$ e $k$ de posições
        $\left(k<l\right)$;

  \item

        $E_{3}A$: soma uma linha $k$ com outra linha $l$ multiplicada por um escalar $c\in\mathbb{F}$.
\end{enumerate}

% \[
%   E_{1}=
%   \left(e_{ij}\right)_{m\times m}\equiv
%   A=\left(a_{jl}\right)_{m\times n} \\
%   E_{1}A=
%   \left(c_{il}\right)_{m\times n}\text{com}
%   c_{il}=
%   \sum_{j=1}^{n}e_{ij}a_{je}=
%   a_{il}, i\neq k,
%   ca_{il},i=k
%   \underbrace{
%   a_{11} & \cdots & a_{14} \\
%   a_{m1} & \vdots & a_{mn}
%     _{A}
%   \underbrace{
%     x_{2} \\
%     \vdots \\
%     x_{n}
%   }_{X}=
%   \underbrace{
%     \gamma_{1} \\
%     \vdots \\
%     y_{m}
%   }_{y}
%   a_{1} x_{2}+\cdots+a_{n} x_{n}=y_{2} \\
%   \vdots \\
%   a_{m 1} x_{2}+\cdots+a_{w_{x}} x_{n}=y_{m}
% \]
\chapter{Aula de reposição\quad$\left(09/01/2021\right)$}

\begin{definition}[Matriz reducida por linhas]
  Uma matriz $A={\left(a_{ij}\right)}_{m\times n}$ sobre $\mathbb{F}$
  é deja reduzida por linhas se
  \begin{enumerate}
    \item

          O primeiro elemento não nulo de cada linha não nula é
          igual $1$;

    \item

          Cada columna que possui o primeiro elemento não nulo de

    \item

          uma linha não possui todos os outros elementos iguais a
          $0$;
  \end{enumerate}


  Sea além disso, esa matriz $A$ satisfaz

  \begin{enumerate}

    \item

          Todas linhas nulas ocorrem abaixo das linhas não nulas;

    \item

          Se $1,\ldots,r$ $\left(r\leq m\right)$ são as linhas não
          nulas de $A$ com os primeiros elementos não nunos ocurrendo
          nas colunas $k_{1},k_{2},k_{r}$, respectivamente, então
          $k_{1}<k_{2}<\cdots<k_{r}$, dizemos que $A$ está na forma
          escada reduzida.
  \end{enumerate}
\end{definition}

\begin{example}
  \begin{enumerate}
    \item As seguintes matrizes estão na forma reduzida:
          \begin{enumerate}
            \item
                  \begin{math}
                    \begin{pmatrix}
                      0 & 1 & 2 & 3 \\
                      1 & 0 & 1 & 1
                    \end{pmatrix},
                  \end{math}
            \item

                  \begin{math}
                    \begin{pmatrix}
                      0 & 0 & 0 \\
                      1 & 0 & 1 \\
                      0 & 1 & 0
                    \end{pmatrix},
                  \end{math}

            \item

                  \begin{math}
                    \begin{pmatrix}
                      0 & 0 & 0 \\
                      0 & 0 & 0 \\
                      0 & 0 & 0
                    \end{pmatrix},
                  \end{math}

            \item

                  \begin{math}
                    \begin{pmatrix}
                      1 & 0 & 0 & 0 \\
                      0 & 1 & 1 & 2 \\
                      0 & 0 & 0 & 0
                    \end{pmatrix},
                  \end{math}
          \end{enumerate}
    \item As seguientes matrizes estão na forma escada reducida
          \begin{enumerate}
            \item

                  \begin{math}
                    \begin{pmatrix}
                      1 & 0 & 2 & 0 \\
                      0 & 1 & 3 & 0 \\
                      0 & 0 & 0 & 1
                    \end{pmatrix}
                  \end{math}

            \item

                  \begin{math}
                    \begin{pmatrix}
                      1 & 0 \\
                      0 & 1 \\
                      0 & 0
                    \end{pmatrix}
                  \end{math}


            \item

                  \begin{math}
                    \begin{pmatrix}
                      1 & 2 & 3 & 0 & 1 \\
                      0 & 0 & 0 & 1 & 0
                    \end{pmatrix}
                  \end{math}
          \end{enumerate}

  \end{enumerate}
\end{example}

\begin{remark}
  
\end{remark}
\chapter{Espaços vetoriais\quad$\left(12/01/2021\right)$}

\begin{definition}
  Um conjunto não vazio $V$ é chamado de espaço vetorial sobre um corpo $\mathbb{F}$ se em $V$ estão definidas duas operações

  \[
    \begin{aligned}
      +\colon V\times V & \longrightarrow V \\
      \left(u,v\right)  & \longmapsto u+v
    \end{aligned}\qquad
    \begin{aligned}
      \cdot\colon\mathbb{F}\times V & \longrightarrow V    \\
      \left(c,v\right)              & \longmapsto c\cdot v
    \end{aligned}
  \]

\end{definition}

% (A1) $u+\left(v+w\right)=\left(u+v\right)+w,\forall u,v,\omega\in V$.
% (A2) Exr un $O\in V$ taa ave avk $v+\left(-v\right)=\left(-v\right)+v=0$;
% (A4) $u_{+}v=v+u,\forall u,v\in V_{i}$
% (E1) $I.v=v,\forall v\in V$, owor $1\in\mathbb{F}$;
% (E2) $\left(ck\right)v=c\left(kv\right), \forall v\in V,E\forall c, k \in \mathbb{F}$
% $(\pm 3) c\left(u+v\right)=cu+cv,\forall u,v\in V,\forall c\in\mathbb{F}$
% (E4) $\left(c+k\right)v=cv+kv,\quad\forall v\in V_{E}c,k\in\mathbb{F}$.

% \mathbb{F}^{n}=\left\{\left(x_{2},\dotsc,x_{n}\right)\mid x_{i}\in \mathbb{F},i=1,\dotsc,n\right\},

% \mathbb{F}^{m\times n}=
% \mathscr{N}_{N\times n}\left(\mathbb{F}\right)=
% \left\{A=\left(a_{ij}\right)_{w+1}\mid a_{ij}\in\mathbb{F}\right\}


% +\colon A+B=
% \left(a_{ij}\right)_{mx+1}+
% \left(b_{i j}\right)_{wxn}=
% \left(a_{ij}+b_{ij}\right)_{w\times n}\\
% \therefore cA=c\cdot\left(a_{ij}\right)_{mx_{n}}=
% \left(c\cdot a_{ij}\right)_{m\times n}


% i) $0\in W$;
% iii) $c\omega\in\omega,\forall c\in\mathbb{F}_{r}w\in\omega$

% i) W $\pi$ wat vazro
% ii) $w_{1}+cw_{2}\in W,\forall c\in\mathbb{F}_{k}\forall\omega_{1},w_{2}\in W$.

% A\left(X_{0}+c X_{1}\right)=A X_{0}+c A X_{1}

% c) W=
% \left\{A=\left(a_{ij}\right)_{3\times 2}\mid a_{n}=0\right\}

% \begin{aligned}
%    & \text { d) } \omega=\{f\colon\mathbb{R}\to\mathbb{R}\mid f(1)=0\} \text { i } \\
%    & \mathbb{F}=\{f: \mathbb{R} \rightarrow \mathbb{R}) \text { f is runcto }\}
% \end{aligned}

% +:\left(f+g\right)\left(x\right)=f\left(x\right)+g\left(x\right),\quad\therefore\left(cf\right)\left(x\right)=cf\left(x\right)


% i) \cap \omega_{i} \text { ; } \\
% ii)\sum_{i\in 1}\omega_{i}=\left\{w_{i}+a_{i}+\dotsb+w_{i_{n}}\mid \omega_{i}\in W_{j}\in n\in\mathbb{Z}_{>1}\right\}


% x+c v & =\left(u_{i_{1}}+\dotsb+u_{i_{e}}\right)+c\left(v_{j_{1}}+\dotsb+v_{j k}\right)=          \\
% & =u_{i_{2}}+\dotsb+u_{i_{e}}+c v_{j_{2}}+\dotsb+c_{v_{k}} \in \sum_{i \in \tau} \omega_{i}

% \bigcap_{i\in 2}\omega_{i}\subseteq U_{i\in I}\omega_{i}\subseteq \sum_{i\in Z}\omega_{i}

% \left(x,y,z\right)=x\left(1,0,0\right)+y\left(0,1,0\right)+z\left(0,0,1\right)

% \left(x,y,z\right)=-x i\left(i,1,0\right)+\left(y+xi-z\right)\left(0,1,0\right)-zi\left(0,i,i\right)

% W=\left\{c_{2}v_{2}+\dotsb+c_{n}v_{n}\mid v_{i}\in S_{\varepsilon}c_{i}\in\mathbb{F}\right\}

% \begin{aligned}
%   u+cv & =\left(c_{2}u_{2}+\dotsb+c_{n}u_{n}\right)+c\left(k_{1}v_{2}+\dotsb+k_{m}v_{m}\right)=       \\
%        & =c_{1}u_{2}+\dotsb+c_{n}u_{n}+\left(ck_{1}\right)v_{2}+\dotsb+\left(ck_{m}\right)v_{m}\in W.
% \end{aligned}

% \left\langle S\right\rangle=\bigcap_{S\leq w}w

% \left\langle S\right\rangle=
% \left\{c_{2}v_{2}+\dotsb+c_{n}v_{n}\mid v_{i}\in S_{\varepsilon}c_{2} \in\mathbb{F}\right\}=W

% \omega=c_{2}v_{2}+\dotsb+c_{n}v_{n}\in\cap u

% \mathbb{C}^{3}=
% \left\langle\left(i,1,0\right),\left(0,1,0\right),\left(0,i,i\right)\right\rangle

% x+y+z=0                                        \\
% 2x+y+2z=0                                      \\
% \mathbb{L}<z\to\alpha_{2}-2\alpha_{L} \\
% x+y+z=0\to x=-z                         \\
% -y=0\to y=0

% DS Szs7rum DADO. No7k aun
% \[
%   W=
%   \left\{
%   z\left(-1,0,1\right)\mid
%   z\in\mathbb{R}
%   \right\}= \\
%   =\left\langle\left(-1,0,1\right)\right\rangle
% \]

% \[
%   W=
%   \left\{
%   \left(z+y,y,z\right)\mid
%   z,y\in\mathbb{R}
%   \right\}=       \\
%   =\left\{
%   y\left(1,1,0\right)+
%   z\left(1,0,1\right)\mid
%   y,z\in\mathbb{R}\right\}= \\
%   =
%   \left\langle
%   \left(1,1,0\right),
%   \left(1,0,1\right)
%   \right\rangle
% \]

% \[
%   x+y+z=1 \\
%   2 x+y+2 z=2
%   \quad \implies
%   x+y+z =1 \to x=1-z \\
%   -y    =0 \to y=0
% \]


% $S=
%   \left\{
%   \left(1-z,0,z\right)\mid
%   z\in\mathbb{R}
%   \right\}=
%   \left\{
%   \left(1,0,0\right)+
%   z\left(-1,0,1\right)\mid
%   z\in\mathbb{R}
%   \right\}
% $

\setpartpreamble{%
  \begin{center}
    \includesvg[width=.4\paperwidth]{The_four_subspaces-X-1}
  \end{center}
}
\part{Prática}

\chapter{Exercícios de Fixação\quad$\left(08/01/2021\right)$}

\begin{questions}
  \question\label{exercício:1.1}

  Seja $\mathbb{F}$ um corpo.
  Dizemos que um subconjunto $\mathbb{K}$ de $\mathbb{F}$ é um
  subcorpo de $\mathbb{F}$ se $\mathbb{K}$ munido das operações de
  adição e multiplicação de $\mathbb{F}$ é um corpo.
  Mostre que os seguintes subconjuntos são subcorpos de $\mathbb{C}$.

  \begin{parts}
    \begin{multicols}{3}
      \part\label{exercício:1.1a}

      \begin{math}
        \mathbb{Q}
        \left(\sqrt{3}\right)=
        \left\{
        a+b\sqrt{3}\mid a,b\in\mathbb{Q}
        \right\}
      \end{math};

      \part\label{exercício:1.1b}

      \begin{math}
        \mathbb{Q}
        \left(i\right)=
        \left\{
        a+bi\mid a,b\in\mathbb{Q}
        \text{ e }i^{2}=-1
        \right\}
      \end{math};

      \part\label{exercício:1.1c}

      \begin{math}
        \mathbb{Q}
        \left(i\sqrt{2}\right)=
        \left\{
        a+bi\sqrt{2}\mid a,b\in\mathbb{Q}
        \text{ e }i^{2}=-1
        \right\}
      \end{math}.
    \end{multicols}
  \end{parts}

  \begin{solutionordottedlines}
    \begin{parts}
      \part
      .

      \part

      .

      \part

      .
    \end{parts}
  \end{solutionordottedlines}

  \question\label{exercício:1.2}

  Mostre que:

  \begin{parts}
    \part\label{exercício:1.2a}

    Todo subcorpo de $\mathbb{C}$ tem $\mathbb{Q}$ como subcorpo;

    \part\label{exercício:1.2b}

    Todo corpo de característica $0$ tem uma cópia de $\mathbb{Q}$;

    \part\label{exercício:1.2c}

    Se $\mathbb{K}$ contém propriamente $\mathbb{R}$ e é um subcorpo de
    $\mathbb{C}$, então $\mathbb{K}=\mathbb{C}$.

  \end{parts}

  \begin{solutionordottedlines}
    \begin{parts}
      \part
      .

      \part

      .

      \part

      .
    \end{parts}
  \end{solutionordottedlines}

  \question\label{exercício:1.3}

  Considere o corpo finito com $5$ elementos
  \begin{math}
    \mathbb{Z}/5\mathbb{Z}=
    \left\{
    \overline{0},
    \overline{1},
    \overline{2},
    \overline{3},
    \overline{4}
    \right\}
  \end{math}.

  \begin{parts}
    \part\label{exercício:1.3a}

    Mostre que
    \[
      \mathbb{F}=
      \left\{
      a+bi\mid a,b\in\mathbb{Z}/5\mathbb{Z}
      \text { e }
      i^{2}=
      \overline{3}
      \right\}
    \]

    munido das operações

    \[
      \begin{aligned}
        \overline{+}\colon\mathbb{F}\times\mathbb{F}     & \longrightarrow\mathbb{F}     \\
        \left(\left(a+bi\right),\left(c+di\right)\right) & \longmapsto \left(a+c\right)+
        \left(b+d\right)i
      \end{aligned}\quad
      \begin{aligned}
        \overline{\cdot}\colon\mathbb{F}\times\mathbb{F}  & \longrightarrow\mathbb{F}                   \\
        \left(\left(a+b i\right),\left(c+di\right)\right) & \longmapsto \left(ac+\overline{3}bd\right)+
        \left(ad+bc\right)i
      \end{aligned}
    \]

    é um corpo com 25 elementos;

    \part\label{exercício:1.3b}

    Mostre que $\mathbb{Z}/5\mathbb{Z}$ é um subcorpo de $\mathbb{F}$.
    Qual é a característica de $F$?
  \end{parts}

  \begin{solutionordottedlines}
    \begin{parts}
      \part

      .

      \part

      .
    \end{parts}
  \end{solutionordottedlines}

  \question\label{exercício:1.4}

  Determine o conjunto solução de cada sistema linear dado.

  \begin{parts}%\setlength{\columnsep}{-20pt}
    \begin{multicols}{2}
      \part\label{exercício:1.4a}

      \begin{math}
        \csysteme[xyzw]{
          x-2y+z+w=1,
          2x+y-z=3,
          2x+y-5z+w=4
        }
      \end{math}
      em $\mathbb{R}$,

      \part\label{exercício:1.4b}

      \begin{math}
        \csysteme[xyzw]{
          x-\sqrt{3}y+z+w=1+\sqrt{3},
          \left(2\+\sqrt{3}\right)x+y-z=3,
          2x+y-\left(1\-\sqrt{3}\right)z+w=4
        }
      \end{math}
      em $\mathbb{Q}\left(\sqrt{3}\right)$,

      \part\label{exercício:1.4c}

      \begin{math}
        \csysteme[xyzw]{
          x-2iy+2z-w=0,
          \left(2\+i\right)x+z+w=0,
          2ix+y-5z+\left(1\+i\right)w=0
        }
      \end{math}
      em $\mathbb{C}$,

      \part\label{exercício:1.4d}

      \begin{math}
        \csysteme[xyzw]{
          x-\overline{2}y+\overline{2}z-w=\overline{0},
          \overline{2}x+z+w=\overline{0},
          \overline{2}x+y-\overline{3}z+w=\overline{0}
        }
      \end{math}
      em $\mathbb{Z}/5\mathbb{Z}$,

      \part\label{exercício:1.4e}

      \begin{math}
        \csysteme[xyzw]{
          x-\overline{2}iy+\overline{2}z-w=\overline{0},
          \left(\overline{2}\+i\right)x+z+w=\overline{0},
          \overline{2}ix+y-\overline{3}z+\left(\overline{1}\+i\right)w=\overline{0}
        }
      \end{math}
      em $\mathbb{F}$ de (a) da questão 3.
    \end{multicols}
  \end{parts}%\setlength{\columnsep}{10pt}

  \begin{solutionordottedlines}
    \begin{parts}
      \part

      .

      \part

      .

      \part

      .

      \part

      .

      \part

      .
    \end{parts}
  \end{solutionordottedlines}

  \question\label{exercício:1.5}

  Mostre que se dois sistemas lineares $2\times2$ possuem o mesmo
  conjunto solução, então eles são equivalentes.
  Determine, se existir, dois sistemas lineares $2\times3$ com mesmo
  conjunto solução mas não equivalentes.

  \begin{solutionordottedlines}
  \end{solutionordottedlines}

  \question\label{exercício:1.6}

  Considere o sistema linear sobre $\mathbb{Q}$

  \[
    \csysteme[xyzw]{
      x-2y+z+2w=1,
      x+y-z+w=2,
      x+7y-5z-w=3
    }
  \]

  Mostre que esse sistema não tem solução.

  \begin{solutionordottedlines}
  \end{solutionordottedlines}

  \question\label{exercício:1.7}

  Determine todos $a,b,c,d\in\mathbb{R}$ tais que o sistema linear

  \[
    \begin{bNiceMatrix}
      3  & -6 & 2 & 1 \\
      -2 & 4  & 1 & 3 \\
      0  & 0  & 1 & 1 \\
      1  & -2 & 1 & 0
    \end{bNiceMatrix}
    \cdot
    \begin{bNiceMatrix}
      x \\
      y \\
      z \\
      w
    \end{bNiceMatrix}=
    \begin{bNiceMatrix}
      a \\
      b \\
      c \\
      d
    \end{bNiceMatrix}
  \]
  tem solução.

  \begin{solutionordottedlines}
  \end{solutionordottedlines}

  \question\label{exercício:1.8}

  Encontre duas matrizes $A$ e $B$ de ordens iguais a $3\times3$ tais
  que $AB$ é uma matriz nula mas $BA$ não é.

  \begin{solutionordottedlines}
  \end{solutionordottedlines}

  \question\label{exercício:1.9}

  Mostre que toda matriz elementar é inversível e calcule a inversa
  de cada tipo.

  \begin{solutionordottedlines}
  \end{solutionordottedlines}

  \question\label{exercício:1.10}

  Determine a matriz inversa da matriz

  \[
    A=
    \begin{bNiceMatrix}
      1 & 2 & 3 & 4 \\
      0 & 2 & 3 & 4 \\
      0 & 0 & 3 & 4 \\
      0 & 0 & 0 & 4
    \end{bNiceMatrix}.
  \]

  \begin{solutionordottedlines}
  \end{solutionordottedlines}


  \question\label{exercício:11.1}

  Considere a matriz

  \NiceMatrixOptions{
    cell-space-top-limit = 3pt,
    %cell-space-bottom-limit = 2pt
  }

  \[
    A=
    \begin{bNiceMatrix}
      \overline{1} & \overline{2} & \overline{3} & \overline{4} \\
      \overline{0} & \overline{2} & \overline{3} & \overline{4} \\
      \overline{0} & \overline{0} & \overline{3} & \overline{4} \\
      \overline{0} & \overline{0} & \overline{0} & \overline{4}
    \end{bNiceMatrix}
  \]

  \NiceMatrixOptions{
    cell-space-top-limit = 0pt,
  }

  com entradas no corpo com cinco elementos
  \begin{math}
    \mathbb{Z}/5\mathbb{Z}=
    \left\{
    \overline{0},
    \overline{1},
    \overline{2},
    \overline{3},
    \overline{4}
    \right\}
  \end{math}.
  Calcule sua inversa.

  \begin{solutionordottedlines}
  \end{solutionordottedlines}

  \question\label{exercício:1.12}

  Considere a matriz
  \[
    A=
    \begin{bNiceMatrix}
      1  & 2  & 1 & 0 \\
      -1 & 0  & 3 & 5 \\
      1  & -2 & 1 & 1
    \end{bNiceMatrix}.
  \]
  Encontre uma matriz na forma e uma matriz invertível $P$ tal que
  $R=PA$.

  \begin{solutionordottedlines}
  \end{solutionordottedlines}
\end{questions}

\chapter{Exercícios de Fixação\quad$\left(15/01/2021\right)$}

\begin{questions}
  \question\label{exercício:2.1}

  Defina sobre $\mathbb{R}^{2}$ as seguintes operações:

  \[
    \begin{aligned}
      +\colon\mathbb{R}^{2}\times\mathbb{R}^{2}      & \longrightarrow\mathbb{R}^{2} \\
      \left(\left(x,y\right),\left(a,b\right)\right) & \longmapsto
      \left(x+a,0\right)
    \end{aligned}\qquad
    \begin{aligned}
      \cdot\colon\mathbb{R}\times\mathbb{R}^{2} & \longrightarrow\mathbb{R}^{2} \\
      \left(c,\left(x,y\right)\right)           & \longmapsto
      \left(cx,0\right)
    \end{aligned}
  \]

  O conjunto $\mathbb{R}^{2}$ é um espaço vetorial com essas operações?

  \begin{solutionordottedlines}
  \end{solutionordottedlines}

  \question\label{exercício:2.2}

  Defina sobre
  \begin{math}
    V=
    \left\{
    \left(x,y\right)\in
    \mathbb{R}^{2}\mid
    x>0,y>0
    \right\}
  \end{math}
  as seguintes operações:

  \[
    \begin{aligned}
      +\colon\mathbb{R}^{2}\times\mathbb{R}^{2}      & \longrightarrow\mathbb{R}^{2} \\
      \left(\left(x,y\right),\left(a,b\right)\right) & \longmapsto
      \left(xa,yb\right)
    \end{aligned}\qquad
    \begin{aligned}
      \cdot\colon\mathbb{R}\times\mathbb{R}^{2} & \longrightarrow\mathbb{R}^{2} \\
      \left(c,\left(x,y\right)\right)           & \longmapsto
      \left(x^{c},y^{c}\right)
    \end{aligned}
  \]
  Mostre que $V$ é um espaço vetorial com essas operações.

  \begin{solutionordottedlines}
  \end{solutionordottedlines}

  \question\label{exercício:2.3}

  Resolva:
  \begin{parts}
    \part\label{exercício:2.3a}

    O vetor $\left(3,-1,0,-1\right)$ pertence ao subespaço
    \begin{math}
      W=
      \left\langle
      \left(2,-1,3,2\right),
      \left(-1,1,1,3\right),
      \left(1,1,9,-5\right)
      \right\rangle
    \end{math}
    de $\mathbb{R}^{4}$?

    \part\label{exercício:2.3b}

    Determine uma base para o subespaço vertorial de $\mathbb{R}^{5}$
    das soluções do sistema linear homogêneo
    $\csysteme[xyzwt]{x-2y+z+w+t=0,3x+y-z-4t=0,2x+y-3z+w=0}$;

    \part\label{exercício:2.3c}

    Determine uma base para o subespaço vertorial de
    ${\left(\mathbb{Z}/5\mathbb{Z}\right)}^{5}$ das soluções do
    sistema linear homogêneo
    \begin{math}
      \csysteme[xyzwt]{
        x-\overline{2}y+z+w+t=\overline{0},
        \overline{2}x+y-z+t=\overline{0},
        \overline{3}x+y+\overline{3}z+w=\overline{0}
      }
    \end{math},
  \end{parts}

  \begin{solutionordottedlines}
    \begin{parts}
      \part

      .

      \part

      .

      \part

      .
    \end{parts}
  \end{solutionordottedlines}

  \question\label{exercício:2.4}

  Sejam $V$ um espaço vetorial sobre um corpo $F$ e $U$ e $W$
  subespaços de $V$ tais que $U+W=V$ e $U \cap W=\left\{0\right\}$.
  Mostre que cada vetor $v\in V$ é escrito de maneira única como
  $v=u+w$, onde $u\in U$ e $w\in W$.

  \begin{solutionordottedlines}
  \end{solutionordottedlines}

  \question\label{exercício:2.5}

  Mostre que o conjunto dos polinômios sobre uma variável com
  coeficientes em $\mathbb{R}$ é um espaço vetorial sobre
  $\mathbb{R}$ munido das operações usuais de soma e multiplicação
  por escalar.
  Determine uma base para esse espaço vetorial.

  \begin{solutionordottedlines}
  \end{solutionordottedlines}

  \question\label{exercício:2.6}

  Seja $S$ um subconjunto de um espaço vetorial $V$.
  Mostre que $S$ é LD se, e somente se, existir um vetor $v\in S$ que
  pode ser escrito como combinação linear dos elementos de
  $S\setminus\left\{v\right\}$.

  \begin{solutionordottedlines}
  \end{solutionordottedlines}

  \question\label{exercício:2.7}

  Considere o seguinte espaço vetorial sobre $\mathbb{R}$:

  \[
    \mathcal{P}_{3}
    \left(\mathbb{R}\right)=
    \left\{
    a+bx+cx^{2}+dx^{3}\mid
    a,b,c,d\in\mathbb{R}
    \right\}.
  \]

  \begin{parts}
    \part\label{exercício:2.7a}

    Mostre que $\alpha=\left\{1,2+x,3x-x^{2},x-x^{3}\right\}$ é uma
    base de $\mathcal{P}_{3}\left(\mathbb{R}\right)$;

    \part\label{exercício:2.7b}

    Escreva as coordenadas de $p\left(x\right)=1+x+x^{2}+x^{3}$ com
    relação a base $\alpha$;

    \part\label{exercício:2.7c}

    Determine as matrizes mudança de base
    ${\left[I\right]}_{\alpha}^{e}$ e
    ${\left[I\right]}_{e}^{\alpha}$, onde
    $e=\left\{1,x,x^{2},x^{3}\right\}$.
  \end{parts}

  \begin{solutionordottedlines}
    \begin{parts}
      \part

      .

      \part

      .

      \part

      .
    \end{parts}
  \end{solutionordottedlines}

  \question\label{exercício:2.8}

  Faça o que se pede:

  \begin{parts}
    \part\label{exercício:2.8a}

    Considere a função
    $T\colon\mathbb{C}\to\mathcal{M}_{2}\left(\mathbb{R}\right)$
    dada por

    \[
      T\left(x+yi\right)=
      \begin{bmatrix*}
        x+7y & 5y \\
        -10y & x-7y
      \end{bmatrix*}.
    \]

    Moste que $T$ é uma transformação linear.
    Prove que
    \begin{math}
      T\left(z_{1}z_{2}\right)=
      T\left(z_{1}\right)
      T\left(z_{2}\right),
      \forall z_{1},z_{2}\in\mathbb{C}
    \end{math};

    \part\label{exercício:2.8b}

    Mostre que a composta de transformações lineares é uma transformação linear;

    \part\label{exercício:2.8c}

    Mostre que uma função $T\colon\mathbb{F}^{n}\to\mathbb{F}$ é uma
    transformação linear se, e somente se, existem escalares
    $c_{1},\dotsc,c_{n}$ no corpo $\mathbb{F}$ tais que

    \[
      T
      \left(x_{1},\dotsc,x_{n}\right)=
      c_{1}x_{1}+\dotsc+c_{n}x_{n}.
    \]
  \end{parts}

  \begin{solutionordottedlines}
    \begin{parts}
      \part

      .

      \part

      .

      \part

      .
    \end{parts}
  \end{solutionordottedlines}

  \question\label{exercício:2.9}

  Faça o que se pede:

  \begin{parts}
    \part\label{exercício:2.9a}

    Considere $\mathbb{R}^{4}$ e seus subespaços
    \begin{math}
      W=
      \left\langle
      \left(1,0,1,1\right),
      \left(0,-1,-1,-1\right)
      \right\rangle
    \end{math}
    e
    \begin{math}
      U=
      \left\{
      \left(x,y,z,w\right)\in
      \mathbb{R}^{4}\mid
      x+y=0,
      z+t=0
      \right\}
    \end{math}.
    Determine uma transformação linear
    $T\colon\mathbb{R}^{4}\to\mathbb{R}^{4}$ tal que
    $\operatorname{Nuc}\left(T\right)=U$ e
    $\operatorname{Im}\left(T\right)=W$;

    \part\label{exercício:2.9b}

    Considere ${\left(\mathbb{Z}/5\mathbb{Z}\right)}^{4}$ e seus
    subespaços
    \begin{math}
      W=
      \left\langle
      \left(
      \overline{1},
      \overline{0},
      \overline{1},
      \overline{1}
      \right),
      \left(
      \overline{0},
      \overline{4},
      \overline{4},
      \overline{4}
      \right)
      \right\rangle
    \end{math}
    e
    \begin{math}
      U=
      \left\{
      \left(x,y,z,w\right)\in
      {\left(\mathbb{Z}/5\mathbb{Z}\right)}^{4}\mid
      x+y=
      \overline{0},
      z+t=
      \overline{0}
      \right\}
    \end{math}.
    Determine uma transformação linear
    $T\colon\mathbb{R}^{4}\to\mathbb{R}^{4}$ tal que
    $\operatorname{Nuc}\left(T\right)=V$ e
    $\operatorname{Im}\left(T\right)=W$;

    \part\label{exercício:2.9c}

    Determine uma base para o núcleo e uma base para a imagem da
    transformação linear $T\colon\mathbb{C}^{2}\to\mathbb{R}^{2}$
    dada por $T\left(x+yi,a+bi\right)=\left(x+2a,-x+2b\right)$.
  \end{parts}

  \begin{solutionordottedlines}
    \begin{parts}
      \part

      .

      \part

      .
      \part

      .
    \end{parts}
  \end{solutionordottedlines}

  \question\label{exercício:2.10}

  Sejam $V$ e $W$ espaços vetoriais sobre um mesmo corpo $\mathbb{F}$
  e $T\colon V\to W$.
  Mostre que $T$ é injetora se, e somente se, $T$ leva subconjunto LI
  em subconjunto LI.

  \begin{solutionordottedlines}
  \end{solutionordottedlines}

  \question\label{exercício:2.11}

  Seja
  $T\colon\mathbb{C}^{3}\to\mathcal{P}_{2}\left(\mathbb{C}\right)$
  a transformação linear definida por
  $T\left(1,0,0\right)=1+ix^{2}$,
  $T\left(0,1,0\right)=x+x^{2}$ e
  $T\left(0,0,1\right)=i+x$.
  Exiba uma fórmula para $T$ e decida se $T$ é um isomorfismo.

  \begin{solutionordottedlines}
  \end{solutionordottedlines}

  \question\label{exercício:2.12}

  Seja $F$ um corpo e $T\colon\mathbb{F}^{2}\to\mathbb{F}^{2}$ dada
  por
  \begin{math}
    T\left(x,y\right)=
    \left(x+y,x\right),
    \forall\left(x,y\right)\in
    \mathbb{F}^{2}
  \end{math}.
  Mostre que $T$ é um isomorfismo e exiba uma fómula para $T^{-1}$.

  \begin{solutionordottedlines}
  \end{solutionordottedlines}

  \question\label{exercício:2.13}

  Considere as bases $\alpha=\left\{1,1+x,1+x^{2}\right\}$ de
  $\mathcal{P}_{2}\left(\mathbb{R}\right)$ e
  \begin{math}
    \beta=
    \left\{
    \left(1,0\right),
    \left(i,0\right),
    \left(1,1\right),
    \left(1,i\right)
    \right\}
  \end{math}
  de $\mathbb{C}^{2}$ como espaços vetoriais sobre $\mathbb{R}$.
  Determine as coordenadas da transformação linear
  $T\colon\mathcal{P}_{2}\left(\mathbb{R}\right)\to\mathbb{C}^{2}$ dada
  por $T\left(a+bx+cx^{2}\right)=\left(a+bi,b+ci\right)$ com relação
  à base de
  \begin{math}
    L
    \left(
    \mathcal{P}_{2}\left(\mathbb{R}\right),
    \mathbb{C}^{2}
    \right)
  \end{math}
  construída no Teorema 2-(ii) da Aula $9$.

  \begin{solutionordottedlines}
  \end{solutionordottedlines}

  \question\label{exercício:2.14}

  Considere a base
  \begin{math}
    \alpha=
    \left\{
    \left(1,0,-1\right),
    \left(1,1,1\right),
    \left(2,2,0\right)
    \right\}
  \end{math}
  de $\mathbb{C}^{3}$ como espaço vetorial sobre $\mathbb{C}$.
  Determine a base dual $\alpha^{\ast}$ de
  ${\left(\mathbb{C}^{3}\right)}^{\ast}$.

  \begin{solutionordottedlines}
  \end{solutionordottedlines}

  \question\label{exercício:2.15}

  Considere $T\colon\mathbb{R}^{2}\to\mathbb{R}^{3}$ dada por
  $T\left(x,y\right)=\left(2x+3y,y-x,3x\right)$ e as bases
  $\alpha=\left\{\left(1,2\right),\left(2,-1\right)\right\}$ de
  $\mathbb{R}^{2}$ e
  \begin{math}
    \beta=
    \left\{
    \left(1,1,1\right),
    \left(0,1,1\right),
    \left(0,0,1\right)
    \right\}
  \end{math}
  de $\mathbb{R}^{3}$.

  Calcule ${\left[T\right]}_{\beta}^{\alpha}$,
  ${\left[T\right]}_{\beta}^{e_{1}}$ e
  ${\left[T\right]}_{e_{2}}^{\alpha}$ onde $e_{1}$ é a base canônica
  de $\mathbb{R}^{2}$ e $e_{2}$ é a base canônica de
  $\mathbb{R}^{3}$.

  \begin{solutionordottedlines}
  \end{solutionordottedlines}

  \question\label{exercício:2.16}

  Sejam
  $T\colon\mathbb{R}^{3}\to\mathcal{P}_{2}\left(\mathbb{R}\right)$ e
  $G\colon\mathcal{P}_{2}\left(\mathbb{R}\right)\to\mathbb{R}^{3}$
  transformações lineares tais que
  \[
    {\left[T\right]}_{\beta}^{\alpha}=
    \begin{bmatrix*}
      1 & 2 & -1 \\
      1 & 0 & -1 \\
      0 & -1 & 0
    \end{bmatrix*}\quad
    \text { e }\quad
    {\left[G\right]}_{\alpha}^{\beta}=
    \begin{bmatrix*}
      1 & 1 & 2 \\
      1 & -1 & 0 \\
      -1 & 1 & 2
    \end{bmatrix*}
  \]
  onde
  \begin{math}
    \alpha=
    \left\{
    \left(1,1,0\right),
    \left(0,1,0\right),
    \left(0,0,1\right)
    \right\}
  \end{math}
  é base de $\mathbb{R}^{3}$ e $\beta=\left\{1,1+x,1+x^{2}\right\}$ é
  base de $\mathcal{P}_{2}\left(\mathbb{R}\right)$.
  Determine bases para
  \begin{math}
    \operatorname{Nuc}
    \left(T\right),
    \operatorname{Im}
    \left(T\right),
    \operatorname{Nuc}
    \left(G\circ T\right)
  \end{math}
  e $\operatorname{Im}\left(G\circ T\right)$.

  \begin{solutionordottedlines}
  \end{solutionordottedlines}

  \question\label{exercício:2.17}

  Seja
  \begin{math}
    T\colon
    \mathcal{M}_{2\times2}
    \left(\mathbb{C}\right)\to
    \mathcal{M}_{2\times2}
    \left(\mathbb{C}\right)
  \end{math}
  a transformação linear definida por
  \[
    T
    \begin{bmatrix*}
      x & y \\
      z & w
    \end{bmatrix*}
    =
    \begin{bmatrix*}
      0 & x \\
      z-w & 0
    \end{bmatrix*}.
  \]

  \begin{parts}
    \part\label{exercício:2.17a}

    Determine a matriz $\left[T\right]$ de $T$ com relação à base canônica $e$;

    \part\label{exercício:2.17b}

    Determine a matriz de $T$ com relação à base
    \[
      \alpha=
      \left\{
      \begin{bmatrix*}
        1 & 0 \\
        0 & 1
      \end{bmatrix*},
      \begin{bmatrix*}
        0 & 1 \\
        1 & 0
      \end{bmatrix*},
      \begin{bmatrix*}
        1 & 0 \\
        1 & 1
      \end{bmatrix*},
      \begin{bmatrix*}
        0 & 1 \\
        0 & 1
      \end{bmatrix*}
      \right\};
    \]

    \part\label{exercício:2.17c}

    Exiba a matriz $M$ tal que ${\left[T\right]}_{\beta}=M^{-1}\left[T\right]M$.
  \end{parts}

  \begin{solutionordottedlines}
    \begin{parts}
      \part


      \part


      \part

    \end{parts}
  \end{solutionordottedlines}

  \question\label{exercício:2.18}

  Seja
  \begin{math}
    T\colon
    \mathcal{M}_{2\times2}
    \left(\mathbb{Z}/7\mathbb{Z}\right)\to
    \mathcal{M}_{2\times2}
    \left(\mathbb{Z}/7\mathbb{Z}\right)
  \end{math}
  a transformação linear definida por
  \[
    T
    \begin{bmatrix*}
      x & y \\
      z & w
    \end{bmatrix*}
    =
    \begin{bmatrix*}
      \overline{0} & x \\
      z+\overline{6}w & \overline{0}
    \end{bmatrix*}.
  \]

  \begin{parts}
    \part\label{exercício:2.18a}

    Determine a matriz $[T]$ de $T$ com relação à base canônica $e$;

    \part\label{exercício:2.18b}

    Determine a matriz de $T$ com relação à base

    $$
      \alpha=\left\{
      \begin{bmatrix*}
        \overline{1} & \overline{0} \\
        \overline{0} & \overline{1}
      \end{bmatrix*},
      \begin{bmatrix*}
        \overline{0} & \overline{1} \\
        \overline{1} & \overline{0}
      \end{bmatrix*},
      \begin{bmatrix*}
        \overline{1} & \overline{0} \\
        \overline{1} & \overline{1}
      \end{bmatrix*},
      \begin{bmatrix*}
        \overline{0} & \overline{1} \\
        \overline{0} & \overline{1}
      \end{bmatrix*}.
      \right\}
    $$

    \part\label{exercício:2.18c}

    Exiba a matriz $M$ tal que ${\left[T\right]}_{\beta}=M^{-1}\left[T\right]M$.
  \end{parts}

  \question\label{exercício:2.19}

  Seja
  \begin{math}
    T\colon
    \mathcal{M}_{2\times2}
    \left(\mathbb{Z}/7\mathbb{Z}\right)\to
    \mathcal{M}_{2\times2}
    \left(\mathbb{Z}/7\mathbb{Z}\right)
  \end{math}
  a transformação linear definida por
  $$
    T
    x y \\
    z w
    =
    \overline{0}  x \\
    z+\overline{6} w \overline{0}
  $$

  \begin{parts}
    \begin{multicols}{3}
      \part\label{exercício:2.19a}

      Determine a matriz $\left[T\right]$ de $T$ com relação à base
      canônica $e$;

      \part\label{exercício:2.19b}

      Determine a matriz de $T$ com relação à base
      $$
        \alpha=
        \left\{
        \overline{1} \overline{0} \\
        \overline{0} \overline{1}
        ,
        \overline{0} \overline{1} \\
        \overline{1} \overline{0}
        ,
        \overline{1} \overline{0} \\
        \overline{1} \overline{1}
        \overline{0} \overline{1} \\
        \overline{0} \overline{1}
        \right\}
      $$

      \part\label{exercício:2.19c}

      Exiba a matriz $M$ tal que
      ${\left[T\right]}_{\beta}=M^{-1}\left[T\right]M$.
    \end{multicols}
  \end{parts}

  \begin{solutionordottedlines}
    \begin{parts}
      \part


      \part


      \part

    \end{parts}
  \end{solutionordottedlines}

  \question\label{exercício:2.20}

  Seja $T\colon\mathbb{Q}^{3}\to\mathbb{Q}^{3}$ uma transformação linear
  cuja matriz com relação à base canônica seja
  $$
    1 1 0 \\
    -1 0 1 \\
    0 -1 -1
  $$

  \begin{parts}
    \begin{multicols}{3}
      \part\label{exercício:2.20a}

      Determine $T\left(x,y,z\right)$;

      \part\label{exercício:2.20b}

      Qual é a matriz do operador linear $T$ com relação à base
      \begin{math}
        \alpha=
        \left\{
        \left(-1,1,0\right),
        \left(1,-1,1\right),
        \left(0,1,-1\right)
        \right\}
      \end{math}?

      \part\label{exercício:2.20c}

      O operador $T$ é invertível? Justifique!
    \end{multicols}
  \end{parts}


  \begin{solutionordottedlines}
    \begin{parts}
      \part


      \part


      \part

    \end{parts}
  \end{solutionordottedlines}
\end{questions}

\setpartpreamble{%
  \begin{center}
    \includesvg[width=.4\paperwidth]{julia}
  \end{center}
}
\part{Tutorial}

\appendix

\chapter{\texttt{LinearAlgebra} from Julia}

\begin{listing}[H]
  \footnotesize
  %\inputminted{c}{exercise3_1.c}
  \begin{minted}[autogobble]{julia}
    f(x) = x.^2 + π
    const ⊗ = kron
    const Σ = sum # Although `sum` may be just as good in the code.
    # Calculate Σ_{j=1}^5 j^2
    Σ([j^2 for j ∈ 1:5])
    \end{minted}
  \caption{Programa \texttt{exercise3\_1.c}.}
  \label{lst:3.1}
\end{listing}


\printindex

\end{document}