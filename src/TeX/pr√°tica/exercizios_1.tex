\chapter{Exercícios de Fixação\quad$\left(08/01/2021\right)$}

\begin{questions}
  \question\label{exercício:1.1}

  Seja $\mathbb{F}$ um corpo.
  Dizemos que um subconjunto $\mathbb{K}$ de $\mathbb{F}$ é um
  subcorpo de $\mathbb{F}$ se $\mathbb{K}$ munido das operações de
  adição e multiplicação de $\mathbb{F}$ é um corpo.
  Mostre que os seguintes subconjuntos são subcorpos de $\mathbb{C}$.

  \begin{parts}
    \begin{multicols}{3}
      \part\label{exercício:1.1a}

      \begin{math}
        \mathbb{Q}
        \left(\sqrt{3}\right)=
        \left\{
        a+b\sqrt{3}\mid a,b\in\mathbb{Q}
        \right\}
      \end{math};

      \part\label{exercício:1.1b}

      \begin{math}
        \mathbb{Q}
        \left(i\right)=
        \left\{
        a+bi\mid a,b\in\mathbb{Q}
        \text{ e }i^{2}=-1
        \right\}
      \end{math};

      \part\label{exercício:1.1c}

      \begin{math}
        \mathbb{Q}
        \left(i\sqrt{2}\right)=
        \left\{
        a+bi\sqrt{2}\mid a,b\in\mathbb{Q}
        \text{ e }i^{2}=-1
        \right\}
      \end{math}.
    \end{multicols}
  \end{parts}

  \begin{solutionordottedlines}
    \begin{parts}
      \part
      .

      \part

      .

      \part

      .
    \end{parts}
  \end{solutionordottedlines}

  \question\label{exercício:1.2}

  Mostre que:

  \begin{parts}
    \part\label{exercício:1.2a}

    Todo subcorpo de $\mathbb{C}$ tem $\mathbb{Q}$ como subcorpo;

    \part\label{exercício:1.2b}

    Todo corpo de característica $0$ tem uma cópia de $\mathbb{Q}$;

    \part\label{exercício:1.2c}

    Se $\mathbb{K}$ contém propriamente $\mathbb{R}$ e é um subcorpo de
    $\mathbb{C}$, então $\mathbb{K}=\mathbb{C}$.

  \end{parts}

  \begin{solutionordottedlines}
    \begin{parts}
      \part
      .

      \part

      .

      \part

      .
    \end{parts}
  \end{solutionordottedlines}

  \question\label{exercício:1.3}

  Considere o corpo finito com $5$ elementos
  \begin{math}
    \mathbb{Z}/5\mathbb{Z}=
    \left\{
    \overline{0},
    \overline{1},
    \overline{2},
    \overline{3},
    \overline{4}
    \right\}
  \end{math}.

  \begin{parts}
    \part\label{exercício:1.3a}

    Mostre que
    \[
      \mathbb{F}=
      \left\{
      a+bi\mid a,b\in\mathbb{Z}/5\mathbb{Z}
      \text { e }
      i^{2}=
      \overline{3}
      \right\}
    \]

    munido das operações

    \[
      \begin{aligned}
        \overline{+}\colon\mathbb{F}\times\mathbb{F}     & \longrightarrow\mathbb{F}     \\
        \left(\left(a+bi\right),\left(c+di\right)\right) & \longmapsto \left(a+c\right)+
        \left(b+d\right)i
      \end{aligned}\quad
      \begin{aligned}
        \overline{\cdot}\colon\mathbb{F}\times\mathbb{F}  & \longrightarrow\mathbb{F}                   \\
        \left(\left(a+b i\right),\left(c+di\right)\right) & \longmapsto \left(ac+\overline{3}bd\right)+
        \left(ad+bc\right)i
      \end{aligned}
    \]

    é um corpo com 25 elementos;

    \part\label{exercício:1.3b}

    Mostre que $\mathbb{Z}/5\mathbb{Z}$ é um subcorpo de $\mathbb{F}$.
    Qual é a característica de $F$?
  \end{parts}

  \begin{solutionordottedlines}
    \begin{parts}
      \part

      .

      \part

      .
    \end{parts}
  \end{solutionordottedlines}

  \question\label{exercício:1.4}

  Determine o conjunto solução de cada sistema linear dado.

  \begin{parts}%\setlength{\columnsep}{-20pt}
    \begin{multicols}{2}
      \part\label{exercício:1.4a}

      \begin{math}
        \csysteme[xyzw]{
          x-2y+z+w=1,
          2x+y-z=3,
          2x+y-5z+w=4
        }
      \end{math}
      em $\mathbb{R}$,

      \part\label{exercício:1.4b}

      \begin{math}
        \csysteme[xyzw]{
          x-\sqrt{3}y+z+w=1+\sqrt{3},
          \left(2\+\sqrt{3}\right)x+y-z=3,
          2x+y-\left(1\-\sqrt{3}\right)z+w=4
        }
      \end{math}
      em $\mathbb{Q}\left(\sqrt{3}\right)$,

      \part\label{exercício:1.4c}

      \begin{math}
        \csysteme[xyzw]{
          x-2iy+2z-w=0,
          \left(2\+i\right)x+z+w=0,
          2ix+y-5z+\left(1\+i\right)w=0
        }
      \end{math}
      em $\mathbb{C}$,

      \part\label{exercício:1.4d}

      \begin{math}
        \csysteme[xyzw]{
          x-\overline{2}y+\overline{2}z-w=\overline{0},
          \overline{2}x+z+w=\overline{0},
          \overline{2}x+y-\overline{3}z+w=\overline{0}
        }
      \end{math}
      em $\mathbb{Z}/5\mathbb{Z}$,

      \part\label{exercício:1.4e}

      \begin{math}
        \csysteme[xyzw]{
          x-\overline{2}iy+\overline{2}z-w=\overline{0},
          \left(\overline{2}\+i\right)x+z+w=\overline{0},
          \overline{2}ix+y-\overline{3}z+\left(\overline{1}\+i\right)w=\overline{0}
        }
      \end{math}
      em $\mathbb{F}$ de (a) da questão 3.
    \end{multicols}
  \end{parts}%\setlength{\columnsep}{10pt}

  \begin{solutionordottedlines}
    \begin{parts}
      \part

      .

      \part

      .

      \part

      .

      \part

      .

      \part

      .
    \end{parts}
  \end{solutionordottedlines}

  \question\label{exercício:1.5}

  Mostre que se dois sistemas lineares $2\times2$ possuem o mesmo
  conjunto solução, então eles são equivalentes.
  Determine, se existir, dois sistemas lineares $2\times3$ com mesmo
  conjunto solução mas não equivalentes.

  \begin{solutionordottedlines}
  \end{solutionordottedlines}

  \question\label{exercício:1.6}

  Considere o sistema linear sobre $\mathbb{Q}$

  \[
    \csysteme[xyzw]{
      x-2y+z+2w=1,
      x+y-z+w=2,
      x+7y-5z-w=3
    }
  \]

  Mostre que esse sistema não tem solução.

  \begin{solutionordottedlines}
  \end{solutionordottedlines}

  \question\label{exercício:1.7}

  Determine todos $a,b,c,d\in\mathbb{R}$ tais que o sistema linear

  \[
    \begin{bNiceMatrix}
      3  & -6 & 2 & 1 \\
      -2 & 4  & 1 & 3 \\
      0  & 0  & 1 & 1 \\
      1  & -2 & 1 & 0
    \end{bNiceMatrix}
    \cdot
    \begin{bNiceMatrix}
      x \\
      y \\
      z \\
      w
    \end{bNiceMatrix}=
    \begin{bNiceMatrix}
      a \\
      b \\
      c \\
      d
    \end{bNiceMatrix}
  \]
  tem solução.

  \begin{solutionordottedlines}
  \end{solutionordottedlines}

  \question\label{exercício:1.8}

  Encontre duas matrizes $A$ e $B$ de ordens iguais a $3\times3$ tais
  que $AB$ é uma matriz nula mas $BA$ não é.

  \begin{solutionordottedlines}
  \end{solutionordottedlines}

  \question\label{exercício:1.9}

  Mostre que toda matriz elementar é inversível e calcule a inversa
  de cada tipo.

  \begin{solutionordottedlines}
  \end{solutionordottedlines}

  \question\label{exercício:1.10}

  Determine a matriz inversa da matriz

  \[
    A=
    \begin{bNiceMatrix}
      1 & 2 & 3 & 4 \\
      0 & 2 & 3 & 4 \\
      0 & 0 & 3 & 4 \\
      0 & 0 & 0 & 4
    \end{bNiceMatrix}.
  \]

  \begin{solutionordottedlines}
  \end{solutionordottedlines}


  \question\label{exercício:11.1}

  Considere a matriz

  \NiceMatrixOptions{
    cell-space-top-limit = 3pt,
    %cell-space-bottom-limit = 2pt
  }

  \[
    A=
    \begin{bNiceMatrix}
      \overline{1} & \overline{2} & \overline{3} & \overline{4} \\
      \overline{0} & \overline{2} & \overline{3} & \overline{4} \\
      \overline{0} & \overline{0} & \overline{3} & \overline{4} \\
      \overline{0} & \overline{0} & \overline{0} & \overline{4}
    \end{bNiceMatrix}
  \]

  \NiceMatrixOptions{
    cell-space-top-limit = 0pt,
  }

  com entradas no corpo com cinco elementos
  \begin{math}
    \mathbb{Z}/5\mathbb{Z}=
    \left\{
    \overline{0},
    \overline{1},
    \overline{2},
    \overline{3},
    \overline{4}
    \right\}
  \end{math}.
  Calcule sua inversa.

  \begin{solutionordottedlines}
  \end{solutionordottedlines}

  \question\label{exercício:1.12}

  Considere a matriz
  \[
    A=
    \begin{bNiceMatrix}
      1  & 2  & 1 & 0 \\
      -1 & 0  & 3 & 5 \\
      1  & -2 & 1 & 1
    \end{bNiceMatrix}.
  \]
  Encontre uma matriz na forma e uma matriz invertível $P$ tal que
  $R=PA$.

  \begin{solutionordottedlines}
  \end{solutionordottedlines}
\end{questions}
